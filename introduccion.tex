\chapter*{Introducción}

\par Los hornos de proceso son los equipos que suministran la mayor cantidad de energía térmica a los fluidos de trabajo en procesos de refinación, mejoramiento y producción, dentro del sector petrolero y petroquímico \cite{bib:sandoval}. Desde este punto de vista, la eficiencia térmica de cada horno es una variable importante a considerar para reducir el consumo de combustibles y minimizar las emisiones de gases de combustión.

\par Por lo tanto, optimizar la eficiencia de los hornos se traduce en menores costos operacionales y en la extensión de la vida útil del equipo, crucial para la economía de la refinería ya que son equipos de alto costo que trabajan con llamas a muy altas temperaturas ($\approx$2000 °C) y cuyas fallas suelen ser críticas para el proceso y los operadores \cite{bib:economics}.

\par Los hornos consumen grandes sumas de combustibles fósiles, cientos de kilogramos por hora por cada megavatio cedido al fluido de proceso, y emiten cantidades proporcionales de dióxido de carbono a la atmósfera. Hay que resaltar que el dióxido de carbono atmosférico promedio mundial, en 2019, fue de 409,8 ppm, como resultado de un aumento del 3\% anual en los 5 años previos, siendo este valor el más alto registrado históricamente \cite{bib:clima}.
\par Teniendo como referencia las cifras de las emisiones de gases invernadero del año 2018 reportadas por la Organización de Naciones Unidas, se arrojaron al ambiente 55,3 GtCO$_2$eq (giga toneladas de dióxido de carbono equivalente)\footnote{El equivalente de dióxido de carbono (CO$_2$eq), es una medida en toneladas de la huella de carbono. Huella de carbono es el nombre dado a la totalidad de la emisión de gases de efecto invernadero.} \cite{onu}, la Agencia Internacional de Energía (IEA) prevé un aumento del 5,6\% para el año 2021, cifra que se podrá corroborar al cerrar los registros de consumo durante el año en curso; lo que contrarrestaría la contracción del 4\% del año 2020, luego del Covid-19, generando un nuevo valor máximo histórico de emisiones de CO$_2$ \cite{bib:iea}.

\par Lamentablemente, no es posible automatizar la totalidad de la operación de un horno de procesos \cite{bib:instrumentacion}, puesto que no es un equipo completamente hermético como un intercambiador de calor cerrado, siendo afectado por infiltraciones de aire que no participan en la combustión y afectan las mediciones de O$_2$, también, por cambios en el clima, por ráfagas de viento o incluso por lluvias; al usar aire ambiental para la combustión.
\par Esta condición, es decir, estar ``abiertos a la atmósfera'', hace recaer sobre los operadores de sala de control y de campo (la interfaz humano-horno) la responsabilidad última sobre el proceso de combustión y el aprovechamiento del combustible.

\par Por experiencia práctica dentro de las refinerías, se ha identificado que los operadores suelen manejar estos equipos con altos niveles de exceso de aire \cite{bib:thermox}. Estos valores en la operación están alejados de las condiciones recomendadas al operar el horno, lo cual provoca un mayor consumo de combustible para compensar el calor perdido al calentar el aire en exceso. Adicionalmente, provoca alteraciones en las características de las llamas, reduciendo su tamaño y causando posibles separaciones de las llamas y los quemadores, disminuyendo, la eficiencia térmica del proceso y aumentando las emisiones de dióxido de carbono.

\par Dada esta situación, y con la intención de mejorar la competencia de los ingenieros y operadores de los hornos de proceso mediante una herramienta moderna de aprendizaje, se desarrolló este proyecto de grado que utiliza ecuaciones de balance de masa y transferencia de calor para simular las condiciones de operación a través de un algoritmo capaz de responder a diferentes modificaciones en las variables de entrada del proceso, mediante una interfaz gráfica de uso amigable.