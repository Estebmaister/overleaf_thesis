\chapter*{Introducción}

\par En refinerías de petróleo crudo, plantas químicas y petroquímicas, los hornos representan los equipos que suministran más del 80\% de la totalidad de la energía requerida para los procesos de separación y conversión química de los productos. Desde este punto de vista, la eficiencia energética de cada horno es una variable crítica a optimizar, con el fin de reducir el consumo de combustibles quemados, minimizar las emisiones de gases de combustión que, finalmente, se traducen en menores costos operativos [1].

\par Los \texttt{hornos de procesos} juegan un rol primordial en una refinería. Consumen altísimas cantidades de combustibles fósiles ({\tt miles de kilogramos por hora}) y emiten proporcionales cantidades de \ac{co2} a la atmósfera. Son equipos de alto costo que trabajan con llamas a muy altas temperaturas (\textgreater 2000 °C) y cuyas fallas suelen ser críticas para la economía de la refinería. Es importante destacar que el dióxido de carbono atmosférico promedio mundial en 2019 fue de 409,8 ppm, con un rango de incertidumbre de 0,1 ppm. Lo que indica que los niveles de dióxido de carbono resultaron los más altos de los últimos 800.000 años [2]. Teniendo como referencia las cifras del 2018 reportadas por la \ac{ONU}, 55,3 Gt\ac{co2}e (giga toneladas de \ac{co2} equivalente) [3], se puede apreciar el impacto significativo que esto genera sobre el medio ambiente. 

\par Lamentablemente, no es posible automatizar la totalidad de la operación de un horno de procesos [4] puesto que no son equipos completamente herméticos como un intercambiador de calor o una columna de destilación. Esta condición, es decir, estar "abiertos a la atmósfera", hace recaer sobre los operadores de sala de control y de campo (la interfase humano-horno), la responsabilidad última sobre el proceso de combustión y el aprovechamiento del combustible.  

\par Por experiencias prácticas dentro de las refinerías, se ha identificado que los operadores suelen manejar estos equipos con altos niveles de tiro y de exceso de aire. Estas variables de operación están alejadas de las condiciones recomendadas al diseñar el horno pues traen consigo un mayor consumo del combustible para compensar el calor perdido al calentar el aire en exceso, lo que a su vez provoca alteraciones en las características de las llamas, haciéndolas incidir directamente sobre los tubos de la sección radiante, afectando la integridad mecánica del equipo y causando la coquización localizada de la carga dentro de los tubos, disminuyendo, aún más, la eficiencia del proceso y aumentando las emisiones de CO2.

\par Dada esta problemática, y en la búsqueda de mejorar la eficiencia en los hornos de proceso, este proyecto utilizará ecuaciones de balance de masa y transferencia de calor para simular las condiciones de operación a través de un algoritmo capaz de responder a diferentes estímulos en las variables dominantes del proceso, mediante una interfaz gráfica de uso amigable para los usuarios.