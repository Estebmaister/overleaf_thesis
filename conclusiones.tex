\chapter*{Conclusiones}

\begin{itemize}
    \item Se desarrollo exitosamente un modelo capaz de simular la operación de hornos de proceso, enfocado al adiestramiento de ingenieros y operadores, y con el objetivo de aumentar la eficiencia de estos equipos para disminuir el consumo de combustibles fósiles y generar menos gases de efecto invernadero.

    \item El modelo integra las ecuaciones de transferencia de calor y conservación de masa y energía en un algoritmo que simula el comportamiento de un horno, se comprobó su estabilidad para reportar los resultados del proceso en el rango de las variables de operación establecidas.
    
    \item El simulador para adiestramiento de hornos de proceso es una herramienta de disposición pública y de código abierto, que refiere parámetros confiables y prácticos para el uso académico o industrial dentro de las características mecánicas establecidas. El acceso directo se encuentra siguiendo el enlace (\url{https://e-usb.github.io/heater}).
    
    \item Fue posible validar los resultados de las simulaciones mediante comparaciones con programas comerciales, e igualmente, las tendencias de estos resultados mostraron coherencia física en cada caso planteado dentro de las condiciones de diseño.
    
    \item Como todo simulador, siempre existe la posibilidad de aumentar su alcance, por ejemplo, se pueden incluir ecuaciones empíricas para calcular la formación de CO y NOx, o adicionalmente, desarrollar un modelo que relacione la posición del dámper con la entrada del aire al al horno.
\end{itemize}