\chapter*{Conclusiones}

%\begin{itemize}
    %\item 
    \par Se desarrolló satisfactoriamente un modelo capaz de simular la operación de hornos de proceso, enfocado al adiestramiento de ingenieros y operadores, y con el objetivo de aumentar la eficiencia de estos equipos para disminuir el consumo de combustibles fósiles y generar menos gases de efecto invernadero.

    %\item 
    \par El modelo integra las ecuaciones de transferencia de calor y conservación de masa y energía en un algoritmo que simula el comportamiento estacionario de un horno de fuego directo, se comprobó su capacidad para generar resultados estables y confiables de las variables de proceso establecidas.
    
    %\item 
    \par El simulador para adiestramiento de hornos de proceso es una herramienta computacional de disposición pública y de código abierto, escrita en el lenguaje de programación JavaScript, que refiere parámetros confiables y prácticos para el uso académico o industrial dentro de las características mecánicas establecidas. El acceso directo se encuentra siguiendo el enlace (\url{https://e-usb.github.io/heater}).
    
    %\item 
    \par Los resultados generados por el simulador fueron validados aceptablemente mediante su comparación con un programa comercial empleado para el diseño y evaluación termomecánica de hornos de proceso. Las tendencias de las variables operacionales simuladas mostraron resultados totalmente coherentes con la operación real de estos hornos en refinerías y plantas petroquímicas.

    %\item 
    \par Se incrementó el alcance del algoritmo desarrollado con la implementación de un modo comparativo y un modo de visualización de tendencias, aumentando las opciones de los usuarios al interactuar con el simulador.
%\end{itemize}