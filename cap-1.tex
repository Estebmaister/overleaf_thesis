\chapter{Antecedentes}

\par En este capítulo se desarrollan los puntos que respaldan este proyecto, como la justificación y objetivos. Adicionalmente se describen los proyectos que anteceden a la investigación y que se usaron de referencia.

\section{Justificación}

\par En virtud de la dependencia de los hornos de proceso de la intervención continua y acertada de sus operadores, se hace evidente la obligación de fomentar la capacitación de dichos operadores como una manera de incrementar el desempeño y la eficiencia de estos equipos; lo que lograría disminuir la cantidad de residuos que se producen al medio ambiente y anticipar resultados a cambios en el proceso de una manera controlada.

\par Adicionalmente, para un estudiante o ingeniero que intente entrar en esta área por primera vez los recursos para experimentar son limitados, dígase privados o de difícil uso.

\par Por esta razón, se ha considerado necesario construir un simulador de hornos de proceso, con base en los estándares API 560\cite{bib:api560} y API RP 535, para su utilización como herramienta práctica en la capacitación del personal operativo de las refinerías de petróleo. Esta herramienta permitiría, de una manera más didáctica, observar e interactuar con las variables fundamentales de control del horno, a saber, el exceso de aire, tipo de combustibles, cantidad de fluido y sus temperaturas de entrada y salida, para controlar los efectos de las operaciones alejadas de las condiciones de diseño.

\section{Objetivos}

\subsection{Objetivo General}

\par Diseñar un simulador web de hornos de proceso, que pueda ser fácilmente utilizado para la enseñanza del manejo de estos equipos al permitir visualizar los efectos de alterar las variables que gobiernan el proceso.

\subsection{Objetivos específicos}

\par Incorporar en a una herramienta práctica, de aplicaciones académica y/o industrial, los conocimientos de ingeniería química relacionados con hornos de proceso y todos los fenómenos de transferencia de calor asociados a estos equipos.

\par Diseñar una herramienta para la enseñanza y uso en la industria de los hornos de proceso, mediante la cual se simulen, a través de algoritmos y ecuaciones matemáticas, los procesos asociados a estos equipos.

\par Integrar la herramienta creada a navegadores en línea, tomando ventaja del desarrollo de tecnologías web libres, que permitan el acceso desde cualquier dispositivo con internet y aumenten el alcance de dicha herramienta.

\section{Proyectos Previos}

\par Existen investigaciones enfocadas a la optimización de hornos de proceso particulares\cite{bib:leti} con características y condiciones definidas, pero no se ha encontrado un estudio que pretenda ir más allá e intente analizar condiciones distintas a las de diseño que también suelen verse en la práctica.

\par Sí hay simuladores privados capaces de otorgar un resultados en un formato de reporte, como WinHeat\textsuperscript{\textcopyright}, que contando con licencias otorgadas por los tutores, pudieron ser usados como referencia en el desarrollo del proyecto.

\par Lo innovador en esta investigación es hacer el objetivo del simulador un aporte educacional y no una herramienta de diseño.