\chapter{Antecedentes}

\par En este capítulo se desarrollan los puntos que respaldan este proyecto, como la justificación y objetivos. Adicionalmente se describen algunos proyectos que anteceden a la investigación y que se usaron como referencias.

\section{Justificación}

\par En virtud de que los hornos de proceso requieren de la intervención continua y acertada de sus operadores, se hace evidente la necesidad de fortalecer la capacitación de dichos operadores como una manera de incrementar la eficiencia en las condiciones de operación de estos equipos, lo que lograría disminuir la cantidad de emisiones que se arrojan al medio ambiente, disminuir el consumo de combustible, prolongar la vida útil del equipo y anticipar cambios en el proceso de una manera controlada.
\par Adicionalmente, para un estudiante o ingeniero que intente formarse en esta área los recursos para experimentar son limitados.
\par Por esta razón, se consideró necesario diseñar y desarrollar un simulador de hornos de proceso con base en los estándares API 560 \cite{bib:api560} y API RP 535 \cite{bib:api535}, para su utilización como herramienta práctica en la capacitación del personal operativo de las refinerías de petróleo, ya que las opciones de simuladores existentes no son abiertas al público y solo están enfocadas al diseño de hornos.
\par Esta herramienta debe permitir, de una manera didáctica, observar e interactuar con las variables fundamentales de control del horno, a saber: el exceso de aire en la combustión, la composición del combustible, la carga de flujo a manejar y las temperaturas de entrada y salida del fluido al horno; con el fin de simular y comparar los efectos de operar el horno alejado de las condiciones óptimas de diseño.

\section{Objetivos}
\par Con las consideraciones antes expuestas, el presente proyecto de grado se plantea el cumplimiento de los siguientes objetivos:

\subsection{Objetivo General}
\par Desarrollar un simulador web de hornos de proceso de tiro natural que pueda ser fácilmente utilizado para la enseñanza del manejo de estos equipos, al permitir visualizar los efectos de modificar los valores de las variables que gobiernan el proceso y su importancia en la eficiencia del equipo.

\subsection{Objetivos específicos}
\par Aplicar los diferentes conceptos asociados a la operación de hornos de proceso.
\par Aplicar las ecuaciones de balance de masa y energía, y las ecuaciones de transferencia de calor en la construcción del modelo matemático para el cálculo de las condiciones de operación de los hornos de proceso.
\par Diseñar una herramienta para la enseñanza de la operación de los hornos de proceso, usando un algoritmo basado en el modelo matemático anterior, con una interfaz de entrada y salida de datos amigable para los usuarios, con la que se simulen las modificaciones de las condiciones de proceso de un horno.
\par Desarrollar la herramienta mencionada e integrarla a una interfaz en línea, tomando ventaja del desarrollo de tecnologías web libres, que permita el acceso desde cualquier dispositivo con internet y aumente el alcance de dicha herramienta.

\section{Proyectos Previos}

\par Existen investigaciones enfocadas en la optimización de hornos de proceso particulares con características y condiciones definidas \cite{bib:leti}; también existen distintos simuladores enfocados en el diseño de hornos, pero no se ha encontrado un estudio con un alcance mayor, que intente analizar todo el comportamiento del horno en condiciones distintas a las de diseño, que también son observadas en la práctica y responden a un carácter académico.
\par También existen simuladores comerciales con enfoques orientados al diseño, pero ninguno con fines explícitamente educativos.
\par Lo innovador de este proyecto consiste en hacer como objetivo del simulador un aporte educacional y no una herramienta de diseño, con muchas aplicaciones que, además, pueden ampliarse si se continua desarrollando la interfaz con el algoritmo de base.