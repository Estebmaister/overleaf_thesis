\chapter{Antecedentes}

\par En este capítulo se desarrollan los puntos que respaldan este proyecto, como la justificación y objetivos. Adicionalmente se describen los proyectos que anteceden a la investigación y que se usaron de referencia.

\section{Justificación}

¿Que es lo que pasa con los proceso de horno?

¿Porque necesitaríamos una herramienta?

\par Simular procesos
\par Evaluar diferentes escenarios para tomar mejores decisiones

Buscando eficiencia, disminuir la cantidad de residuos que produce al medio ambiente el proceso. Anticipar resultados de una manera controlada

\section{Objetivos}

\par Llevar a una herramienta práctica de aplicaciones académica y/o industrial los conocimientos de ingeniería química relacionados con hornos de procesos y todos los fenómenos asociados.

Integrar los conocimientos relacionados con la ingeniería química y el uso de  herramientas y aplicaciones para ser usados en la optimización de procesos de hornos

Crear, proveer una herramienta útil para la enseñanza, aplicaciones y uso industrial mediante la cual se simulen de forma controlada los procesos  asociados a hornos industriales.
Desarrollar una herramienta práctica de aplicación académica y/o industrial que permita optimizar el proceso de horno.

Entrar en la nueva era tecnológica usando herramientas como las aplicaciones para simular procesos de hornos que permiten no solo capacitación y aprendizaje sino además controlar y disminuir los residuos generados.

\section{Proyectos Previos}

Que herramientas parecidas, simuladores, que más hay parecido

Que sería lo diferente de tu proyecto

- El objetivo del modelo es educacional y no de diseño.
