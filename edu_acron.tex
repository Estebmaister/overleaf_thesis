\chapter{SOBRE EL USO DE ACR\'ONIMOS Y LA LISTA DE S\'IMBOLOS}
\section{Acr\'onimos}
En este cap\'itulo se describe una forma de crear los acr\'onimos compatiblemente con las normas de los decanatos. Su uso es opcional pero recomendado.  

El uso de los acronimos se hace a trav\'es del paquete {\sl acronym} (que debe ser cargado en el prea\'ambulo) y es habilitado
con el comando \verb+\useacronyms+ al principio del archivo (luego de los \'indices durante \verb+\frontmatter+). Los
acr\'onimos se definen {\em todos} en el archivo {\tt acronimos.tex} bajo el ambiente \verb+\begin{acronym}+\verb+\end{acronym}+. Para definir un acr\'onimo se usa el comando \verb+\acro{}[]{}+. La primera entrada es el nombre del acr\'onimo, la segunda entrada es el acr\'onimo propiamente y la tercera entrada es la expansi\'on del acr\'onimo. Por ejemplo, lo siguiente es el contenido del archivo \texttt{acronimos.tex} donde se crean algunas definiciones de acr\'onimos.
\begin{verbatim}
\chapter*{LISTA DE ACR\'ONIMOS}
\begin{acronym}
\acro{USB}[USB]{Universidad Sim\'on Bol\'ivar}
\acro{CCE}[Dpto.~CCE]{Departamento de C\'omputo Cient\'ifico 
     y Estad\'isitica}
\acro{DEP}[DEP]{Decanato de Estudios Profesionales}
\acro{PDF}[PDF]{Documento en Formato Portable\copyright}
\acro{PS}[PS]{PostScript\copyright}
\end{acronym}
\end{verbatim}
Una vez definidos los acr\'onimos se pueden referenciar con el comando \verb+\ac{}+ usando el nombre definido para el acr\'onimo. Por
ejemplo, \verb+\ac{USB}+ producir\'a \ac{USB}, mientras que \verb+\ac{CCE}+ producir\'a \ac{CCE}.

La clase {\sl tesis-usb} en acorde a las disposiciones del \ac{DEP}, despliega la descripci\'on
de cada acr\'onimo una sola vez cuando es referenciada la primera vez en cada cap\'itulo. Todo
los usos sucesivos no son expandidos, por ejemplo:

\noindent \verb+\ac{LI}+$\rightarrow$ \ac{LI}\\
\verb+\ac{LI}+$\rightarrow$ \ac{LI}\\
\verb+\ac{LI}+$\rightarrow$ \ac{LI}\\
\verb+\ac{LI}+$\rightarrow$ \ac{LI}\\
\verb+\ac{CCE}+$\rightarrow$ \ac{CCE}

Nota: Si el comando \verb+\useacronyms+ est\'a comentado al principio de {\tt main.tex} y se
usan acr\'onimos a lo largo del libro, estos no van a funcionar resultando en errores de compilación.

\section{Lista de s\'imbolos}
La lista de s\'imbolos o notaci\'on matem\'atica se recomienda hacer manualmente. Por ejemplo, se puede incluir el siguiente c\'odigo luego de los \'indices.
\begin{verbatim}
\chapter*{Notaci\'on matem\'atica}
\begin{tabular}{ll}
$\mathbb{R}$ & Conjunto de n\'umeros reales\\
$M_{m,n}$ & Espacio de las matrices de tama\~no $m$ por 
     $n$ con entradas reales\\
$\mathcal{L}$ & Operador de Laplace\\
$\emptyset$ & Conjunto vac\'io
\end{tabular}
\end{verbatim}