\documentclass[pregrado]{tesis-usb}

% paquetes
\usepackage[utf8]{inputenc}
\usepackage{verbatim}
\usepackage{acronym}
\usepackage{amsmath}
\usepackage{amsfonts}
\usepackage{amssymb}
\usepackage{enumitem}
\usepackage{multirow}
\usepackage{array}
% \usepackage{hyperref}

% estilo de las referencias
\usepackage[fixlanguage]{babelbib-and}\selectbiblanguage{spanish}
\usepackage{url}
\bibliographystyle{babplain-lf}

\autor{Esteban de Jesús Rafael Camargo Rodríguez}
\usbid{11-10145}
\titulo{Diseño de un simulador instruccional para ingenieros y operadores de hornos de proceso}
\tutor{Antonio Cavero N.}
\usarcotutor
\cotutor{Euler Jimenez G.} 
\trabajo{Proyecto de grado}
\coord{Ingeniería Química}
\grado{Ingeniero Químico}
\fecha{Octubre~de~2022}

\autori{E. Camargo}
\agno{2022}
\fechadefensa{30~de~Octubre~de~2022}
\carrera{Ing. Química}

\programa{Nombre del Programa de Postgrado}
\juradouno{Alejandro Requena}
\juradodos{Sabrina D'Scipio \mbox{(Afiliación)}}
\juradotres{Claudio Olivares}

% Cambia comillas simple por comilla cerrada en ambiente verbatim 
\makeatletter
\let \@sverbatim \@verbatim
\def \@verbatim {\@sverbatim \verbatimplus}
{\catcode`'=13 \gdef \verbatimplus{\catcode`'=13 \chardef '=13 }} 
\makeatother

\begin{document}

\frontmatter
\maketitle
\chapter*{Dedicatoria}

\par Dedicado a la Universidad Simón Bolívar y a su comunidad.
\par ~
\par A mis padres.

\chapter*{Agradecimientos}

Gracias a mis tutores, Euler y Antonio, por su gu\'ia en este trabajo, respondiendo atentamente a todas mis preguntas, por su paciencia corrigiendo mis errores y por su constante motivación.\\

Gracias a mi madre, Marlene, por apoyarme incondicionalmente y siempre haber creído en mí.
\begin{resumen}
    \par El conocimiento al operar un equipo en cualquier planta de ingeniería siempre ha sido algo fundamental, pero hay ocasiones en que el conocimiento práctico se queda corto y no basta con saber manipularlo físicamente, si no también conocer los principios fundamentales detrás del funcionamiento del equipo. Los hornos de procesos en refinerías industriales son equipos que aportan mas de un 80\% de la energía del proceso, y por tanto fundamentales, además son complejos de automatizar con sistemas de control por lo que requieren de fuertes bases para su correcta manipulación, adicional a esto, la eficiencia en manipular este equipo repercute directamente en la eficiencia global del proceso, afectando la cantidad de residuos liberados al ambiente.
    \par Teniendo esto en cuenta se diseñó este proyecto para generar un simulador con interfaz web, capaz de transmitir de manera efectiva los principios y consecuencias del manejo del horno de procesos.
    \par El objetivo de la herramienta es encargarse de simular las posibles interacciones de un operador con el equipo, representando diferentes estados asociados con las variables manipulables, con su uso los operadores podrán tener una referencia de como cambiará el proceso dependiendo de las variables que decidan alterar, las cuales pueden ser el flujo del residuo de vacío (crudo), las temperaturas de entrada o salida de este crudo, la abertura o cierre de los ductos de alimentación de aire (representado en exceso de aire) o salida de gases de chimenea, humedad relativa del ambiente, factores de ensuciamiento, pérdidas de calor al ambiente y por último los componentes del combustible suministrado al horno.\\
     \vspace{10pt}\\
     Palabras claves: Horno de proceso, Fired heater, Combustión, Intercambiador, Simulador.
\end{resumen}
\tableofcontents
\listoffigures
\listoftables
\useacronyms
%\chapter*{Lista de símbolos}
\begin{tabular}{ll}

$AC_{masa}$ & Relación aire/combustible másica [-] \\
$AC_{molar}$& Relación aire/combustible molar [-] \\
$Afo$   & Área de superficie externa de aletas [m$^2$] \\
$Apo$   & Área de superficie expuesta de tubos lisos [m$^2$] \\
$A_0$   & Área de superficie externa total [m$^2$] \\
$At$    & Área externa del banco de tubos por zona del horno [m$^2$] \\
$Acp$   & Área de plano equivalente por zona del horno [m$^2$] \\
\\
$C_{1,3,5}$ & Coeficientes del factor de Colburn \\
$C_p$   & Calor específico [kJ/kg-K] \\
$C_{p_a}$   & Calor específico del aire [kJ/kg-K] \\
$C_{p_c}$   & Calor específico del combustible [kJ/kg-K] \\
$C_{p_f}$   & Calor específico del fluido [kJ/kg-K] \\
$C_{p_g}$   & Calor específico de los gases de combustión [kJ/kg-K] \\

$d_f$ & Diámetro externo de la aleta [m] \\
$d_o$ & Diámetro externo del tubo [m] \\

$E$   & Eficiencia de las aletas [-]\\
$F$   & Factor global de transferencia radiante [-]\\
$G$   & Velocidad másica basada en el área interna de tubos radiantes [kg/h-m$^2$]\\
$G_n$ & Velocidad másica basada en el área libre para el flujo de gas [kg/h-m$^2$]\\
\\
$h_{c}$ & Coeficiente de transferencia de calor convectivo [W/m$^2$-K]\\
$h_{e}$ & Coeficiente de transferencia de calor externo [W/m$^2$-K]\\
$h_{ee}$& Coeficiente de transferencia de calor externo efectivo [W/m$^2$-K]\\
$h_{i}$ & Coeficiente de transferencia de calor interno [W/m$^2$-K]\\
$h_{r}$ & Coeficiente de transferencia de calor radiante [W/m$^2$-K]\\
\\
$j$     & Factor de Colburn [-] \\
$k_f$   & Conductividad térmica del fluido [W/m-K] \\
$k_g$   & Conductividad térmica de los gases de combustión [W/m-K] \\
\\
$\dot m_a$  & Flujo másico de aire [kg/h] \\
$\dot m_c$  & Flujo másico de combustible [kg/h] \\
$\dot m_f$  & Flujo másico de residuo de vacío [kg/h] \\
$\dot m_g$  & Flujo másico de los gases de combustión [kg/h] \\
\end{tabular}

\begin{tabular}{ll}
\\
\\
$n_a$  & Número de moles del aire [mol] \\
$n_c$  & Número de moles del combustible [mol] \\
\\
$l_f$   & Altura de aleta [m] \\
$L_{tubo}$ & Largo efectivo del tubo [m] \\
$N_r$   & Numero de tubos por fila [-] \\
$N_{tubo}$ & Número de tubos por sección [-] \\
\\
$PL$    & \multirow{2}{26em}{Multiplicación de las presiones parciales de CO$_2$ y H$_2$O por MBL [atm-ft]} \\
\\
$PM_a$    & Peso molecular del aire [kg/kmol]\\
$PM_c$    & Peso molecular del combustible [kg/kmol]\\
\\
$P_l$   & Paso de tubo longitudinal [-]\\
$P_t$   & Paso de tubo transversal [-]\\
\\
$Pr$    & Número de Prandtl [-]\\
\\
$Q_{a}$    & Calor sensible del aire [MW] \\
$Q_{absorbido}$ & Calor absorbido por el fluido en el horno [MW] \\
$Q_{c}$    & Calor sensible del combustible [MW] \\
$Q_{conv}$ & Calor transferido por convección [MW] \\
$Q_{CONV}$ & Calor cedido al fluido en la sección convectiva [MW] \\
$Q_{ESC}$ & Calor cedido al fluido en la sección escudo [MW] \\
$Q_{f_e}$ & Calor del fluido entrando [MW] \\
$Q_{f_s}$ & Calor del fluido saliendo [MW] \\
$Q_{g}$   & Calor sensible de los gases de combustión [MW] \\
$Q_{perdidas}$  & Pérdidas de calor al ambiente [MW] \\
$Q_{rad}$ & Calor transferido por radiación [MW] \\
$Q_{RAD}$ & Calor cedido al fluido en la sección radiante [MW] \\
$Q_{radEsc}$& Calor de radiación que pasa a la sección escudo [MW] \\
$Q_{suministrado}$ & Calor suministrado por combustión [MW] \\
\\
$q_{rad}$ & \multirow{2}{26em}{Flujo de calor por unidad de área exterior del tubo en zona radiante [MW/m$^2$]} \\
\\
$Re$    & Número de Reynolds [-]\\
\\
$R_{fi}$  & Factor de ensuciamiento interno de los tubos [m$^2$-K/W] \\
$R_{fe}$  & Factor de ensuciamiento externo de los tubos [m$^2$-K/W] \\
\end{tabular}

\begin{tabular}{ll}
\\
\\
$R_{int}$  & Resistencia convectiva interna [m$^2$-K/W]\\
$R_{ext}$  & Resistencia convectiva externa [m$^2$-K/W] \\
$R_{tube}$ & Resistencia conductiva del tubo [m$^2$-K/W]\\
$\Sigma R$ & Suma de las resistencias de transferencia de calor [m$^2$-K/W]\\
\\
$s_f$      & Espaciado de aleta [1/m]\\
$S_{tubo}$ & Espaciado de tubos [m] \\
\\
$T_b$ & Temperatura de mezcla del fluido por zona [K] \\
$T_{fe}$& Temperatura del fluido entrando al horno [K] \\
$T_{fs}$& Temperatura del fluido saliendo del horno [K] \\
$T_{fer}$& Temperatura del fluido entrando a la zona radiante [K] \\
$T_{fee}$& Temperatura del fluido entrando a la zona escudo [K] \\
$T_{gc}$& Temperatura de los gases de combustión a la salida de la zona convectiva [K] \\
$T_{ge}$& Temperatura de los gases de combustión a la salida de la zona escudo [K] \\
$T_{gr}$& Temperatura de los gases de combustión a la salida de la zona radiante [K] \\
$T_g$   & Temperatura efectiva de llama, equivalente a $T_{gr}$ [K] \\
$T_{gb}$& Temperatura de mezcla del gas por zona [K] \\
$T_s$ & Temperatura promedio de aleta [K]\\
\\
$U_0$ & Coeficiente de transferencia de calor global [W/m$^2$-K] \\
\\
$\sigma$    & Constante de Stefan-Boltzmann [W/m$^2$-K$^4$] \\
$\rho$      & Densidad [kg/m$^3$] \\
$\alpha$    & Factor de efectividad del banco de tubos [-]\\
$\gamma_r$  & Factor de radiación externo [-] \\
$\mu_f$     & Viscosidad del fluido [cP] \\
$\mu_g$     & Viscosidad de los gases de combustión [cP] \\
\end{tabular}

\mainmatter
\chapter*{Introducción}

\par En refinerías de petróleo crudo, plantas químicas y petroquímicas, los hornos representan los equipos que suministran más del 80\% de la totalidad de la energía requerida para los procesos de separación y conversión química de los productos. Desde este punto de vista, la eficiencia energética de cada horno es una variable crítica a optimizar, con el fin de reducir el consumo de combustibles quemados, minimizar las emisiones de gases de combustión que, finalmente, se traducen en menores costos operativos [1].

\par Los \texttt{hornos de procesos} juegan un rol primordial en una refinería. Consumen altísimas cantidades de combustibles fósiles ({\tt miles de kilogramos por hora}) y emiten proporcionales cantidades de \ac{co2} a la atmósfera. Son equipos de alto costo que trabajan con llamas a muy altas temperaturas (\textgreater 2000 °C) y cuyas fallas suelen ser críticas para la economía de la refinería. Es importante destacar que el dióxido de carbono atmosférico promedio mundial en 2019 fue de 409,8 ppm, con un rango de incertidumbre de 0,1 ppm. Lo que indica que los niveles de dióxido de carbono resultaron los más altos de los últimos 800.000 años [2]. Teniendo como referencia las cifras del 2018 reportadas por la \ac{ONU}, 55,3 Gt\ac{co2}e (giga toneladas de \ac{co2} equivalente) [3], se puede apreciar el impacto significativo que esto genera sobre el medio ambiente. 

\par Lamentablemente, no es posible automatizar la totalidad de la operación de un horno de procesos [4] puesto que no son equipos completamente herméticos como un intercambiador de calor o una columna de destilación. Esta condición, es decir, estar "abiertos a la atmósfera", hace recaer sobre los operadores de sala de control y de campo (la interfase humano-horno), la responsabilidad última sobre el proceso de combustión y el aprovechamiento del combustible.  

\par Por experiencias prácticas, dentro de las refinerías se ha identificado que los operadores suelen manejar estos equipos con altos niveles de tiro y de exceso de aire, estas variables de operación están alejadas de las condiciones recomendadas al diseñar el horno, lo que conlleva mayores emisiones de CO2 al ambiente y además afecta la integridad mecánica del equipo. Los altos niveles de exceso de aire, provocados por altos niveles de tiro, traen consigo un mayor consumo del combustible para compensar el calor perdido al calentar el aire en exceso, lo que a su vez provoca alteraciones en las características de las llamas haciéndolas incidir directamente sobre los tubos de la sección radiante causando la coquización localizada de la carga dentro de los tubos. 

\par En virtud de la dependencia de los hornos de proceso de la intervención continua y acertada de sus operadores se hace perentorio fomentar su capacitación como una manera de incrementar el desempeño y la eficiencia de estos equipos. 

\par Por esta razón, se ha considerado necesario construir un simulador de hornos de procesos, con base en los estándares API 560 y API RP 535, para su utilización como herramienta práctica en la capacitación del personal operativo de las refinerías de petróleo. Esta herramienta permitiría, de una manera más didáctica, observar e interactuar con las variables fundamentales de control del horno, a saber, el tiro y el exceso de aire, para controlar los efectos de las operaciones alejadas de las condiciones de diseño.

\par Este proyecto utilizará ecuaciones de balance de masa y transferencia de calor para representar las condiciones de operación en un modelo gráfico capaz de responder a diferentes estímulos en las variables. 
\vfill
\par Esta es la versión \version~del documento de tesis para la \ac{USB}, creada y mantenida por:\\
\begin{tabular}{lc}
Br. Esteban Camargo & 11-10145@usb.ve 
\end{tabular}
\chapter{Antecedentes}

\par En este capítulo se desarrollan los puntos que respaldan este proyecto, como la justificación y objetivos. Adicionalmente se describen algunos proyectos que anteceden a la investigación y que se usaron como referencias.

\section{Justificación}

\par En virtud de que los hornos de proceso requieren de la intervención continua y acertada de sus operadores, se hace evidente la necesidad de fortalecer la capacitación de dichos operadores como una manera de incrementar la eficiencia en las condiciones de operación de estos equipos, lo que lograría disminuir la cantidad de emisiones que se arrojan al medio ambiente, disminuir el consumo de combustible, prolongar la vida útil del equipo y anticipar cambios en el proceso de una manera controlada.
\par Adicionalmente, para un estudiante o ingeniero que intente formarse en esta área los recursos para experimentar son limitados.
\par Por esta razón, se consideró necesario diseñar y desarrollar un simulador de hornos de proceso con base en los estándares API 560 \cite{bib:api560} y API RP 535 \cite{bib:api535}, para su utilización como herramienta práctica en la capacitación del personal operativo de las refinerías de petróleo, ya que las opciones de simuladores existentes no son abiertas al público y solo están enfocadas al diseño de hornos.
\par Esta herramienta debe permitir, de una manera didáctica, observar e interactuar con las variables fundamentales de control del horno, a saber: el exceso de aire en la combustión, la composición del combustible, la carga de flujo a manejar y las temperaturas de entrada y salida del fluido al horno; con el fin de simular y comparar los efectos de operar el horno alejado de las condiciones óptimas de diseño.

\section{Objetivos}

\subsection{Objetivo General}

\par Desarrollar un simulador web de hornos de proceso de tiro natural que pueda ser fácilmente utilizado para la enseñanza del manejo de estos equipos, al permitir visualizar los efectos de modificar los valores de las variables que gobiernan el proceso y su importancia en la eficiencia del equipo.

\subsection{Objetivos específicos}

\par Aplicar los diferentes conceptos asociados a la operación de hornos de proceso.

\par Aplicar las ecuaciones de balance de masa y energía, y las ecuaciones de transferencia de calor en la construcción del modelo matemático para el cálculo de las condiciones de operación de los hornos de proceso.

\par Diseñar una herramienta para la enseñanza de la operación de los hornos de proceso, usando un algoritmo basado en el modelo matemático anterior, con una interfaz de entrada y salida de datos amigable para los usuarios, con la que se simulen las modificaciones de las condiciones de proceso de un horno.

\par Desarrollar la herramienta mencionada e integrarla a una interfaz en línea, tomando ventaja del desarrollo de tecnologías web libres, que permita el acceso desde cualquier dispositivo con internet y aumente el alcance de dicha herramienta.

\section{Proyectos Previos}

\par Existen investigaciones enfocadas en la optimización de hornos de proceso particulares con características y condiciones definidas \cite{bib:leti}; también existen distintos simuladores enfocados en el diseño de hornos, pero no se ha encontrado un estudio con un alcance mayor, que intente analizar todo el comportamiento del horno en condiciones distintas a las de diseño, que también son observadas en la práctica y responden a un carácter académico.

\par También existen simuladores comerciales con enfoques orientados al diseño, pero ninguno con fines explícitamente educativos.

\par Lo innovador de este proyecto consiste en hacer como objetivo del simulador un aporte educacional y no una herramienta de diseño, con muchas aplicaciones que, además, pueden ampliarse si se continua desarrollando la interfaz con el algoritmo de base.
\chapter{Marco Teórico}

\par En este capitulo se trataran las bases teóricas necesarias para desarrollar los modelos matemáticos y realizar los cálculos que se usaran luego en la sección metodológica. 

Ecuaciones
Descripción de los fenómenos
Conceptos
Variables

\section{Secciones}

\par El horno puede ser dividido en secciones para simplificar los cálculos, las cuatro secciones b\'asicas son las siguientes:

\subsection{Combustión}
\subsection{Radiación}
\subsection{Escudo}
\subsection{Convección}
\chapter{Desarrollo del Modelo Matemático e Interfaz}
\par En este capítulo se describirá el algoritmo desarrollado para el cálculo de las variables que determinan el proceso de transferencia de calor en cada sección del horno, haciendo referencia a las ecuaciones y los métodos de aproximación antes descritos. También se describirá la interfaz gráfica diseñada para interactuar con el simulador y consistente en pantallas de introducción de datos y pantallas para mostrar los resultados del simulador.

\section{Algoritmo}
\par El diagrama mostrado en la Figura \ref{fig:diagrama-algo}, representa la secuencia del algoritmo. Consta de un ciclo externo que incluye una sección de combustión, sección de radiación, sección de escudo y sección de convección. Su descripción será desarrollada en las subsecciones consiguientes.
\begin{figure}[hbt]
\begin{center}
\includegraphics[scale=0.45]{images/diagrama-algo}
\caption[Diagrama de algoritmo]{Diagrama descriptivo del algoritmo desarrollo para el simulador.}
\label{fig:diagrama-algo}
\end{center}
\end{figure}

\par Datos iniciales que recibe el simulador (no todas editables en la interfaz):
\begin{itemize}
    \item Propiedades del fluido, flujo volumétrico, gravedad especifica, calor especifico, conductividad térmica, viscosidad y temperatura, para la entrada y salida.
    \item Composición y temperatura del combustible.
    \item Composición, humedad relativa y temperatura del aire atmosférico.
    \item Configuración y dimensiones del horno.
    \item Propiedades y dimensiones de los tubos y aletas.
\end{itemize}

\subsection{Sección de combustión}
\par En esta sección se ejecutan los cálculos de toda la energía que entra al horno a través de un balance de masa y energía sobre el combustible.
\par Las propiedades del combustible y del gas de combustión resultante, como capacidad calorífica especifica, peso molecular, viscosidad y conductividad térmica, se calculan a partir de las temperaturas, de entrada o por zona, y las fracciones molares de sus componentes individuales.
\par En el simulador se programaron funciones cuadráticas para obtener la viscosidad y conductividad de los componentes del gas de combustión, con dependencia en la temperatura, en caso de la capacidad calorífica se usa la ecuación cúbica mostrada en el libro de Borgnakke y Sonntag\cite{bib:vanwylen}, los coeficientes para cada una de las propiedades se calcularon con los datos del Instituto Nacional de Estándares y Tecnología (NIST)\cite{nist}, el código desarrollado para las ecuaciones se encuentra en los apéndices. 
\par Finalmente se calcula la relación aire/combustible con la ecuación \ref{eq:ac}, la temperatura adiabática de llama y el poder calorífico neto del combustible (NCV) de la ecuación \ref{eq:pc}.


\subsection{Cálculo térmico del horno}
\par Aquí se utiliza un ciclo encargado de recibir todos los parámetros de entrada y correr cada una de las subsecciones como funciones, es donde se define la distribución de calor entre las secciones del horno y varia este valor para alcanzar la tolerancia deseada con un método iterativo aplicado a la siguiente ecuación.
\begin{equation}
\label{eq:ciclo_externo}
\frac{(Q_{residuo} - Q_{calc})}{Q_{calc}} \approx 0
\end{equation}
\par Donde:\\
$Q_{residuo} = m_{residuo} * C_{p_residuo} * {\Delta}T_{residuo}$ \\
$Q_{calc} = Q_{RAD} + Q_{ESC} + Q_{CONV}$ \\
$Q$ = Calor transferido. \\

\par Lo que se traduce en que el calor absorbido por el residuo debe ser igual al calor suministrado en cada una de las secciones del horno.
\par El valor inicial de distribución radiante esta definido en 64\% y esta converge para el rango de temperaturas permitido en la interfaz del simulador. En las siguientes subsecciones se detallan las ecuaciones para cada etapa del horno.

\subsection{Sección de radiación}
\par Tres variables quedan por determinar en esta sección $Q_{RAD}$, la temperatura de entrada del fluido a la zona $T_{fer}$ radiante y la temperatura efectiva del gas $T_g$.

\par \textbf{Suposición A}: Para obtener un primer valor de la temperatura de entrada del residuo de vacío a esta sección se supone una absorción del calor del horno ($Q_{absorbido}$) como 70\% en la zona radiante y el 30\% entre la zona escudo y zona convectiva, en consecuencia (ec. \ref{eq:sup-a}):
\begin{equation}\label{eq:sup-a} Q_{RAD} = 0.7 * Q_{absorbido} \end{equation}
\par El $Q_{RAD}$ supuesto corresponde al valor inicial de un procedimiento de ensayo y error que cierra con al obtener un $Q_{absorbido}$ calculado ($Q_{RAD} + Q_{ESC} + Q_{CONV}$) igual al $Q_{absorbido}$ especificado. A partir de $Q_{RAD}$ se pueden estimar las siguientes variables:
\begin{itemize}
\item De la ecuación (\ref{eq:rad-fluid}) se obtiene la temperatura de entrada del fluido a la zona radiante $T_{fer}$ y la temperatura de mezcla $T_b$.
\item Conocida el área de los tubos se obtiene $Flux_R = Q_{RAD} /At$.
\item De la ecuación (\ref{eq:hi}) se obtiene el coeficiente de película $h_i$. Los números de Re y Pr se calculan a $T_b$.
\item De la ecuación (\ref{eq:tw}) se obtiene la temperatura de pared del tubo $T_w$.
\item Con $T_w$ calculada se recalcula $h_i$, incluyendo la corrección por viscosidad y $T_w$.
\end{itemize}
\par Todos los valores calculados dependen del supuesto A, el  posterior ajuste de QR modifica estos valores.
\par Finalmente mediante la ecuación (\ref{eq:rad-tgr}) y usando el Método de Newton Raphson se calcula la Temperatura Efectiva del gas, $T_g$ y mediante la ecuación (\ref{eq:rad-comp}) el consumo  de combustible correspondiente a la suposición A.

\subsection{Sección de escudo}
\par De la zona escudo se conocen tres variables, la temperatura de salida del fluido $T_{fer}$, la temperatura de entrada del gas de combustión $T_{gr}$ = $T_g$ y la cantidad de calor por radiación que escapa hacia la zona escudo $Q_{radEsc}$. Son incógnitas la temperatura de salida del gas $T_{ge}$, la temperatura de entrada del fluido $T_{fee}$ y el calor transferido $Q_{ESC}$.
\par \textbf{Suposición B}: Suponer la temperatura de entrada del fluido de la zona escudo $T_{fee}$. Recurriendo nuevamente al calor absorbido especificado, el 30\% se transfiere en la zona escudo y la zona convectiva por la suposición A, si consideramos que en el escudo se absorbe el calor por radiación que escapa de la zona radiante lo que en cierta forma compensa la menor área de transferencia en comparación con la zona convectiva, como una razonable aproximación (ec. \ref{eq:sup-b}) se supone que:
\begin{equation} \label{eq:sup-b} \begin{gathered}
    Q_{ESC} \approx Q_{CONV} \approx 0.15 * Q_{absorbido}\\
    Q_{ESC_{sup}} = 0.15 * Q_{absorbido} 
\end{gathered} \end{equation}
\par Esto conlleva a un valor inicial para el ensayo-error de $T_{fee}$ = ($T_{fer}$ +$T_{fe}$)/2. 

\begin{itemize}
\item A partir de la ecuación (\ref{eq:esc}) calcular la temperatura de salida del gas $T_{ge}$.
\item Calcular la temperatura de mezcla del gas $T_{gb}$ y del fluido $T_{fb}$.
\item Calcular LMTD mediante la ecuación (\ref{eq:qesc-lmtd}).
\item Calcular ho, hr, hi mediante las ecuaciones (\ref{eq:hi}), (\ref{eq:ho}) y (\ref{eq:hr}).
\item Calcular Uo.
\item Calcular $Q_{ESC}$ a partir de la ecuación (\ref{eq:tesc}).
\item Calcular $T_{fee}$ a partir de la ecuación (\ref{eq:lmtd}).
\item Comparar $T_{fee}$ calculada con $T_{fee}$ supuesta.
\end{itemize}
\par Si $|\frac{T_{fee_c} - T_{fee_s}}{T_{fee_s}}| > error$, se hace $T_{fee_s} = T_{fee_c}$ y $Q_{ESC_{sup}} = Q_{ESC_{calc}}$ y se regresa al primer punto.
\par Si $|\frac{T_{fee_c} - T_{fee_s}}{T_{fee_s}} | < error$, se da por cerrado el cálculo de la zona de escudo. Al final del proceso $\frac{Q_{ESCcalc} - Q_{ESCsup}}{ Q_{ESCsup}}$ debe estar en un valor cercano a cero.

\subsection{Sección de convección}
\par En la zona conectiva las variables conocidas provenientes del cálculo de la zona Radiante y Escudo son la temperatura del gas a la entrada $T_{ge}$ y la temperatura de salida del fluido $T_{fee}$.
\par \textbf{Suposición C}: Para tener solo una temperatura como incógnita en el cálculo se supone la temperatura de entrada del residuo de vacío al horno, $T_{fe}$, igual al valor especificado en el ingreso de datos y se procede a:
\begin{itemize}
\item Calcular la temperatura de salida del gas $T_{gc}$ y el calor transferido $Q_{CONV}$ correspondiente mediante la ecuación (\ref{eq:conv}).
\item Calcular temperatura de mezcla del gas y del fluido.
\item Calcular LMTD.
\item Calcular Uo, según las ecuaciones descritas para tubos con aleta.
\item Calcular $Q_{CONV_{calc}}$ a partir de la ecuación (\ref{eq:qconv}).
\end{itemize}
\par Si $|\frac{Q_{CONVcalc} - Q_{CONVsup}}{Q_{CONVsup}} | > error$ se recalcular la temperatura de entrada del fluido $T_{fe}$ usando el método de bisección y se vuelve al punto inicial de la sección.
\par Si $|\frac{Q_{CONVcalc} - Q_{CONVsup}}{Q_{CONVsup}} | < error$, se cierra el cálculo en la zona convectiva.
\par Las variables calculadas al finalizar el procedimiento son  $Q_{CONV}$, $T_{fe}$ y $T_{ge}$.

\subsection{Cierre del algoritmo de cálculo del horno}
\par A este nivel de avance se han calculado $Q_{RAD}, Q_{ESC}, Q_{CONV}$ y todas las temperaturas excepto la temperatura del fluido a la salida del horno $T_{fs}$, la cual se ha mantenido igual a la especificada. El control del proceso de simulación vuelve al ciclo externo y dos comprobaciones con los valores calculados son posibles:\\
\par ¿Es la temperatura de entrada $T_{fe}$ calculada $>, <$ ó = a la $T_{fe}$ especificada?
\par ¿Es $Q_{RAD} + Q_{ESC} + Q_{CONV} >, <$ ó = al $Q_{absorbido}$?\\
\par Ambas están relacionadas y se abre el siguiente flujo condicional para manejar los tres escenarios:
\begin{itemize}
    \item Si $|\frac{T_{fe_{calc}} - T_{fe_{sup}}}{T_{fe_{sup}}} | > error$ y si $T_{fe_{calc}} > T_{fe_{sup}}$ entonces $Q_{RAD} + Q_{ESC} + Q_{CONV} < Q_{absorbido}$. Aumentar el calor transferido en la zona radiante y repetir todo el proceso de cálculo desde la sección de radiación. El resultado será el incremento de la temperatura efectiva $T_{g}$ y su efecto aguas abajo en la zona escudo y convectiva.
    \item Si $|\frac{T_{fe_{calc}} - T_{fe_{sup}})}{T_{fe_{sup}}} | > error$ y si $T_{fe_{calc}} < T_{fe_{sup}}$ entonces $Q_{RAD} + Q_{ESC} + Q_{CONV} > Q_{absorbido}$. Disminuir la distribución del calor transferido a la zona radiante y volver a la sección radiante para repetir los cálculos con esta nueva suposición. El resultado será que decrece la temperatura efectiva $T_{g}$ y su efecto aguas abajo en la zona escudo y convectiva.
    \item Si $|\frac{T_{fe_{calc}} - T_{fe_{sup}})}{T_{fe_{sup}}} | < error$ entonces también $|Q_{RAD} + Q_{ESC} + Q_{CONV} - Q_{absorbido}|$ es menor al error establecido, se considera correcta la suposición A.
\end{itemize}
\par Se continua a reportar los resultados y finaliza el algoritmo.

\section{Interfaz de usuario}
\par Una interfaz de usuario (UI de su término comúnmente usado inglés) es el medio por el cual una persona controla una aplicación de software o dispositivo de hardware. Esto significa que el programa incluye controles gráficos que optimizan la experiencia de usuario usando un ratón, teclado o pantalla táctil.
\par Los elementos más comunes de una interfaz gráfica de usuario, de acuerdo con Career Foundry\cite{ui}, son:
\begin{itemize}
\item Controles de entrada: permiten introducir información en el sistema por parte de los usuarios.
\item Componentes de navegación: ayudan a los usuarios a moverse.
\item Componentes informativos: brindan información a los usuarios.
\item Contenedores: mantienen el contenido organizado, como paneles, ventanas, marcos, etc.
\end{itemize}

\par La interfaz de usuario implementada para el algoritmo se desarrolló usando los lenguajes básicos de desarrollo web, Lenguaje de Marcado de Hipertexto (HTML), Hojas de Estilo en Cascada (CSS) y el lenguaje de programación interpretado JavaScript (JS), lo que hace que sea totalmente editable y pueda correr en cualquier ordenador o dispositivo móvil con navegador web sin instalar otro programa.

\subsection{Descripción de uso y alcances}
\par Esta vista otorga instrucciones y limitaciones de uso del simulador, al ser la vista principal el usuario podrá conocer todas las capacidades y aplicaciones disponibles en la versión actual publicada (ver la Figura \ref{fig:alcance}).
\par Aquí se describen las condiciones de operación simulado para dar una referencia al usuario de las cargas que puede manejar el simulador dentro de los rangos reales de uso.

\subsubsection{Condiciones de Operación}
\begin{itemize}
\item \textbf{Diseño}: 23.0 MW (78.79 MMBtu/h) - Máxima capacidad de procesamiento.
\item \textbf{Normal}: 20.9 MW (71.5 MMBtu/h) [90,75\% del Diseño] - Operación habitual del horno.
\item \textbf{Turndown}*: 10.45 MW (35.77 MBtu/h) [45,4\% del Diseño] - Condición de mínima capacidad.
\end{itemize}

\par *No equivale a una condición de "turndown" de los quemadores puesto que podría lograrse con cierto número de quemadores operando normalmente y otros quemadores simplemente apagados.

\subsection{Introducción de datos}

\par En esta sección se desarrollan dos vistas, (Fig. \ref{fig:datos} y Fig. \ref{fig:fulldatos}) una simplificada para introducir solo los datos más relevantes del proceso y que dirige a una pantalla de resultados donde se pueden comparar dos estados; otra donde se observan todas las variables que permite modificar el proceso y esta a su vez tiene dos opciones de resultados, una pantalla de valores numéricos detallados y una segunda pantalla que muestra una variación gráfica en el rango de cuatro variables a escoger.  

\subsection{Resultados ampliados}

\par En esta vista se detallan todas las variables resultantes por cada sección del horno (Fig. \ref{fig:fullresultados}), se puede escoger el sistema de unidades a expresarse en la pantalla anterior.
\par El objetivo de esta vista es tener una visión del comportamiento de todas las variables en el proceso, de aquí se pueden encontrar explicaciones no tan obvias del comportamiento de las variables finales mostradas en la vista comparativa.

\subsection{Resultados comparativos}

\par Esta es la vista que aspira ser más educativa al resumir todas las variables resultantes y hacer una comparación entre dos condiciones de operación (Fig. \ref{fig:resultados}).
\par Aquí se puede observar rápidamente los cambios en variables particulares y de interés, como los datos de ingreso, las emisiones de \ac{co2}, el consumo de combustible, la distribución de calor dentro del horno, la temperatura de salida de los gases de combustión y la eficiencia térmica.

\subsection{Gráficas de tendencias}

\par El propósito inicial de esta sección fue generar una visual del comportamiento del horno simulado a lo largo de un rango, se uso para conseguir zonas donde los métodos aproximados del simulador no convergían y permitió sintonizar el paso, numero de iteraciones, valor inicial y tolerancia de dichos métodos.
\par Luego de ser optimizada ahora permite observar la tendencia de las variables de salida seleccionadas con precisión, facilitando la comprensión de los resultados de los fenómenos dentro del horno (Fig. \ref{fig:graficas}).
\chapter{Análisis de Resultados}

\par En este capitulo se describirá el simulador desarrollado con sus condiciones de diseño, limitaciones, suposiciones, validaciones, funcionalidades y sus posibles usos y aplicaciones.

\section{Horno simulado}

\par El horno escogido es de tipo cabina de simple fuego, con una capacidad de carga de 23 megavatios, o el equivalente aproximado de un cambio de 50 ºC para un flujo de 16 mil metros cúbicos por día de residuo de petroleo con gravedad especifica de 0.84.

\par Estas variables de diseño pueden ser modificados en un futuro pero se tomaron por los datos disponible para comparar contra la simulación.

\subsection{Diseño mecánico}

\par Se puede observar a detalle cada una de las dimensiones en la Figura \ref{apx:img} ampliada en los anexos, estas dimensiones responden a la distribución interna de los tubos para el fluido y sus configuraciones por sección.

\begin{figure}[hbt]
\begin{center}
\includegraphics[scale=0.45]{images/firebox}
\caption[Diagrama de la cámara de combustión algoritmo]{Diagrama de la cámara de combustión del horno simulado.}
\label{fig:firebox}
\end{center}
\end{figure}

\par En la tabla \ref{tbl:firebox} se muestran las dimensiones de la cámara de combustión, mostradas en pies para su correspondiente uso en el cálculo de la longitud promedio de láser, y se pueden denotar los detalles del techo de la cámara en la Figura \ref{fig:firebox}.

\begin{table}
\begin{center}
\caption[Dimensiones de la cámara de combustión]{Dimensiones de la cámara de combustión, mostrada en pies para su uso en la ecuación \ref{eq:pl} y la tabla \ref{tbl:mbl}}
\label{tbl:firebox}
\begin{tabular}{c|c}
Altura interna, ft	& 27.407 \\
Largo interno, ft 	& 64.552 \\
Ancho interno, ft 	& 17.500 \\
\end{tabular}
\end{center}
\end{table}

\par La descripción detallada de las característica de los tubos y aletas usadas se puede encontrar en la tabla \ref{tbl:tubes} y una vista en perspectiva de la distribución de los tubos se puede apreciar en la Figura \ref{fig:diagrama-meca}.

\begin{table}
\begin{center}
\caption[Distribución de tubos en el horno]{Distribución de tubos en el horno}
\label{tbl:tubes}
\begin{tabular}{l|c|c|c}
Sección 					& Radiante			& Escudo			& Convectiva \\
\hline
Material de tubos			& \multicolumn{3}{c}{-- A-312 TP321 --} \\
Soporte de tubos    		& Interno			& Externo			& Externo		\\
Arreglo			    		& Paralelo			& Escalonado		& Escalonado	\\
No. de tubos total    		& 42				& 16				& 40			\\
No. de tubos por fila		& 2					& 8					& 8				\\
Grosor pared de tubos, cm	& 0.818				& 0.711				& 0.711			\\
Di. interno de tubos, cm	& 7.981				& 16.827			& 16.827		\\
Espaciado de tubos, cm  	& 40.640			& 					& 				\\
Espaciado transversal, cm  	&					& 30.480			& 30.480		\\
Espaciado longitudinal, cm 	&					& 30.480			& 30.480		\\
Largo de tubos efectivo, m	& 18.926			& 18.288			& 18.288		\\
\hline
Material de aletas			&					& 					& 11.5-13.5Cr	\\
Tipo de aletas				&					& 					& Solidas		\\	
Altura de aletas, cm		&					& 					& 2.438			\\
Grosor de aletas, cm		&					& 					& 0.152			\\
Densidad de aletas, aleta/m	& 					& 					& 196.850		\\
\end{tabular}
\end{center}
\end{table}

\subsection{Combustible seleccionado y aire simulado}

\par Como combustible se utilizó un gas de refinería por ser lo más común en la industria petrolera, la composición molar establecida para hacer las pruebas del algoritmo se describe en la tabla \ref{tbl:combustible}. Sin embargo, la posibilidad de cambiar la composición esta abierta en la interfaz del simulador.

\par El aire ambiental seco establecido para simplificar los cálculos solo contiene \ac{n2} y \ac{o2}, otros compuestos como el \ac{argon} son despreciados por sus aporte del 1\% o menos a las ecuaciones de combustión. Como se describió en la sección de combustión del capitulo 2, la composición tomada para el aire seco es 20.95\% de \ac{o2} y 79.05\% de \ac{n2}.

\par La humedad relativa del aire establecida en las pruebas fue del 50\% con una temperatura ambiente de 26.667 ºC.

\begin{table}
\begin{center}
\caption[Composición del combustible base]{Composición molar porcentual del combustible base.}
\label{tbl:combustible}
\begin{tabular}{l|r}
	Gas combustible					& (Moles \%) \\
	\hline
	Metano ($CH_4$)					& 56.470 \\
	Etano ($C_2H_6$)				& 15.150 \\
	Propano ($C_3H_8$)				& 6.220 \\
	n-Butano ($C_4H_{10}$)			& 1.760 \\
	i-Butano ($C_4H_{10}$)			& 0.750 \\
	Etileno ($C_2H_4$)				& 1.580 \\
	Propileno ($C_3H_6$)			& 2.770 \\
	Monóxido de carbono ($CO$)		& 0.660 \\
	Dióxido de carbono ($CO_2$)		& 2.540 \\
	Hidrógeno ($H_2$)				& 11.420 \\
	Nitrógeno ($N_2$)				& 0.680 \\
	Agua ($H_2O$)					& 0.000 \\
	Oxigeno ($O_2$)					& 0.000 \\
	Sulfuro de hidrógeno ($H_2S$)	& 0.000 \\
\end{tabular}
\end{center}
\end{table}

\subsection{Datos y resultados de la combustión simulada}
\par Luego de correr la simulación desarrollada, en la sección de combustión, se obtuvieron los datos de la tabla \ref{tbl:combustion-data}, adicionalmente, la composición del gas de combustión resultante se describe en la tabla \ref{tbl:combustion-gas}.

\begin{table}
\begin{center}
\caption[Datos de la combustión]{Datos de la combustión simulada.}
\label{tbl:combustion-data}
\begin{tabular}{l|c}
	Exceso de aire, \%								& 20.000 \\
	Humedad relativa, \%							& 50.000 \\
	Temperatura del aire, °C						& 26.667 \\
	A/C másica teórica requerido					& 15.574 \\
	A/C másica húmeda con exceso de aire			& 18.689 \\
	Unidad de gas de combustión por unidad de aire  & 19.689 \\
	Poder Calorífico Neto  (NCV), kJ/kg				& 45,718.6 \\
	Poder Calorífico Bruto (GCV), kJ/kg				& 50,268.0 \\
	Peso Molecular (PM) del combustible, kg/kmol	& 21.149 \\
	PM del gas de combustión, kg/kmol				& 27.911 \\
\end{tabular}
\end{center}
\end{table}

\par Para validar la confiabilidad de estos resultados se compararon con los obtenidos en el software privado de diseño de hornos, WinHeat\copyright, con más de 25 años de trayectoria en el mercado, resultados probados en campo y con una subscripción anual valorada en \$4.000.

\par La conclusión de esta comparación arrojó que la diferencia entre los resultados de ambos simuladores es menor al 0.1\% en esta sección.

\begin{table}
\begin{center}
\caption[Composición del gas de combustión]{Composición molar y másica porcentual del gas de combustión}
\label{tbl:combustion-gas}
\begin{tabular}{l|c|c|c}
		& \% Peso & \% Moles \\
	\hline
	Nitrógeno ($N_2$)			& 72.121	& 71.860 \\
	Oxígeno ($O_2$)				& 3.636 	& 3.172	 \\
	Dióxido de carbono ($CO_2$)	& 13.753	& 8.723	 \\
	Vapor de agua ($H_2O$)		& 10.490	& 16.245 \\
	Dióxido de azufre ($SO_2$)	& 0.000 	& 0.000	 \\
	Monóxido de carbono ($CO$)	& 0.000 	& 0.000	 \\
\end{tabular}
\end{center}
\end{table}

\subsection{Datos de la secciones de transferencia de calor}

\par Datos de comparación se muestran en las tablas \ref{tbl:compara-zr}, \ref{tbl:compara-ze} y \ref{tbl:compara-zc}.

\par En estas tablas se pueden apreciar las diferencias dentro de cada sección del simulador contra los resultados del simulador privado de referencia WinHeat\copyright, en las distribuciones de calor la diferencia máxima es de 2.5\%, esto desencadena diferencias en las temperaturas de entrada y salida internas del gas de combustión y del fluido de proceso, naturalmente al cambiar la distribución del calor por sección esta se refleja en los valores intermedios.

\begin{table}
\begin{center}
\caption[Resultados en zona radiante de intercambio de calor]{Comparación de resultados en zona radiante de intercambio de calor.}
\label{tbl:compara-zr}
\begin{tabular}{l|c|c}
	& Sim EC 2022 & WinHeat\copyright \\
Temperatura entrada fluido, °C	& 379 & 378	\\
Temperatura salida fluido, °C	& 411 & 411	\\
Temperatura pared de tubo, °C	& 432 & 459	\\
Temperatura salida gas, °C	    & 798 & 813	\\

Masa de combustible, kg/s		& 0.574 & 0.564	\\
Calor combustión, MW			& 26.233 & 25.790	\\
Calor aire, MW					& 0.123 & 0.119	\\
Calor combustible, MW			& 0.022 & 0.000	\\

Calor gas de combustión, MW		& 11.083 & 10.615	\\
Calor de perdidas, MW			& 0.393 & 0.226	\\
Calor radiante a escudo, MW		& 1.747 & 1.665	\\
Calor de convección, MW			& 1.705 & 1.650	\\
Calor de radiación, MW			& 11.443 & 11.753	\\
Calor del fluido, MW			& 13.154 & 13.417	\\

Distribución de calor radiante, \%	& 62.78 & 64.01 \\

Área total (At), m$^2$				& 547.09 & 547.01 \\
Área refractaria (Ar), m$^2$		& 566.42 & 541.70 \\
Área de plano frío (Acp), m$^2$		& 323.05 & 369.73 \\
Factor de eficiencia alfa			& 0.904 & 0.904 \\

Coef. de conv. int., kJ/h-m$^2$-C	& 2,770.4 & 2,992.7 \\
Coef. de conv. ext., kJ/h-m$^2$-C	& 30.663 & no reporta \\

Longitud media de láser (MBL), ft	& 20.829 & 20.45 \\
P$_{CO_2}$+P$_{H_2O}$, atm 		    & 0.249 & 0.250 \\
\end{tabular}
\end{center}
\end{table}

\par La temperatura de chimenea, a la cual salen los gases de combustión del horno, es 9 ºC mayor en el simulador desarrollado (2.3\%), lo que repercute en los valores de eficiencia obtenidos.

\par El flujo másico de combustible obtenido es ligeramente mayor (1.77\%) al reporta por el simulador de referencia y la máxima diferencia en las temperaturas internas corresponde a la entrada del gas en la zona convectiva (2.97\%).

\begin{table}
\begin{center}
\caption[Resultados en zona escudo de intercambio de calor]{Comparación de resultados en zona escudo de intercambio de calor.}
\label{tbl:compara-ze}
\begin{tabular}{l|c|c}
	& Sim EC 2022 & WinHeat\copyright \\
Temperatura entrada fluido, °C		& 370 & 369	\\
Temperatura salida fluido, °C		& 379 & 378	\\
Temperatura pared de tubo, °C		& 401 & 407	\\
Temperatura entrada gas, °C			& 798 & 813	\\
Temperatura salida gas, °C			& 694 & 674	\\
Temperatura logarítmica media, K	& 370 & 660 \\
Masa gas de combustión, kg/s	& 11.298 & 11.099	\\

Calor gas de combustión, MW		& 1.593 & no reporta \\
Calor de convección, MW			& 1.592 & 1.957	\\
Calor de radiación, MW			& 1.747 & 1.665	\\
Calor del fluido, MW			& 3.340 & 3.621	\\
Distribución de calor escudo, \%	& 15.89\% &  17.3\% \\

Área total (At), m$^2$				& 154.688 & 158.678 \\
Área de plano frío (Acp), m$^2$		& 44.593 & no reporta \\
Factor de eficiencia alfa			& 1.00 & 1.00 \\

Coef. tran. global (Uo), kJ/h-m$^2$-C	& 100.253  & 121.016 \\
Coef. de conv. int., kJ/h-m$^2$-C	& 4,426.5 & 4,411.3 \\
Coef. de conv. ext., kJ/h-m$^2$-C	& 30.663 & no reporta \\
\end{tabular}
\end{center}
\end{table}

\par Las áreas de transferencia totales de los tubos del horno son iguales en la sección de radiación, 2.5\% menores en la sección de escudo y 4.1\% menores en la sección de convección, donde se incluye en área de las aletas.
\par Por último, con más diferencias observadas, se encuentran los coeficientes de transferencia de calor, en la zona radiante el único reportado por WinHeat\copyright es el coeficiente de convección interno que difiere en 7.1\%, en las otras zonas esta diferencia no supera el 0.5\%; mientras que el coeficiente de transferencia global de la zona de escudo y convección difieren en 17.3\% y 4.2\% respectivamente.
\par Los resultados fueron revisados uno a uno, se tomaron los que mostraron diferencias significativas y fueron corroborados a mano con las ecuaciones disponibles y al comprobar su tendencia en los rangos de diseño del horno, cerrando la aproximación del calor absorbido por el fluido al 0.5\%, se culminó el desarrollo del algoritmo de cálculo con éxito.

\subsubsection{Resultados adicionales}
\par Como resultados adicionales el simulador muestra los valores de emisión de \ac{co2} en toneladas/año, la eficiencia térmica del calor neto (NHV) y calor bruto (GHV)\cite{bib:api560}, con sus respectivos poderes caloríficos asociados.

\begin{table}
\begin{center}
\caption[Resultados en zona convectiva de intercambio de calor]{Comparación de resultados en zona convectiva de intercambio de calor.}
\label{tbl:compara-zc}
\begin{tabular}{l|c|c}
	& Sim EC 2022 & WinHeat\copyright \\
Temperatura entrada fluido, °C		& 359 & 369	\\
Temperatura salida fluido, °C		& 370 & 370	\\
Temperatura pared de tubo, °C		& 366 & 361	\\
Temperatura entrada gas, °C			& 694 & 674	\\
Temperatura salida gas, °C			& 391 & 382	\\
Temperatura de aletas media, °C     & 366 & 364\\
Temperatura logarítmica media, K	& 126 & 92 \\

Calor gas de combustión, MW		& 4.452 & no reporta \\
Calor de convección, MW			& 4.459 & 3.939	\\
Calor del fluido, MW			& 3.340 & 3.621	\\
Calor liberado en chimenea, MW	& 4.910 & no reporta \\

Distribución de calor convectiva, \%	& 21.28\% &  18.79\% \\

Área total (At), m$^2$			& 4,670.8 & 4,872.6 \\
Eficiencia de aletas			& 98.77\% & no reporta \\

Coef. tran. global (Uo), kJ/h-m$^2$-C	& 27.295 & 26.493 \\
Coef. de conv. int., kJ/h-m$^2$-C	& 4,300.2 & 4,331.4 \\
Coef. de conv. ext., kJ/h-m$^2$-C	& 26.261 & no reporta \\
Coef. ext. promedio, kJ/h-m$^2$-C	& 27.887 & 35.009 \\
Coef. ext. efectivo, kJ/h-m$^2$-C	& 27.563 & 30.689 \\
Coef. de radiación, kJ/h-m$^2$-C	& 1.626  & no reporta \\
\end{tabular}
\end{center}
\end{table}

\subsection{Curvas de tendencia}

\par Estas curvas definen el rango de uso y la confiabilidad (tendencias continuas y sin inestabilidades) del algoritmo. 
\par Son el resultado de correr la simulación variando cuatro variables de entrada, una a su vez, las variables escogidas fueron:

\begin{enumerate}
\item El flujo de residuo.
\item La temperatura de salida del residuo.
\item El exceso de aire usado en la combustión.
\item La humedad relativa en el aire de combustión.
\end{enumerate}

\par Definido un rango, este se divide entre un número de puntos establecido y se corre la simulación moviendo la variable escogida por cada punto. Dependiendo del número de puntos escogidos el simulador podrá requerir más tiempo para finalizar los cálculos, en los rangos permitidos de la interfaz el tiempo máximo de espera es de cinco segundos.
\par Para visualizar los cambios generados, se escogieron seis variables resultantes, las cuales son:

\begin{enumerate}
\item El flujo de combustible.
\item La temperatura de arco radiante.
\item La temperatura de chimenea.
\item La relación de absorción de calor entre las zona radiante y la convectiva.
\item La eficiencia térmica (@ Valor calorífico neto).
\item Las emisiones de CO$_2$.
\end{enumerate}

\par En la secuencia de Figuras \ref{fig:graph-t_out-fuel}, \ref{fig:graph-t_out-arc}, \ref{fig:graph-t_out-chim}, \ref{fig:graph-t_out-dist}, \ref{fig:graph-t_out-efic} y \ref{fig:graph-t_out-emi} se puede apreciar el cambio de las seis variables resultantes contra la variación de la temperatura de salida del residuo.

\begin{figure}[hbt]
\begin{center}
\includegraphics[scale=0.38]{images/graph-t_out-fuel}
\caption[Flujo de combustible vs Temperatura de salida de residuo]{Flujo de combustible en función de la temperatura de salida de residuo.}
\label{fig:graph-t_out-fuel}
\end{center}
\end{figure}

\par En la Figura \ref{fig:graph-t_out-fuel} se observa la necesidad de incrementar el flujo de combustible para mantener las condiciones de operación si se aumenta la temperatura a la que se desea obtener el residuo al salir del horno, este comportamiento también se observa al aumentar las otras tres posibles variables de entrada de forma independiente. Esto a consecuencia de que el combustible es la fuente de calor del horno y cualquier necesidad de energía adicional en el proceso requiere mayor flujo de combustible.

\par Al aumentar el flujo de combustible por el requerimiento de de una temperatura de salida del fluido mayor, y dejando el resto de las variables de entrada, como exceso de aire y humedad relativa, se nota un aumento en la temperatura de arco radiante (Fig. \ref{fig:graph-t_out-arc}) y por transitividad en la temperatura de salida de los gases de chimenea (Fig. \ref{fig:graph-t_out-chim}).

\begin{figure}[hbt]
\begin{center}
\includegraphics[scale=0.38]{images/graph-t_out-arc}
\caption[Temperatura de arco radiante vs Temperatura de salida de residuo]{Temperatura de arco radiante en función de la temperatura de salida de residuo.}
\label{fig:graph-t_out-arc}
\end{center}
\end{figure}

\begin{figure}[hbt]
\begin{center}
\includegraphics[scale=0.38]{images/graph-t_out-chim}
\caption[Temperatura de chimenea vs Temperatura de salida de residuo]{Temperatura de chimenea en función de la temperatura de salida de residuo.}
\label{fig:graph-t_out-chim}
\end{center}
\end{figure}

\par Para la variable de flujo de residuo también se observa la misma tendencia, siendo solo diferente la tendencia de la temperatura de arco radiante (ver la Figura \ref{fig:graph-air_excess-arc}) para el aumento exceso de aire o humedad relativa, debido que para estos dos casos la ecuación de temperatura de arco debe considerar el calor extra que se debe ceder a estos componentes en la cámara de combustión, como fue descrito en la sección de combustión de hidrocarburos.

\begin{figure}[hbt]
\begin{center}
\includegraphics[scale=0.38]{images/graph-air_excess-arc}
\caption[Temperatura de arco radiante vs Exceso de aire]{Temperatura de arco radiante en función del exceso de aire en el combustible.}
\label{fig:graph-air_excess-arc}
\end{center}
\end{figure}

\par Para la variable de tasa de distribución de calor absorbido por zonas la tendencia observada en la Figura \ref{fig:graph-t_out-dist} es decreciente, lo que se traduce en que la absorción de calor disminuye en la zona radiante y aumenta en la zona convectiva. Esto debido a que la zona convectiva es más sensible al cambio de temperatura de los gases de combustión, tiene un coeficiente de transferencia de calor global mayor y un área de contacto casi diez veces mayor por la presencia de aletas. No obstante, la limitación física del material de la aletas a la temperatura no es considerada en esta simulación, para más detalle se reporta la temperatura de la punta de las aletas en la sección de resultados ampliados.

\begin{figure}[hbt]
\begin{center}
\includegraphics[scale=0.38]{images/graph-t_out-dist}
\caption[Distribución de absorción de calor vs Temperatura de salida de residuo]{Tasa de distribución de absorción de calor entre zona radiante y convectiva en función de la temperatura de salida de residuo.}
\label{fig:graph-t_out-dist}
\end{center}
\end{figure}

\par Comprobando el comportamiento esperado de la eficiencia, se observa una tendencia decreciente en la Figura \ref{fig:graph-t_out-efic}, al aumentar cualquiera de las variables de entrada seleccionadas para este análisis, un comportamiento inversamente proporcional al aumento de la temperatura de los gases en la chimenea.

\begin{figure}[hbt]
\begin{center}
\includegraphics[scale=0.38]{images/graph-t_out-efic}
\caption[Eficiencia Térmica vs Temperatura de salida de residuo]{Eficiencia Térmica (@ Valor Calorífico Neto) en función de la temperatura de salida de residuo.}
\label{fig:graph-t_out-efic}
\end{center}
\end{figure}

\par La tendencia al incremento del flujo de combustible que se observa para todas las variables estudiadas, es directamente proporcional a las emisiones de \ac{co2}, lo que puede corroborarse en la Figura \ref{fig:graph-t_out-emi}. Este comportamiento solo varia si se usa un combustible que no genere \ac{co2}, como al elegir \ac{h2} sin ningún hidrocarburo en la mezcla, donde no existirían estas emisiones.

\begin{figure}[hbt]
\begin{center}
\includegraphics[scale=0.38]{images/graph-t_out-emi}
\caption[Emisiones de CO$_2$ vs Temperatura de salida de residuo]{Emisiones de CO$_2$ en función de la temperatura de salida de residuo.}
\label{fig:graph-t_out-emi}
\end{center}
\end{figure}

\section{Interfaz de usuario (UI)}

\par Sin la interfaz de usuario, el algoritmo, aunque poderoso, tendría un uso muy limitado por la complejidad que implica correrlo directamente como un programa y por lo difícil que es observar los cambios sin una amigable visualización de datos.

\par La interfaz desarrollada consiste en un sitio web donde el usuario puede navegar a distintas vistas con diferentes propósitos. A continuación se describirán dichas vistas y sus aplicaciones pensadas, esencialmente las posibilidades educativas que ofrece la comparación de dos estados del proceso, pero no limitadas a estas.

\subsection{Alcances}

\par La primera vista, que se puede apreciar en la Figura \ref{fig:alcance}, es la bienvenida a los usuarios del simulador, es una página que muestra los alcances, usos y limitaciones del simulador, la primera sección introduce el algoritmo del simulador, seguido por una descripción del horno de referencia y sus condiciones de operación.

\begin{figure}[hbt]
\begin{center}
\includegraphics[scale=0.2]{images/alcance}
\caption[Página de alcances]{Página de alcances, vista introductoria a la aplicación web.}
\label{fig:alcance}
\end{center}
\end{figure}

\par La sección de interfaz describe como utilizar la barra de navegación del sitio web (ver Figura \ref{fig:navbar}), como editar el combustible a utilizar en las diferentes vistas de ingreso de datos y finalmente como usar la sección de gráficas o tendencias.

\begin{figure}[hbt]
\begin{center}
\includegraphics[scale=0.2]{images/navbar}
\caption[Barra de navegación]{Barra de navegación de la interfaz web.}
\label{fig:navbar}
\end{center}
\end{figure}

\subsection{Ingreso de datos}

\par Para la modificación de los datos de datos de entrada al simulador existen dos vistas, una resumida (ver Figura \ref{fig:datos}) que al accionar el cálculo dirige a la vista de resultados comparativos y otra vista ampliada (ver Figura \ref{fig:fulldatos}) que permite la edición de datos mas específicos y en esta vista se encuentra la posibilidad de accionar el calculo para ofrecer una página de resultados extendidos, o de graficar las tendencias escogiendo una variable a ser modificada.

\begin{figure}[hbt]
\begin{center}
\includegraphics[scale=0.2]{images/datos}
\caption[Página de ingreso de datos para comparación]{Página de ingreso de datos utilizada para accionar el cálculo en la vista comparativa de resultados.}
\label{fig:datos}
\end{center}
\end{figure}

\begin{figure}[hbt]
\begin{center}
\includegraphics[scale=0.2]{images/fulldatos}
\caption[Página de ingreso de datos extendida]{Página de ingreso de datos extendida, con la opción de accionar la vista de resultados extendidos o la generación de gráficas de tendencia.}
\label{fig:fulldatos}
\end{center}
\end{figure}

\subsection{Resultados}

\par Como se describió en la sección anterior, hay dos posibles vistas para los resultados, la comparativa, de dos resultados, y la extendida de un resultado puntual. 
\subsubsection{Resultados comparativos}

\par Como se observa en la Figura \ref{fig:resultados}, es una página enfocada en comparar dos estados del horno y permitir el análisis de la consecuencia de modificar una variable desde un estado base hacia un estado modificado, siendo esto lo mas común para un operador en la industria, la recomendación es modificar solo una variable a la vez para su correcta interpretación.
\begin{figure}[hbt]
\begin{center}
\includegraphics[scale=0.15]{images/resultadosdoble}
\caption[Página comparativa de resultados]{Página comparativa de resultados, donde se pueden detallar los cambios de un estado a otro en la operación del horno.}
\label{fig:resultados}
\end{center}
\end{figure}

\subsubsection{Resultados extendidos}
\par En esta vista (Figura \ref{fig:fullresultados}) se pueden detallar los valores obtenidos de todas las variables internas de cada sección en el simulador, su uso inicial fue al depurar las ecuaciones internas del algoritmo para encontrar sus puntos críticos y corregirlos, ahora permite apreciar las variables desglosados y seguir el comportamiento de los fenómenos desarrollados internamente.

\begin{figure}[hbt]
\begin{center}
\includegraphics[scale=0.18]{images/fullresultados.png}
\caption[Página extendida de resultados]{Página extendida de resultados, donde se pueden observar a detalle las variables internas del proceso simulado.}
\label{fig:fullresultados}
\end{center}
\end{figure}

\subsection{Gráficas}

\par Esta página es pensada para los usuarios más curiosos, y para futuras modificaciones del algoritmo o extensión de sus rangos (ver Figura \ref{fig:graficas}).

\par Aquí pueden ser comprobadas las tendencias, la continuidad y la estabilidad de las variables resultantes. Fue ampliamente utilizada en el desarrollo del algoritmo para definir los detalles de los métodos numéricos usados en cada sección del simulador.

\begin{figure}[hbt]
\begin{center}
\includegraphics[scale=0.18]{images/graficas.png}
\caption[Página de gráficas]{Página de gráficas, muestra las tendencias de las variables resultantes escogidas en el rango de variación determinado por el usuario.}
\label{fig:graficas}
\end{center}
\end{figure}

% 3-      El modelo debe ser presentado como una herramienta amplia que está limitado por las características físicas del horno: área de transferencia de calor en las tres zonas de intercambio, capacidad de los quemadores y las dimensiones de la chimenea. Es decir, el diseño mecánico del horno es invariable.

% 5-      El modelo permite variar el volumen del fluido del proceso, temperatura de entrada y salida del fluido del proceso, condiciones ambientales, composición del gas de combustión y relación A/C. Las propiedades del fluido, Cp, viscosidad y gravedad específica que utiliza el modelo corresponden a la temperatura de entrada y salida y deben ser introducidas como datos. Las propiedades del gas de combustión son calculadas a partir de sus componentes.

% 6-      No existe impedimento para variar el fluido del proceso en un amplio margen de crudos con diferentes API u otras corrientes de refinería. Esto debe ser recalcado entre los puntos importantes de las virtudes del modelo.

% 7-      En forma ilustrativa, se debe mantener el fluido del diseño para demostrar la validez del modelo y  la tendencia de los resultados ante la variación de los parámetros independientes.

% 8-      Deben quedar claro las limitantes del modelo en aquellos elementos no considerados cuantitativamente (ej. formación de CO, despegue de la llama por exceso de aire, etc) y que han sido tratados mediante rangos empíricos.
%\chapter{SOBRE EL USO DE ACR\'ONIMOS Y LA LISTA DE S\'IMBOLOS}

\section{Lista de s\'imbolos}
La lista de s\'imbolos o notaci\'on matem\'atica se recomienda hacer manualmente. Por ejemplo, se puede incluir el siguiente c\'odigo luego de los \'indices.
\begin{verbatim}
\chapter*{Notaci\'on matem\'atica}
\begin{tabular}{ll}
$\mathbb{R}$ & Conjunto de n\'umeros reales\\
$M_{m,n}$ & Espacio de las matrices de tama\~no $m$ por 
     $n$ con entradas reales\\
$\mathcal{L}$ & Operador de Laplace\\
$\emptyset$ & Conjunto vac\'io
\end{tabular}
\end{verbatim}
\chapter*{Conclusiones}

%\begin{itemize}
    %\item 
    \par Se desarrollo exitosamente un modelo capaz de simular la operación de hornos de proceso, enfocado al adiestramiento de ingenieros y operadores, y con el objetivo de aumentar la eficiencia de estos equipos para disminuir el consumo de combustibles fósiles y generar menos gases de efecto invernadero.

    %\item 
    \par El modelo integra las ecuaciones de transferencia de calor y conservación de masa y energía en un algoritmo que simula el comportamiento estacionario de un horno de fuego directo, se comprobó su capacidad para generar resultados estables y confiables de las variables de proceso establecidas.
    
    %\item 
    \par El simulador para adiestramiento de hornos de proceso es una herramienta computacional de disposición pública y de código abierto, escrita en el lenguaje de programación JavaScript, que refiere parámetros confiables y prácticos para el uso académico o industrial dentro de las características mecánicas establecidas. El acceso directo se encuentra siguiendo el enlace (\url{https://e-usb.github.io/heater}).
    
    %\item 
    \par Los resultados generados por el simulador fueron validados aceptablemente mediante su comparación con un software comercial empleado para el diseño y evaluación termomecánica de hornos de proceso. Las tendencias de las variables operacionales simuladas mostraron resultados totalmente coherentes con la operación real de estos hornos en refinerías y plantas petroquímicas.

    %\item 
    \par Se incrementó el alcance del algoritmo desarrollado con la implementación de un modo comparativo y un modo de visualización de tendencias, aumentando las opciones de los usuarios al interactuar con el simulador.
%\end{itemize}
\nocite{*}
\bibliography{referencias}
\appendix
\chapter{Código desarrollado}

La versión organizada por archivos y carpetas se encuentra en el siguiente enlace \url{http://github.com}
\chapter{FIGURAS AMPLIADAS}\label{apx:img}

\begin{figure}[ht]
\includegraphics[scale=0.6,angle=0]{images/diagrama-algo}
\caption[Diagrama Algoritmo Ampliado]{Diagrama detallado del algoritmo desarrollado para la simulación del horno de proceso}\label{img:dia-algo-full}
\end{figure}

\begin{figure}[ht]
\includegraphics[scale=0.6,angle=0]{images/diagrama-meca}
\caption[Diagrama Mecánico Ampliado]{Diagrama descriptivo de la estructura mecánica del horno simulado}\label{img:dia-meca-full}
\end{figure}
\chapter{Código desarrollado}\label{apx:code}

\par La versión organizada por archivos y carpetas se encuentra en el siguiente enlace \url{https://github.com/e-usb/heater}

\par El código usado únicamente para el algoritmo, sin incluir la interfaz, se muestra a continuación.

\section{Ciclo externo}

\begin{verbatim}
const { round, logger, options, unitConv, 
  initSystem, linearApprox, newtonRaphson, 
  viscosityApprox, kw_tubes_A312_TP321
} = require('./js/utils');
const data = require('./data/data.json');
const {radSection} = require('./js/heaterSections/rad');
const {convSection} = require('./js/heaterSections/conv');
const {shieldSection} = require('./js/heaterSections/shield');
const {combSection} = require('./js/heaterSections/combustion');
const {browserProcess} = require('./js/browser');

const createParams = (opts) => {
  const
    m_fluid = unitConv.BPDtolb_h(
      unitConv.lbtokg(opts.mFluid),
      opts.spGrav
    ), // kg/h
    t_in  = opts.tIn, // (K)
    t_out = opts.tOut,// (K)
    miu_fluid_in = opts.miuFluidIn,  // (cp)
    miu_fluid_out= opts.miuFluidOut, // (cp)
    cp_fluid_in  = opts.cpFluidIn, // (kJ/kg-C)
    cp_fluid_out = opts.cpFluidOut,// (kJ/kg-C) 
    kw_fluid_in = unitConv.kwENtokwSI(opts.kwFluidIn), // (kJ/h-m-C)
    kw_fluid_out= unitConv.kwENtokwSI(opts.kwFluidOut);// (kJ/h-m-C)

  return {
    runDistCycle: opts.runDistCycle,
    /** Inlet Amb Variables */
    p_atm:  opts.pAtm,         // (Pa) 
    t_fuel: opts.tFuel,        // (K) 
    t_air:  opts.tAir,         // (K)
    t_amb:  opts.tAmb,         // (K) ref
    humidity:  opts.humidity,  // (%) 
    airExcess: opts.airExcess, // (% *.01) 
    o2Excess:  opts.o2Excess,  // (% *.01) 
    
    /** Process Variables */
    sp_grav:   opts.spGrav, // -
    t_in_conv:  t_in,       // (K) global process inlet
    t_out:      t_out,      // (K) global process outlet
    m_fluid:    m_fluid,    // (kg/h) 
    Rfi:      opts.rfi,     // (h-m2-C/kJ) int. fouling rad
    Rfo:      opts.rfoConv, // (h-m2-C/kJ) ext. fouling cnv
    Rfi_conv: opts.rfiConv, // (h-m2-C/kJ) int. fouling conv sect
    Rfi_shld: opts.rfiShld, // (h-m2-C/kJ) int. fouling shld sect
    Rfo_shld: opts.rfoShld, // (h-m2-C/kJ) ext. fouling shld sect
    efficiency: opts.effcy,         // (% *.01)
    duty_rad_dist: opts.radDist,    // (% *.01)
    heat_loss_percent: opts.hLoss,  // (% *.01)
    max_duty: unitConv.BTUtokJ(71.5276*1e3),// (kJ/h) unused
    miu_fluid: viscosityApprox({
      t1: t_in,  v1: miu_fluid_in,
      t2: t_out, v2: miu_fluid_out
    }),                     // (cP)
    Cp_fluid: linearApprox({
      x1: t_in,  y1: cp_fluid_in,
      x2: t_out, y2: cp_fluid_out
    }),                     // (kJ/kg-C) 
    kw_fluid: linearApprox({
      x1: t_in,  y1: kw_fluid_in,
      x2: t_out, y2: kw_fluid_out
    }),                     // (kJ/h-m-C)
    
    /** Mechanic variables for heater */
    Material: 'A-312 TP321',
    h_conv: unitConv.hcENtohcSI(1.5),// (kJ/h-m2-C)
    kw_tube: kw_tubes_A312_TP321,    // (kJ/h-m-C)
    Pass_number: 2,          // - number of tube passes
    
    Pitch_rad: unitConv.intom(2*8),// (m) NPS * 2
    N_rad:  42,                    // - number of tubes 
    L_rad:  unitConv.fttom(62.094),// (m) tube effective length
    Do_rad: unitConv.intom(8.625), // (m) tube external diameter
    Sch_rad:unitConv.intom(0.322), // (m) Schedule thickness

    Burner_number: 13,            // - burner's number
    Do_Burner:   2.24,            // (ft) burner's outside diameter

    Width_rad:  17.50,            // (ft) width in rad sect
    Length_rad: 64.55,            // (ft) length in rad sect
    Height_rad: 27.00,            // (ft) height in rad sect
    
    N_shld: 16,                   // - number of tubes 
    L_shld: unitConv.fttom(60),   // (m) effective tube length
    Do_shld:unitConv.intom(6.625),// (m) external diameter 
    
    Pitch_sh_cnv: unitConv.intom(2*6),// (m) NPS * 2
    Sch_sh_cnv:  unitConv.intom(0.28),// (m) Schedule thickness
    Tpr_sh_cnv:  8,                   // - number of tubes per row

    N_conv: 40,                   // - number of tubes 
    L_conv: unitConv.fttom(60),   // (m) effective tube length
    Do_conv:unitConv.intom(6.625),// (m) external diameter 

    // Fin properties
    Nf:unitConv.mtoft(60),   // (#/m) Fin's number per meter
    Tf:unitConv.fttom(.005), // (m) Fin's thickness
    Lf:unitConv.fttom(0.08), // (m) Fin's height

    /** Miscellaneous */
    FinType: 'Solid',
    FinMaterial: '11.5-13.5Cr',
    FinArrange: 'Staggered Pitch',
    verbose: opts.verbose,       // True or False
    unitSystem: opts.unitSystem, // SI or English
    lang: opts.lang,             // EN or ES
    NROptions: opts.NROptions,   // {object options}
    units: initSystem(opts.unitSystem)
  }
}

const heaterFunc = (fuels, opts) => {
  const params = createParams(opts);

  // if params.o2Excess is set, start airExcess iteration
  if (params.o2Excess != 0) combustionCycle(params, fuels);

  const heat_result = combSection(params.airExcess, fuels, params);

  if (params.runDistCycle) externalCycle(params);

  heat_result.rad_result = radSection(params);
  heat_result.shld_result = shieldSection(params);
  heat_result.conv_result = convSection(params);
  heat_result.rad_result.eff_thermal_val = 
    heat_result.rad_result.eff_thermal(heat_result.conv_result.Q_stack);
  heat_result.rad_result.eff_gcv_val = 
    heat_result.rad_result.eff_gcv(heat_result.conv_result.Q_stack);

  return heat_result
}

const externalCycle = (params) => {
  // cycle iter count and flag for debugging logs 
  let cycle = 0, noLog = true;
  const rad_dist = (radDist) => {
    cycle++;
    if (radDist >0.3 && radDist <1) {
      params.duty_rad_dist = radDist;
    }
    const int_rlt = {
      rad:  radSection(   params, noLog),
      shld: shieldSection(params, noLog),
      conv: convSection(  params, noLog)
    };
    const duty_calc = Math.abs(int_rlt.rad.Q_fluid) + 
    Math.abs(int_rlt.shld.Q_fluid) + Math.abs(int_rlt.conv.Q_fluid);

    return (params.duty - duty_calc)/duty_calc;
  };
  const convNROptions = {...params.NROptions};
  convNROptions.maxIterations *= 5;
  convNROptions.tolerance *= 1e-1;
  convNROptions.epsilon *= 1e-1;
  convNROptions.h *= 1e-1;
  const rad_dist_final = newtonRaphson(rad_dist, 
    params.duty_rad_dist, convNROptions, 'rad_dist_final');
  if (rad_dist_final >0.1 && rad_dist_final <1) { 
    params.duty_rad_dist = rad_dist_final; 
  } else {
    logger.error('external cycle broken, error in rad_dist estimation, using: '+
    params.duty_rad_dist);
  }
  logger.info(`duty_rad_dist: ${
    round(100*rad_dist_final,2)}, ext_cycle_reps: ${cycle}`);
}

const combustionCycle = (params, fuels) => {
  // cycle iter count and flag for debugging logs 
  let cycle = 0, onlyO2 = true;
  const comb_o2 = (airExcessVal) => {
    cycle++;
    const combO2 = combSection(airExcessVal, fuels, params, onlyO2)
    if (!onlyO2) logger.info( `'O2%_comb': ${combO2.flows['O2_%']}, `+
      `O2excess: ${params.o2Excess *100}`);

    return Math.round(combO2.flows['O2_%']*1e5 -params.o2Excess*1e7);
  }
  const convNROptions = {...params.NROptions};
  convNROptions.maxIterations *= 5;
  convNROptions.tolerance *= 1e-1;
  convNROptions.epsilon *= 1e-1;
  convNROptions.h *= 1e-1;
  const airExcess = newtonRaphson(comb_o2,.05,
    convNROptions, 'o2_excess_to_air');

  if (airExcess) params.airExcess = airExcess;
  logger.info(`'air_excess': ${round(100*airExcess,2)}, `+
    `'comb_cycle_reps': ${cycle}`);
}

let fuelsObject = { 
  H2:     .1142, N2:   .0068, CO:   .0066, CO2: .0254, 
  CH4:    .5647, C2H6: .1515, C3H8: .0622, C4H10: .0176, 
  iC4H10: .0075, C2H4: .0158, C3H6: .0277,
};
// Fuel for debugging purpose
// fuelsObject ={
//   CH4: 1,
//   // H2: .7, O2: .2, N2: .1
// };

/** App entry point */
if (typeof window !== 'undefined') {
  browserProcess(fuelsObject, data, options, heaterFunc)
} else {
  logger.info(JSON.stringify(heaterFunc(fuelsObject, options), null, 2))
}
\end{verbatim}

\subsection{Combustión}
\begin{verbatim}
const { newtonRaphson, options, logger,
  round, roundDict, initSystem,
  normalize, flueViscosity, flueThermalCond
} = require('../utils');
const data = require('../../data/data.json');
const dryAirN2Percentage = 79.05;
const dryAirO2Percentage = 20.95;
const N2O2relation = dryAirN2Percentage/dryAirO2Percentage;
const dryAir = {
  O2: .01 * dryAirO2Percentage,
  N2: .01 * dryAirN2Percentage,
  H2O: 0
};

/** Check if the percentages of the fuels sums 100%.
 * In case of check fail an error will be attached to the result.
*/
const checkObjectFraction = (fuels, result = {}) => {
  const total = Object.values(fuels).reduce((acc, value)=> acc + value)
  const tolerance = 3e-12
  const check1 = Math.abs(1 - total) <= tolerance
  if (!check1) result.err += `[fuel fraction not equal to 1,` + 
    ` total: ${total}. fuels: ${Object.keys(fuels)}],`;
  return check1;
};

/** Check if all the components in the fuels are in the data filtered.
 * In case of a bad fuel entered an error will be attached to the result.
*/
const checkFuelData = (fuels, compounds, result = {}) => {
  const badFuels = Math.abs(compounds.length - 
    Object.keys(fuels).length);
  const check1 = badFuels === 0;
  if (!check1) {
    logger.error(`[some fuels aren't in the database, #badFuels: ${badFuels}],`);
    result.err += `[some fuels aren't in the database, #badFuels: ${badFuels}],`;
  }
  return check1;
};

/** (kJ/kg K) to call returning function use Kelvin units 
  * if you want a result in (kJ/kmol K) units, multiply the
  * result by MW or call this with second argument set to true.
 */
const Cp0 = ({c0, c1, c2, c3, MW, Substance}, molResult, noLog) => {
  // Cp equation from table A.6 Van Wylen
  // Teta = T(Kelvin)
  return (teta) => {
    // Approximate equation valid from 250 K to 1200 K.
    if (teta < 250 && !noLog) logger.debug(`"Cp0", "temp": "${round(teta)}",`+
      `"Msg": "${Substance} bellow range for Cp0 formula"`);
    if (teta > 1200&& !noLog) logger.debug(`"Cp0", "temp": "${round(teta)}",`+
      `"Msg": "${Substance} above range for Cp0 formula"`);
    if (c0 === "-") {
      logger.debug(`"Cp0", "Msg": "Wrong use, called for compound `+
      `${Substance}, no data found"`);
      return 0;
    }
    if (molResult) return MW*(c0 + c1*(teta*.001) + 
        c2*(teta*.001)**2 + c3*(teta*.001)**3)
    return (c0 + c1*(teta*.001) + c2*(teta*.001)**2 + c3*(teta*.001)**3)
  }
};

/** (kJ/kg K) argument needs to be a fuel object,
* ie: { CH4: 0.323, ... }
* if you want a result in (kJ/kmol K) units, call it with 
* second argument set to true.
*/
const Cp_multicomp = (fuels, molResult, noLog) => {
  if (fuels.length === 0) return (_t) => 0
  // making a deep copy and normalize if needed
  let normalFuel = JSON.parse(JSON.stringify(fuels));
  if (!checkObjectFraction(fuels)) 
    normalFuel = normalize(normalFuel, "Cp_multicomp", noLog);
  const fuelComps = data.filter(elem => elem.Formula in normalFuel);
  const cps = [];
  let i = 0;
  for (const fuel in normalFuel) {
    cps[i] = (t) => normalFuel[fuel] * Cp0(
      fuelComps.filter(elem => elem.Formula == fuel)[0], molResult
    )(t);
    i++;
  }
  
  return cps.reduce((acc, val) => ((t) => acc(t) + val(t)), (_t) => 0);
};

/** (kg/kmol) argument needs to be a fuel object,
* ie: { CH4: 0.323, ... }
*/
const MW_multicomp = (fuels, noLog) => {
  if (fuels.length === 0) return (_t) => 0;
  // making a deep copy and normalize if needed
  let normalFuel = JSON.parse(JSON.stringify(fuels));
  if (!checkObjectFraction(fuels)) 
    normalFuel = normalize(normalFuel, "MW_multicomp", noLog);
  const fuelComps = data.filter(elem => elem.Formula in normalFuel);
  let MWs = 0;
  for (const fuel in normalFuel) {
    MWs += fuelComps.filter(elem => elem.Formula == fuel)[0].MW * 
    normalFuel[fuel];
  }
  return MWs;
};

/** (Pa) Temperature should be in K, humidity %[0,100] */
const pressureH2OinAir = (temperature, relativeHumidity) => {
  // Equation from Reference: Tetens, O., 1930

  // This eq uses temp in °C
  const temp = temperature - options.tempToK;
  // ps is the saturation vapour pressure, in pascals,
  const ps = 610.78*Math.exp(temp/(temp+238.3)*17.2694);
  // result pw is the actual water vapour pressure.
  return ps * relativeHumidity * 0.01;
};

/** Temperature should be in K, humidity %[0,100] */
const moistAirWeightRatio = (temperature, relativeHumidity) => {
  const pw = pressureH2OinAir(temperature, relativeHumidity)
  // returned value is the weight ratio of water vapour and dry air.
  // (kg-w_vap/kg-dry_a)
  return data[31].MW * pw / 
  ( MW_multicomp(dryAir) * (options.pAtm - pw ) );
  // a simplification can be: 0.62 * 1e-5 * pw
  
  /* weight ratio converted to water per oxygen in air
  const w = data[31].MW * pw / 
    ( MW_multicomp(dryAir) * (options.pAtm - pw ) );
  return w * 7.655;
  //*/
};

/** (kJ/kmol), Enthalpy of formation plus delta enthalpy 
  * returns a function if no temp is passed */
const deltaH = (compound, t) => {
  if (compound.Cp0 === '-') {
    if (compound.h0 === '-') {
      logger.warn(`wrong use of deltaH func,`+
        ` called for compound ${compound.Substance} without data`);
      if (t === undefined) return () => 0;
      return 0;
    }
    if (t === undefined) return () => compound.h0
    return compound.h0;
  }
  // hf0 + deltaH(tempAmbRef -> t)
  if (t === undefined) return (temp) => compound.h0 +(temp-options.tempAmbRef)*
    Cp0(compound,true, true)((options.tempAmbRef+temp)/2);

  return compound.h0 + (t-options.tempAmbRef)*
    Cp0(compound,true, true)((options.tempAmbRef+t)/2);
};

/** (kJ/kmol), Enthalpy of combustion for a certain compound 
  * returns a function if no temp is passed */
const combustionH = (compound, t, tIni, liquidWater = false) => {
  // hrp = HP - HR // H = H0 + deltaH  // H0 = n(hf)
  // SR ni*(hf + deltaH)i = SP ne*(hf + deltaH)e

  const 
    co2_H = deltaH( data[6]  ),
    so2_H = deltaH( data[34] ),
    o2_H  = deltaH( data[2]  );
  let h2o_H = deltaH(data[31]); // gas lower heating value 
  // higher heating value
  if (liquidWater === true) h2o_H = deltaH(data[32]); // liq
  // making tIni equal to t_amb if not specified
  if (tIni === undefined) tIni = options.tAmb;

  if (t === undefined) return (tempParam) => compound.CO2*co2_H(tempParam) 
    + compound.SO2*so2_H(tempParam) +compound.H2O*h2o_H(tempParam)
    - deltaH(compound)(tIni) - compound.O2*o2_H(tIni);
  
  // SR ni*(hf + deltaH)i = SP ne*(hf + deltaH)e
  return ( compound.CO2*co2_H(t) +compound.SO2*so2_H(t) +compound.H2O*h2o_H(t)
    - deltaH(compound)(tIni) - compound.O2*o2_H(tIni) );
};

/** (kJ/kg) Enthalpy of combustion for a certain fuel mix */
const ncv = (fuels, products, compounds, tAmb, gcv = false) => {
  let value = 0;
  for (const fuel in fuels) {
    if (fuel in products) continue;
    const compound = compounds.filter(elem => elem.Formula == fuel)[0]
    value += fuels[fuel]*combustionH(compound, undefined, tAmb, gcv)(tAmb);
    //logger.info(`H of combustion for ${fuel}: ` +
    // `${combustionH(compound)(tAmb)/compound.MW} KJ/Kg` )
  }
  return value;
};

/**Temp unit (K)
* Internal units (kJ/kmol)
* Function to create the adiabatic flame equation 
* used in the newton raphson method to find adFlame temp */
const adFlame = (normalFuels, products, tIni, o2required) => {
  if (tIni === undefined) tIni = options.tAmb;
  if (o2required === undefined) o2required = 0;
  const 
    fuelCompounds = data.filter(elem => elem.Formula in normalFuels),
    // ..._H = (t) => hf0 + MW * Cp(t_prom) * (t - t_amb)
    o2_H =  deltaH(data.filter(elem => elem.Formula == "O2")[0]),
    n2_H =  deltaH(data.filter(elem => elem.Formula == "N2")[0]),
    co2_H = deltaH(data.filter(elem => elem.Formula == "CO2")[0]),
    h2o_H = deltaH(data.filter(elem => elem.Formula == "H2O")[0]),
    so2_H = deltaH(data.filter(elem => elem.Formula == "SO2")[0]);

  // Products enthalpy at the new temp minus dry air inlet
  const pEnthalpy = (t) => products.O2*o2_H(t) + products.SO2*so2_H(t) +
    products.H2O*h2o_H(t) + products.CO2*co2_H(t) + 
    products.N2*n2_H(t) - products.N2*n2_H(tIni) - o2required*o2_H(tIni);
  
  // Reactants enthalpy (fuel)
  const rEnthalpy = [];
  let i = 0;
  for (const fuel in normalFuels) {
    // fuelFraction * [hf0 + MW * Cp(t_prom) * (tIni - t_amb)]
    rEnthalpy[i] = normalFuels[fuel]*deltaH(
      fuelCompounds.filter(elem => elem.Formula == fuel)[0]
    )(tIni);
    i++;
  }
  
  // SR ni*(hf + deltaH)i = SP ne*(hf + deltaH)e
  return (t) => pEnthalpy(t) - rEnthalpy.reduce((acc, value)=> acc + value);
};

/** For every element in the fuel compounds 
 * calculates every product of combustion per fuel element
 * filling the product object.
*/
const combPerFuelCompound = (compounds, products, normalFuel) => {
  for (const elem of compounds) {
    for (const product in products) {
      if (product == 'N2') {
        if (elem['Formula'] == 'N2' || elem['Formula'] =='"N2a') {
          products[product] += normalFuel[elem['Formula']];
          continue;
        }
        products[product] += elem['O2']*normalFuel[elem['Formula']]*N2O2relation;
        continue;
      }
      products[product] += elem[product]*normalFuel[elem['Formula']];
      // logger.default(`${elem['Formula']} req = ${product} ` +
      //   `${elem[product]*normalFuel[elem['Formula']]}` )
    }
  }
}

/** In this process the params object will be updated
*  in every function call with the combustion data
*/
const combSection = (airExcess, fuels, params, onlyO2) => {
  if (!onlyO2) logger.debug(`"airExcess", "value": ${airExcess}`);
  const units = initSystem(params.unitSystem);
  const moisture_val = moistAirWeightRatio(
    params.t_air, params.humidity
  );
  const debug_data = {
    err: "",
    atmPressure:     units.pressure(params.p_atm),
    fuelTemperature: units.tempC(params.t_fuel,0),
    ambTemperature:  units.tempC(params.t_amb,0),
    airTemperature:  units.tempC(params.t_air,0),
    "humidity_%": params.humidity,
    "dryAirN2_%": round(dryAirN2Percentage,2),
    "dryAirO2_%": round(dryAirO2Percentage,2),
    moisture:   units.moist(moisture_val),
    spGrav: params.sp_grav,
    cpFluidTb: units.cp(params.Cp_fluid((params.t_in_conv + params.t_out) /2)),
    unitSystem: units.system[params.lang]
  };
  const compounds = data.filter((elem, _i, _arr) => elem.Formula in fuels)

  let normalFuel = {...fuels};
  if (!checkObjectFraction(fuels, debug_data)) 
    normalFuel = normalize(fuels, "combSection");
  checkFuelData(normalFuel, compounds, debug_data);

  const products = {O2:0, N2:0, H2O:0, CO2:0, SO2:0}, air = {...dryAir};

  // filling products object with stoichiometric ratio
  combPerFuelCompound(compounds, products, normalFuel);

  // air excess and humidity shouldn't be less than 0
  if (airExcess - 0.000001 < 0) airExcess = 0;
  if (params.humidity - 0.000001 < 0) params.humidity = 0;
  /** Percentage of O2 in excess = 100% + x% airExcess */
  let o2required = products['O2'];
  let o2excess = o2required * (1 + airExcess);
  // If O2 requirements are negative 
  if (products['O2'] <= 0 || products['N2'] < 0) {
    logger.error(`airExcess set to 0, O2 in fuel >= O2 needed.`+
    ` Products: {O2:${products['O2']}, N2:${products['N2']}}`);
    o2excess   = 0;
    o2required = 0;
    products['N2'] = normalFuel['N2'];
    products['O2'] = -products['O2'];
  } else {
    const 
      waterPressure  = pressureH2OinAir(params.t_air, params.humidity),
      dryAirPressure = params.p_atm - waterPressure;
    air.N2  = .01* dryAirN2Percentage* dryAirPressure / params.p_atm;
    air.O2  = .01* dryAirO2Percentage* dryAirPressure / params.p_atm;
    air.H2O = waterPressure / params.p_atm;
    
    debug_data.dryAirPressure = units.pressure(dryAirPressure);
    debug_data.waterPressure  = units.pressure(waterPressure);
    debug_data["H2OPressure_%"] = round(100 * air.H2O);
    debug_data["N2Pressure_%"]  = round(100 * air.N2);
    debug_data["O2Pressure_%"]  = round(100 *air.O2);

    products['O2'] = o2excess -products['O2']; // Subs O2 used in combustion
    products['N2']  += products['O2']*(air.N2/air.O2);
    products['H2O'] += products['N2']*(waterPressure/(air.N2*params.p_atm));
  }

  let totalPerMol = 0, totalPerM_Dry = 0;
  for (const product in products) {
    totalPerMol += products[product];
    if (product !== 'H2O') totalPerM_Dry += products[product];
  }
  const flows = {
    total_flow: totalPerMol,
    dry_total_flow: totalPerM_Dry,
    // 'O2%_DRY': 100*products['O2'] /totalPerM_Dry,
    // 'CO2%_DRY':100*products['CO2']/totalPerM_Dry,
    // 'N2%_DRY': 100*products['N2'] /totalPerM_Dry,
    'N2_%': 100*products['N2'] /totalPerMol,
    'H2O_%':100*products['H2O']/totalPerMol,
    'CO2_%':100*products['CO2']/totalPerMol,
    'O2_%': 100*products['O2'] /totalPerMol,
    
    moisture_val,

    O2_mol_req_theor: o2required,
    O2_mass_req_theor:units.mass(o2required * data[2].MW),
    'air_excess_%':   100 * params.airExcess,
    AC:               o2excess / air.O2,
    AC_theor_dryAir:  o2required / (.01 * dryAirO2Percentage),
    AC_mass:          o2excess / air.O2 * 
      MW_multicomp(air)/MW_multicomp(normalFuel),
    AC_mass_theor_moistAir: o2required / air.O2 * 
      MW_multicomp(air)/MW_multicomp(normalFuel),

    fuel_MW: MW_multicomp(normalFuel),
    Cp_fuel: Cp_multicomp(normalFuel),

    flue_MW: MW_multicomp(products,onlyO2),
    Cp_flue: Cp_multicomp(products,false,onlyO2)
  };

  if (onlyO2) return {flows, products, debug_data};

  /** Adding results in parameters to be used in following sections */

  params.m_flue_ratio = totalPerMol * flows.flue_MW/MW_multicomp(normalFuel);  
  // kg/h
  params.m_air_ratio  = o2excess / air.O2 *
  MW_multicomp(air)/MW_multicomp(normalFuel); // kg/h

  params.Pco2 = products['CO2']/totalPerMol; // fraction
  params.Ph2o = products['H2O']/totalPerMol; // fraction

  // Functions of temp (kJ/kg-K)
  params.Cp_air  = Cp_multicomp(air);
  params.Cp_fuel = Cp_multicomp(normalFuel);
  params.Cp_flue = flows.Cp_flue;
  params.miu_flue= flueViscosity( data, products );
  params.kw_flue = flueThermalCond(data, products);
  flows.Cp_fuel_val  = flows.Cp_fuel(params.t_fuel);
  flows.Cp_fuel  = units.cp(flows.Cp_fuel_val);
  flows.Cp_flue  = units.cp(flows.Cp_flue(params.t_air));
  flows.flue_MW  = units["mass/mol"](flows.flue_MW);

  params.NCV = -ncv(normalFuel, products, compounds, params.t_amb)/
  MW_multicomp(normalFuel); // kJ/kg
  params.GCV = -ncv(normalFuel, products, compounds, params.t_amb, true)/
  MW_multicomp(normalFuel); // kJ/kg
  flows.NCV = units["energy/mass"](params.NCV,0);
  flows.GCV = units["energy/mass"](params.GCV,0);
  flows.NCV_val = params.NCV;
  flows.GCV_val = params.GCV;

  params.adFlame = newtonRaphson(
    adFlame(normalFuel, products, params.t_amb, o2excess),
    2000, params.NROptions, "fuel_adFlame");
  logger.info( `Adiabatic flame temp: [${round(params.adFlame)} K]`+
    ` ${units.tempC(params.adFlame)}`);

  roundDict(products);
  if (debug_data.err == "") delete debug_data.err;
  return {flows, products, debug_data};
};

module.exports = {
  combSection
};
\end{verbatim}

\subsection{Zona Radiante}
\begin{verbatim}
const {newtonRaphson, logger, round, unitConv} = require('../utils');

/** radSection receives the parameters dictionary and
 * calculates the required mass fluid or the necessary
 * temperature change at the radiant section.
 * 
 * First assumption to start the calc is duty transferred
 * in radiant section = 70% of total duty
 * 
 * Q_in = Q_rls + Q_air + Q_fuel =
 * Q_out = Q_R + Q_shield + Q_losses + Q_flue
 * Q_in(tg_out) - Q_out(tg_out) ~= 0
 * 
 * Q_rad + Q_conv = Q_R = Q_fluid(out-in)
*/
const radSection = (params, noLog) => {
  /** There are two starting cases A & B
   * Case A: given the fluid temp at the exit point of heater.
   * Case B: given the flow mass of the fuel at the heater burners. 
   * */
  let 
    tg_out = 0, // (K) Leaving/effective gas temp
    t_in   = 0, // (K) Inlet fluid temp
    t_out  = params.t_out,  // (K) Outlet fluid temp
    m_fuel = params.m_fuel; // (kg/h)

  let 
    duty_total  = 0, // (kJ/h) Duty in the hole fired heater
    duty_rad = 0; // (kJ/h) Duty in the radiant section
  
  const // Temperatures
    t_air     = params.t_air,    // (K) Inlet air temp
    t_fuel    = params.t_fuel,   // (K) Inlet fuel temp
    t_amb     = params.t_amb,    // (K) Ambient temp
    t_in_conv = params.t_in_conv,// (K) Heater inlet fluid temp
    /** (K) bulk temp (second arg default to rad inlet fluid temp) */
    Tb = (tOut, tIn = t_in) => 0.5*(tIn + tOut);

  const // Fired heater parameters
    Rfi = params.Rfi,      // (h-m2-C/kJ) internal fouling factor
    N = params.N_rad,      // (-) number of tubes in rad sect
    N_shld = params.N_shld,// (-) number of tubes in shld sect
    L = params.L_rad,      // (m) effective tube length
    L_shld = params.L_shld,// (m) effective tube length
    Do = params.Do_rad,    // (m) external diameter rad section
    Di = params.Do_rad - params.Sch_rad,// (m) int diameter rad sect
    S_tube = params.Pitch_rad || 0.394, // (m) center to center distance of tube
    S_tube_shld = params.Pitch_sh_cnv,    // (m) center to center distance of tube
    h_conv = params.h_conv || 30.66, 
    // (kJ/h-m2-C) Film convective heat transfer coff

    /** (ft) Mean Beam Length, dim ratio 1-2-1 to 1-2-4*/
    MBL = (2/3)*(params.Width_rad*params.Length_rad*params.Height_rad)**(1/3),
    PL  = (params.Ph2o + params.Pco2) * MBL, // atm-ft
    alpha = 1 +.49*(S_tube/Do)/6 -.09275*(S_tube/Do)**2 +
        .065*(S_tube/Do)**3/6 +.00025*(S_tube/Do)**4,
    alpha_shld =  1, // (-) alpha shield factor
    
    At = N *Math.PI *Do *L, // (m2) Bank tube's external surface area
    Acp = N * S_tube * L,   // (m2) Cold plane area of radiant tube bank
    Acp_shld = N_shld/2 *S_tube_shld*L_shld, 
    // (m2) Cold plane area of shield tube bank
    Ar = Ar_calc(params),   // (m2) Total refractory area
    {Aw, Aw_aAcp} = Aw_calc(alpha, Acp, alpha_shld, Acp_shld, Ar), // (m2)
    Ai = Math.PI*(Di**2)/2, // (m2) Tube's inside flux area x2
    cnv_fact = 3_600 *1e-3, // (g/s -> kg/h) secondsToHours * 1/k

    sigma = 5.670374e-8 *cnv_fact, // (W/m2-K4) -> (kJ/h-m2-K4)
    F = (temp) => effectivity(
        PL, alpha, Acp, alpha_shld, Acp_shld, Ar)(unitConv.KtoF(temp));

  const // Process Variables
    duty_rad_dist = params.duty_rad_dist         || .7,   // (-) % *.01
    efficiency = params.efficiency               || .8,   // (-) % *.01
    heat_loss_percent = params.heat_loss_percent || .015, // (-) % *.01
    NCV = params.NCV, // (kJ/kg) net calorific value
    GCV = params.GCV, // (kJ/kg) net calorific value
    m_fluid= params.m_fluid, // (kg/h) Fluid mass flow
    m_air  = (mFuel = m_fuel) => params.m_air_ratio*mFuel, // (kg/h) Air mass flow
    m_flue = (mFuel = m_fuel) => params.m_flue_ratio*mFuel,// (kg/h) Flue mass flow
    Cp_fuel = params.Cp_fuel(Tb(t_fuel, t_amb)), // (kJ/kg.K) Fuel mass heat
    Cp_air  = params.Cp_air( Tb(t_air, t_amb) ), // (kJ/kg.K) Air mass heat
    Cp_fluid = (tIn,tOut=tIn) => params.Cp_fluid(Tb(tIn, tOut)), // (kJ/kg.K)
    Cp_flue  = (tG,tG_out=tG) => params.Cp_flue(Tb(tG, tG_out)), // (kJ/kg.K)
    kw_fluid = (temp) => params.kw_fluid(temp),
    // (kJ/h-m-C) fluid thermal conductivity
    kw_tube  = (temp) => params.kw_tube(temp), 
    // (kJ/h-m-C - J/s-m-C-3.6) tube thermal conductivity
    miu_fluid= (temp) =>params.miu_fluid(temp),//(cP - g/m-s) fluid Viscosity
    G = (m_fluid/cnv_fact) /Ai, // Fluid mass speed inside radiant tubes
    prandtl = (t) => miu_fluid(t)*Cp_fluid(t)*cnv_fact/kw_fluid(t), // (miu*Cp/kw)
    reynolds = (t) => G * Di/miu_fluid(t); // (-) G*Di/miu
  
  const 
    /** (kJ/h-m2-C) internal heat transfer coff */
    hi = (tB,tW = tB) => .023 *(kw_fluid(tB) /Di) *reynolds(tB)**.8 *
      prandtl(tB)**(1/3) *(miu_fluid(tB)/miu_fluid(tW))**.14,
    /** Average tube wall temp (K) */
    Tw = (tB, tW = tB, dutyRad = duty_rad) => (dutyRad/At) *(Do/Di)* 
      (Rfi +1/hi(tB,tW) +(Di*Math.log(Do/Di)/(2*kw_tube(tW))) ) +tB;

  const // ******* Heat input to the radiant section ********
    Q_air    = (mFuel) => m_air(mFuel) *Cp_air *(t_air -t_amb), 
    // Sensible heat of air
    Q_fuel   = (mFuel) => mFuel * Cp_fuel*(t_fuel -t_amb), // Sensible heat of fuel
    Q_rls    = (mFuel) => mFuel * NCV, // Combustion heat of fuel
    Q_in     = (mFuel) => Q_rls(mFuel) + Q_air(mFuel) + Q_fuel(mFuel), // Heat input
    Q_rls_gcv= (mFuel) => mFuel * GCV, 
    // Combustion heat of fuel with liquid water out
    Q_in_gcv = (mFuel) => Q_rls_gcv(mFuel) + Q_air(mFuel) + Q_fuel(mFuel); 
    // Heat input
  
  const // ******* Heat taken out of radiant section ********
    Q_flue = (tG, mFuel) => m_flue(mFuel)*Cp_flue(tG,t_amb)*(tG-t_amb), 
    // Flue gases's sensible heat 
    Q_losses = (mFuel) => Q_rls(mFuel) *heat_loss_percent,    
    // Heat losses through setting
    Q_conv = (tG, tW) => h_conv * At * (tG - tW),             
    // Convective heat transfer
    Q_rad  = (tG, tW) => F(tG)*sigma*alpha*Acp*(tG**4-tW**4), 
    // Radiant heat transfer
    Q_shld = (tG, tW) => F(tG)*sigma*alpha_shld*Acp_shld*(tG**4-tW**4), 
    // Shld_rad heat transfer
    Q_R = (tG, tW) => Q_rad(tG,tW) + Q_conv(tG,tW), 
    // Heat absorbed by radiant tubes
    Q_out = (tG, mFuel=m_fuel, tW = Tw(Tb(t_out), Tw(Tb(t_out)))) => 
    Q_R(tG, tW) + Q_shld(tG, tW) + Q_losses(mFuel) + Q_flue(tG, mFuel);

  const Q_fluid = (tOut, tIn) => m_fluid*Cp_fluid(tIn,tOut)*(tOut -tIn); 
  // Fluid's sensible heat

  // **************************************************

  /* Calculating tg_out the option missing from given variables 
  (mass_fuel or temp_out) */

  if (t_out !== 0) { // Given temp_out
    duty_total = Q_fluid(t_out,t_in_conv); // Duty effective from t_out
    duty_rad = duty_total * duty_rad_dist; 
    // Calculate Tw with seed from 30-70 duty distribution
    // Approximating t_in_rad with assumption from duty distribution
    t_in = t_in_conv + 
      duty_total*(1 -duty_rad_dist)/(m_fluid*Cp_fluid(t_in_conv,t_out));

    // Calculating tg_out (effective gas temp)
    const tg_out_func = (tG) => Q_fluid(t_out,t_in) -
      Q_R(tG,Tw(Tb(t_out),Tw(Tb(t_out))));
    const flame = newtonRaphson(tg_out_func, 1000, 
      params.NROptions, "Tg_Tout-seed_radiant", noLog);
    if (flame) tg_out = flame;

    // Calculating fuel mass
    const m_fuel_func = (mFuel) => Q_in(mFuel) -
      Q_out(tg_out,mFuel,Tw(Tb(t_out),Tw(Tb(t_out))));
    let mass_fuel_seed = Q_fluid(t_out,t_in_conv) /(NCV*efficiency);
    if (!noLog) logger.debug(`"mass_fuel_seed", "value": "${mass_fuel_seed}"`);
    mass_fuel_seed = newtonRaphson(m_fuel_func, mass_fuel_seed,
      params.NROptions, "M-fuel_T-seed_radiant", noLog);
    if (mass_fuel_seed) m_fuel = mass_fuel_seed;

    duty_rad = Q_R(tg_out,Tw( Tb(t_out), Tw(Tb(t_out)) ));

  } else { // Given mass_fuel
    duty_total = Q_rls(m_fuel) *efficiency; 
    // Duty effective from from q release by fuel
    duty_rad = duty_total *duty_rad_dist; 
    // Calculate Tw with seed from 30-70 duty distribution
    // Approximating t_in_rad and t_out with efficiency and duty dist
    t_in = t_in_conv + duty_total*(1 -duty_rad_dist) /(m_fluid*Cp_fluid(t_in_conv));
    let t_out_seed = t_in_conv + duty_total /(m_fluid*Cp_fluid(t_in));

    // Calculating tg_out (effective gas temp)
    const tg_out_func = (tG) => Q_in(m_fuel) -
    Q_out(tG,m_fuel,Tw(Tb(t_out_seed),Tw(Tb(t_out_seed))));
    const flame = newtonRaphson(tg_out_func, 1000, 
      params.NROptions, "Tg_mFuel-seed_radiant", noLog);
    if (flame) tg_out = flame;

    // Calculating t_out 
    const t_out_func = (tOut) => Q_fluid(tOut,t_in) -
        Q_R(tg_out,Tw(Tb(tOut),Tw(Tb(tOut))));

    t_out_seed = newtonRaphson(t_out_func, t_out_seed, 
      params.NROptions, "Tout_mFuel-seed_radiant", noLog);
    if (t_out_seed) t_out = t_out_seed;

    // TODO: recalculation let t_out_recall = t_in - t_out + (Q_rad(tg_out) + 
    // Q_conv(tg_out)) / (m_fluid*Cp_fluid(t_in,t_out))
    const duty_recalculated = m_fluid*Cp_fluid(t_in,t_out)*(t_out -t_in_conv);
    const t_in_recalculated = params.t_in_conv + 
    duty_recalculated*(1 -duty_rad_dist)/(m_fluid*Cp_fluid(t_in,t_out));  
    // Discrepancies on recalculation
    if (!noLog) logger.info(`t_out, seed: ${t_out_seed} vs calc: ${t_out}`);
    if (!noLog) logger.info(`t_in_rad, seed: ${
        t_in} vs calc: ${t_in_recalculated}`);
  }

  // **************************************************
  if (!noLog) logger.default(`RADI, T_in_calc: ${params.units.tempC(t_in)},`+
    ` M_fuel: ${
    params.units.mass_flow(m_fuel)}, Tg_out: ${params.units.tempC(tg_out)}`);

  params.t_in_rad = t_in;
  params.t_out    = t_out;
  params.tg_rad   = tg_out;
  params.duty     = duty_total;
  params.m_flue   = m_flue(m_fuel);
  params.m_air    = m_air(m_fuel);
  params.t_w_rad  = Tw( Tb(t_out), Tw(Tb(t_out)) );
  params.q_rad_sh = Q_shld(tg_out, params.t_w_rad);

  const rad_result = {
    m_air:    m_air(),
    m_flue:   m_flue(),
    m_fuel:   m_fuel,
    m_fluid:  m_fluid,
    t_in:     t_in,
    t_out:    t_out,
    Tw:       params.t_w_rad,
    tg_out:   tg_out,

    rfi:      Rfi,

    Q_in:     Q_in(m_fuel),
    Q_rls:    Q_rls(m_fuel),
    Q_air:    Q_air(m_fuel),
    Q_fuel:   Q_fuel(m_fuel),
    
    Q_out:    Q_out(tg_out, m_fuel),
    Q_flue:   Q_flue(tg_out, m_fuel),
    Q_losses: Q_losses(m_fuel),
    Q_shld:   Q_shld(tg_out, params.t_w_rad ),
    Q_conv:   Q_conv(tg_out, params.t_w_rad ),
    Q_rad:    Q_rad( tg_out, params.t_w_rad ),

    Q_R:      Q_R(tg_out, params.t_w_rad),
    Q_fluid:  Q_fluid(t_out,t_in),

    At:       At,
    Ar:       Ar,
    Ai:       Ai,
    Aw:       Aw,
    Aw_aAcp:  Aw_aAcp,
    Acp:      Acp,
    aAcp:     alpha*Acp,
    Acp_sh:   Acp_shld,

    hi:       hi( Tb(t_out), params.t_w_rad ),
    h_conv:   h_conv,

    duty_total: duty_total,
    duty:       duty_rad,
    "%":        duty_rad/duty_total,
    eff_total:  duty_total/Q_rls(m_fuel) > 1 ? 100 : 100*duty_total/Q_rls(m_fuel),
    eff_thermal:(Q_stack)=>100 *(Q_in(m_fuel) - Q_losses(m_fuel) - Q_stack) /
      Q_in(m_fuel),
    eff_gcv:    (Q_stack)=>100 *(Q_in(m_fuel) - Q_losses(m_fuel) - Q_stack) /
      Q_in_gcv(m_fuel),
    duty_flux:  duty_rad/At,

    Alpha:    alpha,
    MBL:      round(MBL),
    Pco2:     round(params.Pco2),
    Ph2o:     round(params.Ph2o),
    PL:       round(PL),
    F:        round(F(tg_out)),
    emiss:    round(emissivity(PL)(tg_out)),

    kw_tube:  kw_tube(Tw(Tb(t_in))),
    kw_fluid: kw_fluid(Tb(t_in)),
    kw_flue:  params.kw_flue(tg_out),

    Cp_fluid: Cp_fluid(t_in,t_out),
    Cp_flue:  Cp_flue(tg_out,t_amb),
    Cp_fuel:  Cp_fuel,
    Cp_air:   Cp_air,

    Prandtl:  round(prandtl(Tb(t_out))),
    Reynolds: round(reynolds(Tb(t_out))),

    TUBING: {
      Material: params.Material,
      Nt:       2,
      N:        N,
      Sch:      params.Sch_rad,
      Do:       Do,
      L:        L,
      S_tube:   S_tube
    },
    
    FINING: "None"
  }
  rad_result.miu_flue = params.miu_flue(tg_out);
  rad_result.miu_fluid = miu_fluid(Tb(t_out));
  
  return rad_result
}

/** returns emissivity(temp) function of temperature to calculate F */
const emissivity = (pl) => {
  // constants to calculate emissivity(temp) from PL
  const constants = {
    a: { A:  2.58e-08, B: -3.90e-08, C:  6.80e-09, D: -2.20e-10},
    b: { A: -1.19e-04, B:  5.60e-05, C: -4.10e-06, D: -7.20e-07},
    c: { A:  0.212580, B:  0.225800, C: -0.047351, D:  0.004165}
  };

  const factors = (factor, constant = constants) => {
    return ((temp) => 
    constant.a[factor]*temp**2 + 
    constant.b[factor]*temp + constant.c[factor]);
  };

  const 
    A = factors("A"),
    B = factors("B"),
    C = factors("C"),
    D = factors("D");

  return (temp) => A(temp) + B(temp)*pl + C(temp)*pl**2 + D(temp)*pl**3;
};

/** (m2) parameters must be in ft */
const Ar_calc = (prams) => {
  const
    ExitArea = unitConv.m2toft2(prams.Pitch_sh_cnv*prams.Tpr_sh_cnv*prams.L_shld),
    Base       = prams.Length_rad * prams.Width_rad,
    WallWidth  = prams.Height_rad * prams.Width_rad,
    WallLength = prams.Height_rad * prams.Length_rad,
    WidthConv   = unitConv.mtoft(prams.Pitch_sh_cnv * prams.Tpr_sh_cnv),
    RoofDeclined_X = (prams.Width_rad - WidthConv)/2,
    RoofDeclined_Z = unitConv.mtoft( 4 * prams.Pitch_rad ),
    RoofDeclined_Y = Math.sin( Math.acos(RoofDeclined_X/RoofDeclined_Z) ) *
      RoofDeclined_Z,
    RoofDeclined   = 2*RoofDeclined_X*RoofDeclined_Y + 2*WidthConv*RoofDeclined_Y,
    Burners = 13 * (Math.PI/4)*2.24**2;
    
  const Ar  = 2*WallWidth + 2*WallLength + 
    2*Base - ExitArea - RoofDeclined - Burners;
  return unitConv.ft2tom2(Ar);
};

/** returns {Aw (m2), Aw_aAcp (-)} */
const Aw_calc = (alpha, Acp, a_shld, Acp_shld, Ar) => {
  const Total_Acp = a_shld*Acp_shld + alpha*Acp; // Equivalent cp area
  const Aw = Ar - Total_Acp; // Effective refractory area
  const Aw_aAcp = Aw / Total_Acp;
  return {Aw, Aw_aAcp};
}
/** returns effectivity(temp) function of temperature to use as F */
const effectivity = (pl, alpha, Acp, a_shld, Acp_shld, Ar) => {
  const {Aw_aAcp} = Aw_calc(alpha, Acp, a_shld, Acp_shld, Ar);
  const emiss = emissivity(pl);
  // logger.warn(`{"Aw (ft)": ${unitConv.m2toft2(Aw)},"Aw_aAcp (-)": ${Aw_aAcp}}`);

  // constants to calculate effectivity(temp) from Aw/aAcp
  const constants = {
    a: { A: -0.0005, B:  0.0072, C: -0.0062 },
    b: { A:  0.0022, B: -0.1195, C:  0.1168 },
    c: { A:  0.0713, B:  0.5333, C: -0.6473 },
    d: { A: -0.0152, B:  1.0577, C: -0.1540 }
  };

  const factors = (factor, constant = constants) => {
    return ((Aw_a_Acp) => 
    constant.a[factor]*Aw_a_Acp**3 + 
    constant.b[factor]*Aw_a_Acp**2 + 
    constant.c[factor]*Aw_a_Acp + constant.d[factor]);
  };

  const 
    A = factors("A"),
    B = factors("B"),
    C = factors("C");
  // log logger.debug(`"Aw/aAcp", "value": "${round(Aw_aAcp)}",`+
  // ` "A": "${round(A(Aw_aAcp))}", "B": "${round(B(Aw_aAcp))}",`+
  // ` "C": "${round(C(Aw_aAcp))}"`);

  return (temp) => A(Aw_aAcp) + B(Aw_aAcp)*emiss(temp) + C(Aw_aAcp)*emiss(temp)**2;
};

module.exports = {
  radSection
};
\end{verbatim}

\subsection{Zona Escudo}
\begin{verbatim}
const {newtonRaphson, logger, LMTD, round} = require('../utils');

const shieldSection = (params, noLog) => {
  let // Temperatures declaration
    tg_in = params.tg_rad,  // (K) Inlet flue gases temp.
    tg_out = 0,             // (K) Outlet flue gases temp.
    t_out = params.t_in_rad,// (K) Outlet fluid temp going to rad sect.

    // --- First estimation: duty equal to the one coming from convect sect
    t_in = (params.t_in_rad + params.t_in_conv)*0.5, // (K) Inlet fluid temp.
    t_in_calc = 0;  // (K) Recalculation for Inlet shld fluid temp.

  /** (K) bulk temp func (second arg default to shld outlet fluid temp) */
  const Tb = (tIn, tOut = t_out) => 0.5*(tIn + tOut);

  const // Process Variables
    m_fluid  = params.m_fluid, // (kg/h) Fluid mass flow.
    m_flue   = params.m_flue,  // (kg/h) Flue mass flow.
    Cp_fluid = (tIn,tOut=tIn) => params.Cp_fluid(Tb(tIn, tOut)), // (kJ/kg.K).
    Cp_flue  = (tG,tG_out=tG) => params.Cp_flue(Tb(tG, tG_out)), // (kJ/kg.K).
    kw_fluid = (temp) => params.kw_fluid(temp),// (kJ/h-m-C) fluid thermal cndct.
    kw_tube  = (temp) => params.kw_tube(temp),
    // (kJ/h-m-C ->J/s-m-C-3.6) tube thermal cndct.
    kw_flue  = (temp) => params.kw_flue(temp),// (kJ/h-m-C) flue thermal cndct.
    miu_fluid= (temp) => params.miu_fluid(temp),//(cP - g/m-s) fluid Viscosity.
    miu_flue = (temp) => params.miu_flue(temp); //(cP - g/m-s) flue Viscosity.

  const // Parameters
    Rfo = params.Rfo_shld,// (h-m2-C/kJ) external fouling factor.
    Rfi = params.Rfi_shld,// (h-m2-C/kJ) internal fouling factor.
    N  = params.N_shld, // - number of shld tubes.
    L  = params.L_shld, // (m) effective tube length.
    Do = params.Do_shld,// (m) external diameter shld section.
    Di = params.Do_shld - params.Sch_sh_cnv*2,// (m) int diameter shld sect.
    S_tube = params.Pitch_sh_cnv, // (m) Tube spacing.

    At = N *Math.PI *Do *L,   
    // (m2) Area of tubes in bank, total outside surface area, m2
    Ai = Math.PI *(Di**2) /2, // (m2) Inside tube surface area, m2
    An = N/2 *(S_tube -Do)*L, // Free area for flue flow at shld sect
    
    cnv_fact = 3_600 * 1e-3; // (g/s -> kg/h) secondsToHours * 1/k.

  const // Process Functions
    prandtl = (t) => miu_fluid(t)*cnv_fact *
      Cp_fluid(t)/kw_fluid(t),// (-) miu*Cp/kw.
    prandtl_flue = (t)=> miu_flue(t)*cnv_fact*
      Cp_flue(t)/kw_flue(t),  // (-) miu_fle*Cp_flue/kw_flue.
    G = (m_fluid/cnv_fact) /Ai, // Fluid mass flow per area unit inside tubes.
    reynolds = (t) => G * Di/miu_fluid(t), // (-) G*Di/miu.
    Gn = (m_flue/cnv_fact) /An, 
    // Gn it's the mass speed based on the free area for the gas flow.
    reynolds_flue = (t) => Gn * Do/miu_flue(t), // (-) Gn*Do/miu_flue.
      
    /** (kJ/m²h-°C) internal heat transfer coff */
    hi = (tB, tW = tB) => .023 *(kw_fluid(tB) /Di) *reynolds(tB)**.8 *
      prandtl(tB)**(1/3)*(miu_fluid(tB)/miu_fluid(tW))**.14,
    /** (kJ/m²h-°C) effective radiative coff wall tube */
    hr = (tG_b) => .092 * tG_b - 34,
    /** (kJ/m²h-°C) * film heat transfer coff */
    hc = (tG_b) => .33 *(kw_flue(tG_b) /Do) *
      prandtl_flue(tG_b)**(1/3) *reynolds_flue(tG_b)**.6,
    /** (kJ/m²h-°C) external heat transfer coff */
    ho = (tG_out, tG_in = tg_in) => 1/(1/(hc(Tb(tG_out, tG_in)) + 
      hr(Tb(tG_out, tG_in))) +Rfo);
  
  const
    duty_sh = (tIn) => m_fluid *Cp_fluid(tIn) *(t_out -tIn),
    /** Average tube wall temp (K) */
    Tw = (tB, tW = tB, tIn = t_in) => (duty_sh(tIn)/At) *(Do/Di)*
      (Rfi +1/hi(tB,tW) +(Di *Math.log(Do/Di) /(2*kw_tube(tW))) ) +tB;

  const // Thermal Resistances (hr-ft2-F/Btu)
    R_int = (tB, tW) => Do / (Di * hi(tB,tW)) + (Do/Di)*Rfi, // Inside
    R_tube= (tW) => Do * Math.log(Do/Di) / (2*kw_tube(tW)),  // Tube wall
    R_ext = (tG_out, tG_in = tg_in) => 1/ho(tG_out, tG_in),  // Outside
    
    R_sum = (tG_out, tG_in, tB, tW) => 
      R_ext(tG_out, tG_in) + R_tube(tW) + R_int(tB,tW),
    Uo  = (tG_out, tG_in, tB, tW) => 1 / R_sum(tG_out, tG_in, tB, tW);
  
  const Q_rad = params.q_rad_sh; 
    // (kJ/h) Q_rad = sigma*(alpha*Acp)*F*(Tg**4 - Tw**4)
  /** Q_conv = Uo . Ao . LMTD */
  const Q_conv = (tIn, tG_in, tG_out, tB, tW) => 
    Uo(tG_out, tG_in, tB, tW)*At*LMTD(tIn, t_out, tG_in, tG_out)
  /** Q_R = Q_conv + Q_rad */
  const Q_R = (tIn, tG_in, tG_out, tB, tW) => Q_rad + 
    Q_conv(tIn, tG_in, tG_out, tB, tW);
  /** Q_fluid =  M . Cp . deltaT */
  const Q_fluid = (tIn, tOut = t_out) => m_fluid * 
    Cp_fluid(tIn, tOut) * (tOut - tIn);
  /** Q_flue =  M . Cp . deltaT */
  const Q_flue = (tG_in,tG_out=tg_out) => m_flue * 
    Cp_flue(tG_in, tG_out) * (tG_in - tG_out);
  
  const tg_out_func = (tG_out) => Q_flue(tg_in, tG_out) + 
    Q_rad - Q_fluid(t_in, t_out);
  const Tin_sh_func = (tIn) => Q_fluid(tIn) - 
    Q_R(tIn, tg_in, tg_out, Tb(tIn), Tw( Tb(tIn),Tw(Tb(tIn)) ));
  // -------- 1st estimation of tg_out   #.#.#.#.#
  tg_out = newtonRaphson(tg_out_func, 
    (tg_in - 100), params.NROptions, "Tg_out_shield-1",noLog);
  t_in_calc = newtonRaphson(Tin_sh_func, 
    t_in, params.NROptions, "T_in_shield-1",noLog);

  let iter = 1;
  const 
    normalized_error = 1e-3, // 0.1%
    normalized_diff = (tG_out) => Math.abs((Q_flue(tg_in, tG_out) -
      Q_conv(t_in,tg_in,tG_out,Tb(t_in),Tw(Tb(t_in),Tw(Tb(t_in)))) )/ 
        Q_flue(tg_in, tg_out));
  while (normalized_diff(tg_out) - normalized_error > 0) {
    if (t_in_calc) { t_in = t_in_calc; } else {
      logger.error("Invalid t_in_calc at shield sect");
      break;
    }
    
    t_in_calc = newtonRaphson(Tin_sh_func, 
      t_in, params.NROptions, "T_in_shield-2",true);
    tg_out = newtonRaphson(tg_out_func, 
      (tg_in - 58), params.NROptions, "Tg_out_shield-2",true);

    // Forced break of loop
    iter++;
    if (iter > 35) {
      logger.debug(`"Tin_shield",  "t_in_sh_calc": ${
        round(t_in_calc)}, "t_in_sh_sup": ${round(t_in)}`);
      if (!noLog) logger.info(
        `diff vs error: ${normalized_diff(tg_out)}-${normalized_error}`);
      logger.error("Max iterations reached for inlet temp calc at shield sect");
      break;
    }
  }
  
  if (!noLog) logger.default(
    `SHLD, cycles: ${iter}, T_in_calc: ${params.units.tempC(t_in)}, `+
    `Tg_out: ${params.units.tempC(tg_out)}`)

  params.t_in_sh = t_in;
  params.tg_sh = tg_out;

  return {
    m_flue:   m_flue,
    t_in_sup: (params.t_in_rad + params.t_in_conv)*0.5,
    t_in:     t_in,
    t_out:    t_out,
    Tb:       Tb(t_in),
    Tw:       Tw( Tb(t_in), Tw(Tb(t_in)) ),
    tg_out:   tg_out,
    tg_in:    tg_in,
    Tb_g:     Tb(tg_in, tg_out),
    LMTD:     LMTD(t_in, t_out, tg_in, tg_out),
    DeltaA:   (tg_in - t_out),
    DeltaB:   (tg_out - t_in),

    rfi:          Rfi,
    rfo:          Rfo,

    Q_flue:   Q_flue(tg_in, tg_out),
    Q_fluid:  Q_fluid(t_in),
    Q_R:      Q_R( t_in, tg_in, tg_out, Tb(t_in), Tw(Tb(t_in)) ),
    Q_rad:    Q_rad,
    Q_conv:   Q_conv( t_in, tg_in, tg_out, Tb(t_in), Tw(Tb(t_in)) ),

    Cp_fluid:   Cp_fluid(t_in,t_out),
    Cp_flue:    Cp_flue(tg_in,tg_out),
    miu_fluid:  miu_fluid(Tw(Tb(t_in))),
    miu_flue:   miu_flue(tg_out),

    duty:       duty_sh(t_in),
    "%":        duty_sh(t_in)/params.duty,
    duty_flux:  duty_sh(t_in)/At,

    kw_fluid:   kw_fluid(Tb(t_in)),
    kw_tube:    kw_tube(Tw(Tb(t_in))),
    kw_flue:    kw_flue(Tb(tg_in, tg_out)),

    Prandtl:    round(prandtl(Tb(t_out))),
    Reynolds:   round(reynolds(Tb(t_out))),
    PrandtlFlue: round(prandtl_flue(Tb(t_out))),
    ReynoldsFlue:round(reynolds_flue(Tb(t_out))),

    At:       At,
    Ai:       Ai,
    An:       An,
    Gn:       Gn,

    hi:         hi( Tb(t_in) ),
    hi_tw:      hi( Tb(t_in), Tw(Tb(t_in)) ),
    hr:         hr(tg_in),
    ho:         ho(tg_out),
    hc:         hc( Tb(tg_in, tg_out) ),

    Uo:         Uo(tg_out, tg_in, Tb(t_in), Tw( Tb(t_in),Tw(Tb(t_in)) ) ),
    R_int:      R_int( Tb(t_in), Tw( Tb(t_in),Tw(Tb(t_in)) ) ),
    R_tube:     R_tube( Tw( Tb(t_in),Tw(Tb(t_in)) ) ),
    R_ext:      R_ext(tg_out, tg_in),

    TUBING: {
      Material: params.Material,
      Nt:       params.Tpr_sh_cnv,
      N:        N,
      Sch:      params.Sch_sh_cnv,
      Do:       Do,
      L:        L,
      S_tube:   S_tube
    },
    
    FINING: "None"
  };
}

module.exports = {
  shieldSection
};
\end{verbatim}

\subsection{Zona Convectiva}
\begin{verbatim}
const {logger, LMTD, round} = require('../utils');

const convSection = (params, noLog) => {
  let
    tg_in  = params.tg_sh,  // (K) Inlet gas temp coming from shld sect
    tg_out = 0,             // (K) Outlet gas temp
    t_out  = params.t_in_sh,// (K) Outlet fluid temp going to shld sect

    // --- First estimation: t_in equal to the one given
    t_in   = params.t_in_conv, // (K) Inlet process fluid temp
    t_in_calc = 0;  // (K) Recalculation for Inlet process fluid temp

  /** (K) bulk temp func (second arg default to conv outlet fluid temp) */
  const Tb = (tIn, tOut = t_out) => 0.5*(tIn + tOut);

  const // Process Variables
    m_fluid  = params.m_fluid, // (kg/h) Fluid mass flow
    m_flue   = params.m_flue,  // (kg/h) Flue mass flow
    Cp_fluid = (tIn,tOut=tIn) => params.Cp_fluid(Tb(tIn, tOut)), // (kJ/kg.K)
    Cp_flue  = (tG,tG_out=tG) => params.Cp_flue(Tb(tG, tG_out)), // (kJ/kg.K)
    kw_fluid = (temp) => params.kw_fluid(temp),
    //(kJ/h-m-C - J/s-m-C-3.6) fluid thermal conductivity
    kw_tube  = (temp) => params.kw_tube(temp),
    // (kJ/h-m-C ->J/s-m-C-3.6) tube thermal cndct
    kw_flue  = (temp) => params.kw_flue(temp),// (kJ/h-m-C) flue thermal cndct
    miu_fluid= (temp) =>params.miu_fluid(temp),//(cP - g/m-s) fluid Viscosity
    miu_flue = (temp) => params.miu_flue(temp);//(cP - g/m-s) flue Viscosity

  const // Parameters
    Rfo = params.Rfo, // (h-m2-C/kJ) external fouling factor
    Rfi = params.Rfi_conv,// (h-m2-C/kJ) internal fouling factor
    L  = params.L_conv, // (m) effective tube length
    Do = params.Do_conv,// (m) external diameter conv section
    Di = params.Do_conv - params.Sch_sh_cnv*2,// (m) int diameter conv sect
    S_tube = params.Pitch_sh_cnv, // (m) Tube spacing, 2*NPS
    N  = params.N_conv,        // (-) number of tubes convective sect
    N_tpr = params.Tpr_sh_cnv, // (-) Number of tube wide
    N_f = params.Nf,           // (1/m) Fin's number per meter
    L_fin = params.Lf,         // (m) Fin's height
    Th_fin = params.Tf,        // (m) Fin's thickness
    Ad = N_tpr*S_tube*L,       // Cross sectional area of box
    Ac = Do+2*L_fin*Th_fin*N_f,// Fin tube cross sectional area/ft, ft2/ft
    An  = Ad - Ac*L*N_tpr,     // Free area for flue flow
    Apo = Math.PI*Do*(1-N_f*Th_fin), // (m2) Outside prime tube surface area, m2/m
    Ao  = Math.PI*Do*(1-N_f*Th_fin) +
      Math.PI*N_f*(2*L_fin*(Do +L_fin) +Th_fin*(Do +2*L_fin)),
    Afo = Ao - Apo,            // (m2) Fin outside surface area, m2/m
    At  = N * Ao * L,          // (m2) Total outside surface area, m2/m
    Ai  = Math.PI *(Di**2) /2, // (m2) Inside tube surface area, m2/m

    /** (ft) Mean Beam Length, dim ratio 1-2-1 to 1-2-4*/
    MBL = 2/3 * (params.Width_rad*params.Length_rad*params.Height_rad) **(1/3),
    PL = (params.Ph2o + params.Pco2) * MBL, // PP*MBL

    cnv_fact = 3_600 * 1e-3; // (g/s -> kg/h) secondsToHours * 1/k

  const // Process Functions
    prandtl = (t) => miu_fluid(t)*cnv_fact *Cp_fluid(t)/kw_fluid(t),// (-) miu*Cp/kw
    prandtl_flue = (t)=> miu_flue(t)*cnv_fact*Cp_flue(t)/kw_flue(t),// (-)
    G = (m_fluid/cnv_fact) /Ai, // Fluid mass flow per area unit inside tubes
    reynolds = (t) => G * Di/miu_fluid(t), // (-) G*Di/miu
    Gn = m_flue/An, // mass speed based on the free area for the gas flow
    reynolds_flue = (t) => Gn/cnv_fact *Do/miu_flue(t), // (-) G*Di/miu
    /** (kJ/m²h-°C) internal heat transfer coff */
    hi = (tB, tW = tB) => .023 *(kw_fluid(tB) /Di) *reynolds(tB)**.8 *
      prandtl(tB)**(1/3) *(miu_fluid(tB)/miu_fluid(tW))**.14;

  /** Q_fluid = M *Cp *deltaT */
  const Q_fluid = (tIn, tOut =t_out) => m_fluid *Cp_fluid(tIn, tOut) *(tOut -tIn);
  const duty_conv = (tIn) => Q_fluid(tIn);  
  // Duty in convective sect used for Tw calc

  /** Tw = Average tube wall temperature in Kelvin degrees */
  const Tw = (tB = Tb(t_out, t_in) , tW = tB, tIn = t_in) => (duty_conv(tIn) /At) *
    (Do/Di) *( Rfi +1/hi(tB,tW) +( Di *Math.log(Do/Di) /(2*kw_tube(tW)) ) ) +tB;

  const
    gr = (_tB, _tW) => 2.6*(0.29307107*cnv_fact), 
    // (Btu/hr-ft2-F) Outside radiation factor //HACK: Find implementation
    hr = (tG_b, tW) => 2.2 *gr(tG_b, tW) *(PL)**.50 *(Apo/Ao)**.75; 
    // (kJ/m²h-°C) effective radiative coff wall tube

  let hc = (tG_b, _tW) => .33 *(kw_flue(tG_b)/Do) *
    prandtl_flue(tG_b)**(1/3) *reynolds_flue(tG_b)**.6; // (kJ/m²h-°C)

  const
    ho = (tG_b, tW) => 1/( 1/(hc(tG_b, tW) +hr(tG_b,tW)) +Rfo ), 
    // (kJ/m²h-°C) external heat transfer coff
    /** Fin's Efficiency */
    Kw_fin = 1.36* kw_tube(Tw(Tb(t_in,t_out), Tw(Tb(t_in,t_out)))),
    B = L_fin + (Th_fin /2),
    m = (ho(Tb(tg_in,tg_out), Tw(Tb(t_in,t_out), Tw(Tb(t_in,t_out)))) / 
      (6 * Kw_fin * Th_fin))**0.5,
    x = Math.tanh(m * B) / (m * B),
    y = x * (0.7 + 0.3 * x),
    Df = Do + 2*L_fin,
    Ef  = y * (0.45 * Math.log(Df / Do) * (y - 1) + 1),    // (-) Fin efficiency
    he = (tG_b, tW) => ho(tG_b, tW) *(Ef*Afo + Apo) / Ao,  // (kJ/m²h-°C)
    j = (tG_b, tW) => colburnFactor(reynolds_flue, params, m, B)(tG_b, tW);  
    // Colburn factor

  hc = (tG_b, tW) => j(tG_b, tW) *Gn *Cp_flue(tG_b) *prandtl_flue(tG_b)**(-.67); 
  // (kJ/m²h-°C) film heat transfer coff

  /** LMTD counter-current */
  const LMTD_Tin = (tIn) => LMTD(tIn, t_out, tg_in, tg_out);

  const // Thermal Resistances (hr-ft2-F/Btu)
    R_int = (tB, tW) => Do/Di * (1/hi(tB, tW) + Rfi),         // Inside
    R_tube =  (tW)  => Do * Math.log(Do/Di) / (2*kw_tube(tW)),// Tube wall
    R_ext = (tG_b, tW) => 1/he(tG_b, tW),                     // Outside

    R_sum = (tG_b, tB, tW) => R_ext(tG_b, tW) + R_tube(tW) + R_int(tB,tW),
    Uo  = (tG_b, tB, tW) => 1 / R_sum(tG_b, tB, tW);


  /** Q_flue  = M *Cp *deltaT */
  const Q_flue = (tG_in, tG_out) => m_flue*Cp_flue(tG_in,tG_out) *(tG_in -tG_out);

  /** Q_conv = Uo * Ao *LMTD */
  const Q_conv = (tIn, tG_in, tG_out) =>
    Uo( Tb(tG_out, tG_in), Tb(tIn), Tw(Tb(tIn),Tw(Tb(tIn))) ) *At *LMTD_Tin(tIn);

  const tg_out_func = (tIn, tG_out = tg_in*0.7) => tg_in - Q_fluid(tIn) /
    (m_flue * Cp_flue(Tb(tg_in,tG_out)));
  const err_diff = () => 100 *( Q_conv(t_in_calc, tg_in, tg_out) -
    Q_fluid(t_in_calc) ) /Q_fluid(t_in_calc);
  const err_tol_pass = () => Math.abs(err_diff()) < 0.001
  
  // -------- 1st estimation of tg_out   #.#.#.#.#
  tg_out = tg_out_func(t_in);
  t_in_calc = t_in;

  while ((tg_out - t_in_calc) < 0) {
    t_in_calc *= 1.002;
    tg_out = tg_out_func(t_in_calc);
  }

  let min, max;

  for (let iter = 0; iter < 100; iter++) {
    if (err_tol_pass()) {
      break
    }
    if ((tg_out - t_in_calc) < 0 || err_diff() <= 0) {
      min = t_in_calc;
      if (min && max) {
        t_in_calc = (min + max)/2
      } else {
        t_in_calc *= 1.001
      }
    } else {
      max = t_in_calc;
      if (min && max) {
        t_in_calc = (min + max)/2
      } else {
        t_in_calc *= 0.999
      }
    }
    tg_out = tg_out_func(t_in_calc);
  }
  
  t_in = t_in_calc;

  if (!noLog) logger.default(`CONV, T_in_calc: ${params.units.tempC(t_in_calc)}, ` +
    `T_in_given: ${params.units.tempC(params.t_in_conv)}, ` +
    `Tg_stack: ${params.units.tempC(tg_out)}`);

  params.t_in_conv_calc = t_in;
  params.tg_conv = tg_out;

  return {
    t_fin:        params.Ts( Tb(t_in), Tw( Tb(t_in), Tw(Tb(t_in)) )),
    t_fin_max:    t_out + Q_conv( t_in, tg_in, tg_out)/
      (At/(params.N_conv/ params.Tpr_sh_cnv)) *(Do/Di) 
      *( Rfi +1/hi(Tb(t_in),Tw( Tb(t_in), Tw(Tb(t_in)) )) +
      ( Di *Math.log(Do/Di) /(2*kw_tube(Tw( Tb(t_in), Tw(Tb(t_in)) ))) ) ),
    t_in_given:   params.t_in_conv,
    t_in:         t_in,
    t_out:        t_out,
    Tb:           Tb(t_in),
    Tw:           Tw( Tb(t_in), Tw(Tb(t_in)) ),
    tg_out:       tg_out,
    tg_in:        tg_in,
    Tb_g:         Tb(tg_in, tg_out),

    rfi:          Rfi,
    rfo:          Rfo,

    LMTD:         LMTD_Tin(t_in),
    DeltaA:       (tg_in - t_out),
    DeltaB:       (tg_out - t_in),

    Q_flue:       Q_flue( tg_in, tg_out),
    Q_fluid:      Q_fluid(t_in),
    Q_conv:       Q_conv( t_in, tg_in, tg_out),
    Q_stack:      Q_flue( tg_out, params.t_air),

    duty:         Q_fluid(t_in),
    "%":          Q_fluid(t_in)/params.duty,
    duty_flux:    Q_fluid(t_in)/At,

    Cp_fluid:     Cp_fluid( t_in, t_out   ),
    Cp_flue:      Cp_flue(  tg_in, tg_out ),
    miu_fluid:    miu_fluid(Tw(Tb(t_in))  ) ,
    miu_flue:     miu_flue( tg_out        ),
    kw_fluid:     kw_fluid( Tb(t_in)      ),
    kw_tube:      kw_tube(  Tw(Tb(t_in))  ),
    kw_fin:       Kw_fin,
    kw_flue:      kw_flue(  Tb(tg_in, tg_out)),

    Prandtl:      round(prandtl(Tb(t_out))),
    Reynolds:     round(reynolds(Tb(t_out))),
    PrandtlFlue:  round(prandtl_flue(Tb(t_out))),
    ReynoldsFlue: round(reynolds_flue(Tb(t_out))),

    At:     At,
    Ai:     Ai,
    An:     An,
    Ao :    Ao,
    Apo:    Apo,
    Afo:    Afo,
    Ef :    Ef,
    Gn:     Gn/cnv_fact,

    hi:     hi( Tb(t_in),          Tw(Tb(t_in), Tw(Tb(t_in))) ),
    hr:     hr( Tb(tg_in, tg_out), Tw(Tb(t_in), Tw(Tb(t_in))) ),
    ho:     ho( Tb(tg_in, tg_out), Tw(Tb(t_in), Tw(Tb(t_in))) ),
    hc:     hc( Tb(tg_in, tg_out), Tw(Tb(t_in), Tw(Tb(t_in))) ),
    he:     he( Tb(tg_in, tg_out), Tw(Tb(t_in), Tw(Tb(t_in))) ),
    j:      j(  Tb(tg_in, tg_out), Tw(Tb(t_in), Tw(Tb(t_in))) ),
    gr:     gr( Tb(tg_in, tg_out), Tw(Tb(t_in), Tw(Tb(t_in))) ),

    Uo:     Uo( Tb(tg_in, tg_out), Tb(t_in), Tw(Tb(t_in)) ),
    R_int:  R_int(                 Tb(t_in), Tw(Tb(t_in))),
    R_tube: R_tube(                          Tw(Tb(t_in))),
    R_ext:  R_ext(Tb(tg_in, tg_out),         Tw(Tb(t_in))),

    TUBING: {
      Material: params.Material,
      Nt:       N_tpr,
      N:        N,
      Sch:      params.Sch_sh_cnv,
      Do:       Do,
      L:        L,
      S_tube:   S_tube
    },
    FINING: {
      Material:    params.FinMaterial,
      Type:        params.FinType,
      Height:      params.Lf,
      Thickness:   params.Tf,
      Dens:        params.Nf,
      Arrange:     params.FinArrange
    }
  };
}

const colburnFactor = (reynoldsFlue, parm, m, B) => {
  const
    C1 = (tB_g) => .25 *reynoldsFlue(tB_g)**(-.35), // Reynolds number correction

    Lf = parm.Lf,                         // (m) Fins height
    Sf = 1/parm.Nf - parm.Tf,             // (m) Fin spacing
    C3 = .35 +.65 * Math.exp(-.25*Lf/Sf), 
    // Geometry correction (Solid, staggered pattern)

    Pl = parm.Pitch_sh_cnv,            // (m) Longitudinal tube pitch
    Pt = parm.Pitch_sh_cnv,            // (m) Transverse tube pitch
    Nr = parm.N_conv/ parm.Tpr_sh_cnv, // (-) Tube row's number
    C5 = .7 + (.7 -.8 *Math.exp(-.15*Nr**2)) *Math.exp(-Pl/Pt), 
    // Non-equilateral & row correction

    Df_Do = (2*parm.Lf + parm.Do_conv) / (parm.Do_conv), 
    // (m) Ratio fin's Do per tube's Do

    Ts = (tB_g, tW) => tB_g + (tW - tB_g) / ( ( Math.exp(1.4142*m*B) + 
      Math.exp(-1.4142*m*B) )/2 );// (K) Average fin temp
  parm.Ts = Ts;
  return (tB_g, tW) => C1(tB_g) *C3 *C5 *(Df_Do)**.5 *(tB_g/Ts(tB_g, tW))**.25;
};

module.exports = {
  convSection
};
\end{verbatim}

\section{Funciones compartidas y valores por defecto}
\begin{verbatim}
const logByLevel = (...stringsList) => {
  let finalText = "" + stringsList[1][0]
  for (let i = 1; i < stringsList[1].length; i++) {
    finalText += " " + stringsList[1][i]
  }
  switch (stringsList[0]) {
    case "DEBUG":
      if (options.verbose) 
        console.debug(JSON.parse(`{"${stringsList[0]}": ${finalText}}`));
      break;
    case "INFO": 
      console.info( `%c${stringsList[0]}`,'color: lime;', `'${finalText}'`,);
      break;
    case "ERROR":
      console.error(`%c${stringsList[0]}`,'color: tomato;', `'${finalText}'`,);
      break;
    case "WARN":
      console.warn( `%c${stringsList[0]}`,'color: magenta;', `'${finalText}'`,);
      break;
    default:
      console.log(  `%c${stringsList[0]}`,'color: dodgerblue;', `'${finalText}'`,);
      break;
  }
};
const logger = {
  info:   (...stringsList) => logByLevel("INFO", stringsList),
  warn:   (...stringsList) => logByLevel("WARN", stringsList),
  error:  (...stringsList) => logByLevel("ERROR", stringsList),
  debug:  (...stringsList) => logByLevel("DEBUG", stringsList),
  default:(...stringsList) => logByLevel("DEFAULT", stringsList),
};

/** Receives a function, optional the derivate, a seed and the options object, 
 * finally an identifier name */
const newtonRaphson = (f, fp, x0, nrOptions, name, noLog) => {
  let x1, y, yp, iter, yph, ymh, yp2h, ym2h;

  // Interpret variadic forms:
  if (typeof fp !== 'function') {
    noLog = name;
    name = nrOptions;
    nrOptions = x0;
    x0 = fp;
    fp = null;
  }

  const 
    opts = nrOptions || {},
    tol = opts.tolerance === undefined ? 1e-7 : opts.tolerance,
    eps = opts.epsilon === undefined ? 2.22e-15 : opts.epsilon,
    h = opts.h === undefined ? 1e-4 : opts.h,
    hr = 1 / h,
    maxIter = opts.maxIterations === undefined ? 20 : opts.maxIterations;

  iter = 0;
  while (iter++ < maxIter) {
    // Compute the value of the function:
    y = f(x0);

    if (fp) {
      yp = fp(x0);
    } else {
      // Needs numerical derivatives:
      yph = f(x0 + h);
      ymh = f(x0 - h);
      yp2h = f(x0 + 2 * h);
      ym2h = f(x0 - 2 * h);

      yp = ((ym2h - yp2h) + 8 * (yph - ymh)) * hr / 12;
    }

    // Check for badly conditioned update 
    // (extremely small first deriv relative to function):
    if (Math.abs(yp) <= eps * Math.abs(y)) {
      logger.error(`Newton-Raphson (${
        name}): failed to converged due to nearly zero first derivative`);
      return false;
    }

    // Update the guess:
    x1 = x0 - y / yp;

    // Check for convergence:
    if (Math.abs(x1 - x0) <= tol * Math.abs(x1)) {
      if (!noLog) logger.debug(`"Newton-Raphson", "func":"${name}",`+
        ` "var_converged_to":"${x1}", "iterations":"${iter}"`);
      return x1;
    }

    // Transfer update to the new guess:
    x0 = x1;
  }
  logger.error(`Newton-Raphson (${
    name}): Maximum iterations reached (${maxIter})`);

  return false;
};

const 
  tempToK = 273.15,
  pAtmRef = 101_325,
  /* ATT: Changes here must be done at webInput too. */
  barrelsToft3 = 5.6145833333,
  ft3Tolb = 62.371, // for Water @60°F
  spGrav = 0.84, // for current fluid
  tempAmbRef = tempToK + 15.55556; // 288.7 K

const unitConv = {
  RtoK: (n=1) => n*(5/9),
  KtoR: (n=1) => n*(9/5),
  KtoF: (n=1) => n*(9/5) - 459.67,
  KtoC: (n=1) => n-tempToK,
  CtoK: (n=1) => n+tempToK,
  CtoF: (n=1) => n*(9/5) + 32,
  FtoC: (n=1) => (n-32)*(5/9),
  FtoK: (n=1) => (n-32)*(5/9)+tempToK,

  kgtolb: (n=1) => n*2.20462,
  lbtokg: (n=1) => n/2.20462,
  m3ToBarrels: (n=1) => n/(0.158987295),
  BarrelsTom3: (n=1) => n*(0.158987295),
  BPDtolb_h:(n=1,spG=spGrav) => n*barrelsToft3*ft3Tolb/24*spG,
  lb_htoBPD:(n=1,spG=spGrav) => n/barrelsToft3/ft3Tolb*24/spG,

  kJtoBTU: (n=1) => n/1.05506,
  BTUtokJ: (n=1) => n*1.05506,

  fttom:  (n=1) => n/3.28084,
  ft2tom2:(n=1) => n/(3.28084**2),
  mtoft:  (n=1) => n*3.28084,
  m2toft2:(n=1) => n*(3.28084**2),
  intom:  (n=1) => n/39.3701,
  mtoin:  (n=1) => n*39.3701,

  CpENtoCpSI: (n=1) => n*1.05506/(5/9)*2.20462,     // (kJ/kg-C)
  kwENtokwSI: (n=1) => n*1.05506/(5/9)*3.28084,     // (kJ/h-m-C)
  RfENtoRfSI: (n=1) => n/(1.05506/(5/9)*3.28084**2),// (h-m2-C/kJ)
  hcENtohcSI: (n=1) => n*1.05506/(5/9)*3.28084**2,  // (kJ/h-m2-C)
  BtuHtoW: (n=1) => n/3.4121416331,
};

/** Example call from terminal: node . false ENGLISH 26.6667 50 0 20 1.01325e5 */ 
const getOptions = () => {
  const optObject = {
    // constants
    tempToK,
    tempAmbRef,
    pAtmRef,
    spGrav,

    // Entry default arguments
    runDistCycle: true,     // boolean
    verbose:    true,       // boolean
    tAmb:       tempAmbRef, // K
    tAir:       tempAmbRef, // K
    tFuel:      tempAmbRef, // K
    humidity:   50.0,       // %
    o2Excess:   .01 * 0,    // fr
    airExcess:  .01 * 20,   // fr
    radDist:    .01 * 64,   // % *.01
    hLoss:      .01 * 1.5,  // % *.01
    effcy:      .01 * 80,   // % *.01
    rfi:        0.000,      // hr.ft².°F/Btu
    rfiConv:    0.000,      // hr.ft².°F/Btu
    rfoConv:    0.000,      // hr.ft².°F/Btu
    rfiShld:    0.000,      // hr.ft².°F/Btu
    rfoShld:    0.000,      // hr.ft².°F/Btu
    tIn:        unitConv.FtoK(678),// K
    tOut:       unitConv.FtoK(772),// K
    mFluid:     90e3,       // BPD
    miuFluidIn: 1.45,       // cp
    miuFluidOut:.960,       // cp
    cpFluidIn:  unitConv.CpENtoCpSI(.676), // kJ/kg-C
    cpFluidOut: unitConv.CpENtoCpSI(.702), // kJ/kg-C 
    kwFluidIn:  .038,       // Btu/h-ft-F
    kwFluidOut: .035,       // Btu/h-ft-F
    pAtm:       pAtmRef,    // Pa
    unitSystem: "SI",       // string SI or EN
    lang:       "en",       // string EN or ES
    title:      "base",     // string
    graphVar:   "t_out",    // string one of four
    graphRange: 50,         // uInt
    graphPoints:100,        // uInt
  
    // Newton Raphson arguments
    NROptions: {
      tolerance: 1e-4,
      epsilon: 3e-8,
      maxIterations: 20,
      h: 1e-4,
      verbose: true
    }
  };
  
  if (typeof process == 'undefined') return optObject;

  optObject.verbose =                process.argv[2] == "true";
  optObject.unitSystem =             process.argv[3];
  optObject.tAmb =tempToK+parseFloat(process.argv[4]) || tempAmbRef;
  optObject.humidity =    parseFloat(process.argv[5]) || 0;
  optObject.o2Excess =.01*parseFloat(process.argv[6]) || .01 * 0;
  optObject.airExcess=.01*parseFloat(process.argv[7]) || .01 * 0;
  optObject.pAtm =        parseFloat(process.argv[8]) || 1.01325e5;
  // Newton Raphson arguments
  optObject.NROptions.verbose =      process.argv[2] == "true";

  return optObject;
};
const options = getOptions();
if (options.verbose) 
  logger.debug(`"options","args":${JSON.stringify(options, null, 2)}`);

const round = (number, dec = 3) => number !== undefined ? 
  (number).toLocaleString(
    undefined,
    {minimumFractionDigits: dec, maximumFractionDigits: dec}
  ) : NaN;
const roundDict = (object = {}) => {
  for (const [key, value] of Object.entries(object)) {
    if(!isNaN(value) && value !== "") object[key] = round(value);
  }
};

/** Normalize an object of fuels/products */
const normalize = (fuels, name, noLog) => {
  const normalFuel = {...fuels};
  const total = Object.values(normalFuel).reduce((acc, value)=> acc + value);
  for (const fuel in normalFuel) {
    normalFuel[fuel] = normalFuel[fuel]/total;
  }
  if (!noLog) 
    logger.debug(`"normalize", "name": "${name}", "total": ${total}`);
  return normalFuel;
};

/** Thermal Cond equation func(temp [K]) for certain substance in data */
const kw = ({k0, k1, k2, Substance}) => {
  // Thermal Cond equation from NIST data with polynomial approx. R2=1
  // k2*T^2 + k1*T + k0  (valid from 300K to 1350K)* SO2 only to 500K
  if (k0 == 0 || k0 == "-") {
    logger.debug(`"Thermal Cond func called for '${Substance}' without coffs"`);
    return () => 0;
  }
  const cnv_fact = 3_600 * 1e-3; // Therm. Cond. (W/m*K) -> (kJ/h*m*K)
  return (temp) => (k0 + k1* temp + k2* temp**2)*cnv_fact;
};

/** returns a Thermal Cond function of temp for certain flue composition.
 * 
 * Should be called in the combustion section after flue composition 
 * is determined.
*/
const flueThermalCond = (data, flue) => {
  const 
    normalFlue = normalize(flue, "flueThermalCond"),
    so2_kw = kw(data[34]),
    h2o_kw = kw(data[31]),
    co2_kw = kw(data[6] ),
    n2_kw  = kw(data[3] ),
    o2_kw  = kw(data[2] );

  return (t) => normalFlue.CO2*co2_kw(t) + normalFlue.SO2*so2_kw(t)
  + normalFlue.H2O*h2o_kw(t) + normalFlue.O2*o2_kw(t) + normalFlue.N2*n2_kw(t);
};

/** Viscosity equation func(temp [K]) for certain substance in data */
const miu = ({u0, u1, u2, Substance}) => {
  // Viscosity equation from NIST data with polynomial approx. R2=0.99998
  // u2*T^2 + u1*T + u0  (valid from 300K to 1350K)* SO2 only to 500K
  if (u0 == 0 || u0 == "-") {
    logger.debug(`"Viscosity func called for '${Substance}' without coffs"`);
    return () => 0;
  }
  return (temp) => u0 + u1* temp + u2* temp**2;
};

/** returns a Viscosity function of temp for certain flue composition.
 * 
 * Should be called in the combustion section after flue composition 
 * is determined.
*/
const flueViscosity = (data, flue) => {
  const 
    normalFlue = normalize(flue, "flueViscosity"),
    so2_v = miu(data[34]),
    h2o_v = miu(data[31]),
    co2_v = miu(data[6] ),
    n2_v  = miu(data[3] ),
    o2_v  = miu(data[2] );

  return (t) => normalFlue.CO2*co2_v(t) + normalFlue.SO2*so2_v(t)
  + normalFlue.H2O*h2o_v(t) + normalFlue.O2*o2_v(t) + normalFlue.N2*n2_v(t);
};

/** returns a Thermal conductivity function of temp for tubes A312‐TP321 
 * temp should be used in Kelvin, value returned in (kJ/h-m-C)
*/
const kw_tubes_A312_TP321 = (t) => {
  const 
    temp = t - tempToK, // transforms temp from Kelvin to Celsius
    conv_factor = 3_600 * 1e-3, // (J/s -> kJ/h)
    c0 = 14.643,
    c1 = 1.64e-2,
    c2 = -2e-6;

  return (c0 + c1*temp + c2*temp**2)*conv_factor;
};

/** Returns a linear function f(x)=y to approximate the value,
 * in case that the value is constant or there isn't data
 * about the changes, it can be called with only "y1"
 * to make a function that always return y1.
 */
const linearApprox = ({x1,x2,y1,y2}) => {
  if (typeof y1 !== "number") {
    logger.error(`call for linearApprox with incorrect value type for y1: ${y1}`)
    return () => 0;
  }
  if (x1 == x2 || x2 == undefined || y2 == undefined) 
    return () => y1;
  const m = (y2 - y1) / (x2 - x1);
  return (x) => m * (x - x1) + y1;
};

const viscosityApprox = ({t1,t2,v1,v2}) => {
  if (typeof v1 !== "number") {
    logger.error(`call for viscosityApprox with incorrect value type for v1: ${v1}`)
    return () => 0;
  }
  if (t1 == t2 || t2 == undefined || v2 == undefined) 
    return () => v1;
  const B = Math.log(v1/v2) / (1/t1 - 1/t2)
  const A = v1 * Math.exp(-B/t1);
  return (temp) => A * Math.exp(B/temp);
}

/** (Tref1, Tref2, T1, T2, co-current) Returns the value of the
 * Logarithmic mean temperature difference.
 * 
 * counter-current by default, for co-current set the five argument as true.
 * */
const LMTD = (t_cold_in, t_cold_out, t_hot_in, t_hot_out, co_current) => {
  
  let // counter-current (default)
    delta_t1 = t_hot_in - t_cold_out,
    delta_t2 = t_hot_out - t_cold_in;
    
  if (co_current) { // co-current
    delta_t1 = t_hot_out - t_cold_in;
    delta_t2 = t_hot_in - t_cold_out;
  }
    
  // ( (t_hot_in - t_cold_out) - (t_hot_out - t_cold_in) ) 
  // / ln( (t_hot_in - t_cold_out) / (t_hot_out-t_cold_in) )
  return Math.abs((delta_t1 - delta_t2) /Math.log(Math.abs(delta_t1/delta_t2)));
};

const dualSystem = (onlyUnit, noUnit, decimal=3 ,units="", number=0) => {
  if (onlyUnit) return ` ${units}`;
  return round(number, decimal) + (noUnit ? "" : ` ${units}`);
}

const englishSystem = { //(US Customary)
  "energy/mol":   (n,d,nU,oU) => dualSystem(oU,nU,d,"Btu/mol", unitConv.kJtoBTU(n)),
  "mass/mol":     (n,d,nU,oU) => dualSystem(oU,nU,d,"lb/lbmol", n),
  heat_flow :     (n,d,nU,oU) => dualSystem(oU,nU,d,"MMBtu/h", 
    unitConv.kJtoBTU(n)*1e-6),
  heat_flux:      (n,d,nU,oU) => dualSystem(oU,nU,d,"Btu/h-ft²",
    unitConv.kJtoBTU(n) /unitConv.m2toft2()),
  fouling_factor: (n,d,nU,oU) => dualSystem(oU,nU,d,"h-ft²-°F/Btu", 
    unitConv.m2toft2(n)*unitConv.KtoR()/unitConv.kJtoBTU()),
  "energy/mass":  (n,d,nU,oU) => dualSystem(oU,nU,d,"Btu/lb", 
    unitConv.kJtoBTU(n) / unitConv.kgtolb()),
  "energy/vol":   (n,d,nU,oU) => dualSystem(oU,nU,d,"Btu/ft³", 
    unitConv.kJtoBTU(n) / unitConv.mtoft()**3),

  area:     (n,d,nU,oU) => dualSystem(oU,nU,d,"ft²", unitConv.m2toft2(n)),
  length:   (n,d,nU,oU) => dualSystem(oU,nU,d,"ft", unitConv.mtoft(n)),
  lengthC:  (n,d,nU,oU) => dualSystem(oU,nU,d,"in", unitConv.mtoin(n)),
  lengthInv:(n,d,nU,oU) => dualSystem(oU,nU,d,"1/ft",n/unitConv.mtoft()),
  temp:     (n,d,nU,oU) => dualSystem(oU,nU,d,"°R", unitConv.KtoR(n)),
  tempC:    (n,d,nU,oU) => dualSystem(oU,nU,d,"°F", unitConv.CtoF(n-tempToK)),
  pressure: (n,d,nU,oU) => dualSystem(oU,nU,d,"psi", n *1.450377e-4),
  mass:     (n,d,nU,oU) => dualSystem(oU,nU,d,"lb", unitConv.kgtolb(n)),
  mass_flow:(n,_d,nU,oU) => dualSystem(oU,nU,0,"lb/h", unitConv.kgtolb(n)),
  barrel_flow:(n,d,nU,oU,spG = spGrav) => dualSystem(oU,nU,d,"x10³ BPD", 
    unitConv.kgtolb(n)/ unitConv.BPDtolb_h(1,spG) /1e3),  
  barrel_flowC:(n,d,nU,oU) => dualSystem(oU,nU,d,"BPD", n),
  vol_flow: (n,d,nU,oU) => dualSystem(oU,nU,d,"ft³/h", unitConv.mtoft(n)**3),
  cp:       (n,d,nU,oU) => dualSystem(oU,nU,d,"Btu/lb-°F", n *.238845896627),
  cp_mol:   (n,d,nU,oU) => dualSystem(oU,nU,d,"Btu/lb-mol-°F", n *.238845896627),
  power:    (n,d,nU,oU) => dualSystem(oU,nU,d,"Btu/h", n *3.4121416331),
  moist:    (n,d,nU,oU) => dualSystem(oU,nU,d,"÷10³ lb H2O/lb", n*1e3),
  thermal:  (n,d,nU,oU) => dualSystem(oU,nU,d,"BTU/h-ft-°F", 
    unitConv.kJtoBTU(n)/unitConv.KtoR()/unitConv.mtoft()),
  convect:  (n,d,nU,oU) => dualSystem(oU,nU,d,"BTU/h-ft²-°F", 
    unitConv.kJtoBTU(n)/unitConv.KtoR()/(unitConv.m2toft2())),
  viscosity:(n,d,nU,oU) => dualSystem(oU,nU,d,"cP", n),
  system:   {en: "English", es: "Inglés"}
};

const siSystem = {
  "energy/mol":   (n,d,nU,oU) => dualSystem(oU,nU,d,"kJ/mol", n),
  "mass/mol":     (n,d,nU,oU) => dualSystem(oU,nU,d,"kg/kmol", n),
  heat_flow:      (n,d,nU,oU) => dualSystem(oU,nU,d,"MW", n*1e-6 /3.6),
  heat_flux:      (n,d,nU,oU) => dualSystem(oU,nU,d,"W/m²", n /3600),
  fouling_factor: (n,d,nU,oU) => dualSystem(oU,nU,d,"m²-K/W ÷10³", n*3.6e3),

  "energy/mass":  (n,d,nU,oU) => dualSystem(oU,nU,d,"kJ/kg", n),
  "energy/vol":   (n,d,nU,oU) => dualSystem(oU,nU,d,"kJ/m³", n),
  area:       (n,d,nU,oU)  => dualSystem(oU,nU,d,"m²", n),
  length:     (n,d,nU,oU)  => dualSystem(oU,nU,d,"m", n),
  lengthC:    (n,d,nU,oU)  => dualSystem(oU,nU,d,"cm", n*1e2),
  lengthInv:  (n,d,nU,oU)  => dualSystem(oU,nU,d,"1/m", n),
  tempC:      (n,_d,nU,oU) => dualSystem(oU,nU,0,"°C", n -tempToK),
  temp:       (n,d,nU,oU)  => dualSystem(oU,nU,d,"K", n),
  pressure:   (n,d,nU,oU)  => dualSystem(oU,nU,d,"kPa", n *1e-3),
  mass:       (n,d,nU,oU)  => dualSystem(oU,nU,d,"kg", n *1e-3),
  mass_flow:  (n,d,nU,oU)  => dualSystem(oU,nU,d,"kg/s", n /3600),
  barrel_flow:(n,d,nU,oU,spG = spGrav) => englishSystem.barrel_flow(n,d,nU,oU,spG),
  barrel_flowC:(n,d,nU,oU) => dualSystem(oU,nU,d,"m³/d", n/unitConv.m3ToBarrels()),
  vol_flow: (n,d,nU,oU) => dualSystem(oU,nU,d,"m³/s", n /3600),
  cp:       (n,d,nU,oU) => dualSystem(oU,nU,d,"kJ/kg-K", n),
  cp_mol:   (n,d,nU,oU) => dualSystem(oU,nU,d,"kJ/kmol-K", n),
  power:    (n,d,nU,oU) => dualSystem(oU,nU,d,"W", n /3.6),
  moist:    (n,d,nU,oU) => dualSystem(oU,nU,d,"g H2O/kg", n *1e3),
  thermal:  (n,d,nU,oU) => dualSystem(oU,nU,d,"kJ/h-m-C", n),
  convect:  (n,d,nU,oU) => dualSystem(oU,nU,d,"kJ/h-m²-C", n),
  viscosity:(n,d,nU,oU) => dualSystem(oU,nU,d,"cP", n),
  system:   {en: "SI", es: "SI"}
};

const initSystem = (unitSystem) => {
  if (typeof unitSystem !== "string") {
    if (options.verbose) logger.warn( 
    `invalid type (${unitSystem}) for unit system, using default SI`);
    return siSystem
  }
  switch (unitSystem.toLowerCase()) {
    case "si":
      return siSystem;
    case "english":
      return englishSystem;
    case "en":
      return englishSystem;
    default:
      logger.warn(unitSystem.toLowerCase() + 
      ' - invalid unit system, using default SI')
      return siSystem;
  }
};

module.exports = {
  options,
  unitConv,
  newtonRaphson,
  logger,
  round,
  roundDict,
  linearApprox,
  viscosityApprox,
  initSystem,
  normalize,
  flueViscosity,
  flueThermalCond,
  kw_tubes_A312_TP321,
  LMTD
};
\end{verbatim}


\end{document}
