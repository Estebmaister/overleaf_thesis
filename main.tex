% Universidad Simón Bolívar
% Plantilla LaTeX para manuscritos (tesis y pasantías)
% pregrado y postgrado
%
% Andrés M. Sajo-Castelli
% Carlos Contreras
%
% 15 Abril 2015 -- primera versión pública
% ...
% 11 Mayo 2018 --- Se agrega bibliografía en castellano via babelbib
%
\documentclass[pregrado]{tesis-usb}

% paquetes
\usepackage[utf8]{inputenc}
\usepackage{verbatim}
\usepackage{acronym}
\usepackage{amsmath}
\usepackage{amsfonts}
\usepackage{amssymb}
% \usepackage{hyperref}

% estilo de las referencias
\usepackage[fixlanguage]{babelbib-and}\selectbiblanguage{spanish}
\usepackage{url}
\bibliographystyle{babplain-lf}

\autor{Esteban de Jesús Rafael Camargo Rodríguez}
\usbid{11-10145}
\titulo{Diseño de simulador de hornos de proceso}
\tutor{Antonio Cavero}
\usarcotutor
\cotutor{Euler Jimenez} 
\trabajo{Proyecto de grado}
\coord{Ingeniería Química}
\grado{Ingeniero Químico}
\fecha{Mayo~de~2022}

\autori{E. Camargo}
\agno{2022}
\fechadefensa{30~de~Mayo~de~2022}
\carrera{Ing. Química}

\programa{Nombre del Programa}
\juradouno{Alejandro Requena}
\juradodos{Sabrina D'Scipio \mbox{(Afiliaci\'on)}}
\juradotres{Claudio Olivares}

% Cambia comillas simple por comilla cerrada en ambiente verbatim 
\makeatletter
\let \@sverbatim \@verbatim
\def \@verbatim {\@sverbatim \verbatimplus}
{\catcode`'=13 \gdef \verbatimplus{\catcode`'=13 \chardef '=13 }} 
\makeatother

\begin{document}

\frontmatter
\maketitle
\chapter*{Dedicatoria}

\par Dedicado a la Universidad Simón Bolívar y a su comunidad.
\par ~
\par A mis padres.

\chapter*{Agradecimientos}

Gracias a mis tutores, Euler y Antonio, por su gu\'ia en este trabajo, respondiendo atentamente a todas mis preguntas, por su paciencia corrigiendo mis errores y por su constante motivación.\\

Gracias a mi madre, Marlene, por apoyarme incondicionalmente y siempre haber creído en mí.
\begin{resumen}
     El conocimiento al operar un equipo en cualquier planta de ingeniería siempre ha sido algo fundamental, pero hay ocaciones en que el conocimiento práctico se queda corto y no basta con saber manipularlo físicamente, si no también conocer los principios fundamentales detrás del funcionamiento del equipo. Los hornos de procesos en refinerías industriales son equipos asociados al corazón del proceso, y por tanto, requieren de fuertes bases para su correcta manipulación. Teniendo esto en cuenta se diseño este proyecto para generar una aplicación capaz de transmitir de manera eficiente los principios y consecuencias de la manipulación del horno de procesos. El objetivo de la herramienta es encargarse de simular las posibles interacciones de un operador con el equipo, con su uso los operadores podrán tener una idea de como cambiará el proceso dependiendo de las variables que decidan alterar, las cuales pueden ser el flujo del crudo, la apertura o cierre de los ductos de alimentación de aire o salida de gases de chimenea y por último la cantidad de combustible suministrado al horno. \\
     
     Palabras claves: Fire heater, Horno, Intercambiador, Simulador.
\end{resumen}
\tableofcontents
\listoffigures
\listoftables
\useacronyms
\chapter*{Lista de símbolos}
\begin{tabular}{ll}

$AC_{masa}$ & Relación aire/combustible másica [-] \\
$AC_{molar}$& Relación aire/combustible molar [-] \\
$Afo$   & Área de superficie externa de aletas [m$^2$] \\
$Apo$   & Área de superficie expuesta de tubos lisos [m$^2$] \\
$A_0$   & Área de superficie externa total [m$^2$] \\
$At$    & Área externa del banco de tubos por zona del horno [m$^2$] \\
$Acp$   & Área de plano equivalente por zona del horno [m$^2$] \\
\\
$C_{1,3,5}$ & Coeficientes del factor de Colburn \\
$C_p$   & Calor específico [kJ/kg-K] \\
$C_{p_a}$   & Calor específico del aire [kJ/kg-K] \\
$C_{p_c}$   & Calor específico del combustible [kJ/kg-K] \\
$C_{p_f}$   & Calor específico del fluido [kJ/kg-K] \\
$C_{p_g}$   & Calor específico de los gases de combustión [kJ/kg-K] \\

$d_f$ & Diámetro externo de la aleta [m] \\
$d_o$ & Diámetro externo del tubo [m] \\

$E$   & Eficiencia de las aletas [-]\\
$F$   & Factor global de transferencia radiante [-]\\
$G$   & Velocidad másica basada en el área interna de tubos radiantes [kg/h-m$^2$]\\
$G_n$ & Velocidad másica basada en el área libre para el flujo de gas [kg/h-m$^2$]\\
\\
$h_{c}$ & Coeficiente de transferencia de calor convectivo [W/m$^2$-K]\\
$h_{e}$ & Coeficiente de transferencia de calor externo [W/m$^2$-K]\\
$h_{ee}$& Coeficiente de transferencia de calor externo efectivo [W/m$^2$-K]\\
$h_{i}$ & Coeficiente de transferencia de calor interno [W/m$^2$-K]\\
$h_{r}$ & Coeficiente de transferencia de calor radiante [W/m$^2$-K]\\
\\
$j$     & Factor de Colburn [-] \\
$k_f$   & Conductividad térmica del fluido [W/m-K] \\
$k_g$   & Conductividad térmica de los gases de combustión [W/m-K] \\
\\
$\dot m_a$  & Flujo másico de aire [kg/h] \\
$\dot m_c$  & Flujo másico de combustible [kg/h] \\
$\dot m_f$  & Flujo másico de residuo de vacío [kg/h] \\
$\dot m_g$  & Flujo másico de los gases de combustión [kg/h] \\
\end{tabular}

\begin{tabular}{ll}
\\
\\
$n_a$  & Número de moles del aire [mol] \\
$n_c$  & Número de moles del combustible [mol] \\
\\
$l_f$   & Altura de aleta [m] \\
$L_{tubo}$ & Largo efectivo del tubo [m] \\
$N_r$   & Numero de tubos por fila [-] \\
$N_{tubo}$ & Número de tubos por sección [-] \\
\\
$PL$    & \multirow{2}{26em}{Multiplicación de las presiones parciales de CO$_2$ y H$_2$O por MBL [atm-ft]} \\
\\
$PM_a$    & Peso molecular del aire [kg/kmol]\\
$PM_c$    & Peso molecular del combustible [kg/kmol]\\
\\
$P_l$   & Paso de tubo longitudinal [-]\\
$P_t$   & Paso de tubo transversal [-]\\
\\
$Pr$    & Número de Prandtl [-]\\
\\
$Q_{a}$    & Calor sensible del aire [MW] \\
$Q_{absorbido}$ & Calor absorbido por el fluido en el horno [MW] \\
$Q_{c}$    & Calor sensible del combustible [MW] \\
$Q_{conv}$ & Calor transferido por convección [MW] \\
$Q_{CONV}$ & Calor cedido al fluido en la sección convectiva [MW] \\
$Q_{ESC}$ & Calor cedido al fluido en la sección escudo [MW] \\
$Q_{f_e}$ & Calor del fluido entrando [MW] \\
$Q_{f_s}$ & Calor del fluido saliendo [MW] \\
$Q_{g}$   & Calor sensible de los gases de combustión [MW] \\
$Q_{perdidas}$  & Pérdidas de calor al ambiente [MW] \\
$Q_{rad}$ & Calor transferido por radiación [MW] \\
$Q_{RAD}$ & Calor cedido al fluido en la sección radiante [MW] \\
$Q_{radEsc}$& Calor de radiación que pasa a la sección escudo [MW] \\
$Q_{suministrado}$ & Calor suministrado por combustión [MW] \\
\\
$q_{rad}$ & \multirow{2}{26em}{Flujo de calor por unidad de área exterior del tubo en zona radiante [MW/m$^2$]} \\
\\
$Re$    & Número de Reynolds [-]\\
\\
$R_{fi}$  & Factor de ensuciamiento interno de los tubos [m$^2$-K/W] \\
$R_{fe}$  & Factor de ensuciamiento externo de los tubos [m$^2$-K/W] \\
\end{tabular}

\begin{tabular}{ll}
\\
\\
$R_{int}$  & Resistencia convectiva interna [m$^2$-K/W]\\
$R_{ext}$  & Resistencia convectiva externa [m$^2$-K/W] \\
$R_{tube}$ & Resistencia conductiva del tubo [m$^2$-K/W]\\
$\Sigma R$ & Suma de las resistencias de transferencia de calor [m$^2$-K/W]\\
\\
$s_f$      & Espaciado de aleta [1/m]\\
$S_{tubo}$ & Espaciado de tubos [m] \\
\\
$T_b$ & Temperatura de mezcla del fluido por zona [K] \\
$T_{fe}$& Temperatura del fluido entrando al horno [K] \\
$T_{fs}$& Temperatura del fluido saliendo del horno [K] \\
$T_{fer}$& Temperatura del fluido entrando a la zona radiante [K] \\
$T_{fee}$& Temperatura del fluido entrando a la zona escudo [K] \\
$T_{gc}$& Temperatura de los gases de combustión a la salida de la zona convectiva [K] \\
$T_{ge}$& Temperatura de los gases de combustión a la salida de la zona escudo [K] \\
$T_{gr}$& Temperatura de los gases de combustión a la salida de la zona radiante [K] \\
$T_g$   & Temperatura efectiva de llama, equivalente a $T_{gr}$ [K] \\
$T_{gb}$& Temperatura de mezcla del gas por zona [K] \\
$T_s$ & Temperatura promedio de aleta [K]\\
\\
$U_0$ & Coeficiente de transferencia de calor global [W/m$^2$-K] \\
\\
$\sigma$    & Constante de Stefan-Boltzmann [W/m$^2$-K$^4$] \\
$\rho$      & Densidad [kg/m$^3$] \\
$\alpha$    & Factor de efectividad del banco de tubos [-]\\
$\gamma_r$  & Factor de radiación externo [-] \\
$\mu_f$     & Viscosidad del fluido [cP] \\
$\mu_g$     & Viscosidad de los gases de combustión [cP] \\
\end{tabular}

\mainmatter
\chapter*{Introducción}

\par En refinerías de petróleo crudo, plantas químicas y petroquímicas, los hornos representan los equipos que suministran más del 80\% de la totalidad de la energía requerida para los procesos de separación y conversión química de los productos. Desde este punto de vista, la eficiencia energética de cada horno es una variable crítica a optimizar, con el fin de reducir el consumo de combustibles quemados, minimizar las emisiones de gases de combustión que, finalmente, se traducen en menores costos operativos [1].

\par Los \texttt{hornos de procesos} juegan un rol primordial en una refinería. Consumen altísimas cantidades de combustibles fósiles ({\tt miles de kilogramos por hora}) y emiten proporcionales cantidades de \ac{co2} a la atmósfera. Son equipos de alto costo que trabajan con llamas a muy altas temperaturas (\textgreater 2000 °C) y cuyas fallas suelen ser críticas para la economía de la refinería. Es importante destacar que el dióxido de carbono atmosférico promedio mundial en 2019 fue de 409,8 ppm, con un rango de incertidumbre de 0,1 ppm. Lo que indica que los niveles de dióxido de carbono resultaron los más altos de los últimos 800.000 años [2]. Teniendo como referencia las cifras del 2018 reportadas por la \ac{ONU}, 55,3 Gt\ac{co2}e (giga toneladas de \ac{co2} equivalente) [3], se puede apreciar el impacto significativo que esto genera sobre el medio ambiente. 

\par Lamentablemente, no es posible automatizar la totalidad de la operación de un horno de procesos [4] puesto que no son equipos completamente herméticos como un intercambiador de calor o una columna de destilación. Esta condición, es decir, estar "abiertos a la atmósfera", hace recaer sobre los operadores de sala de control y de campo (la interfase humano-horno), la responsabilidad última sobre el proceso de combustión y el aprovechamiento del combustible.  

\par Por experiencias prácticas, dentro de las refinerías se ha identificado que los operadores suelen manejar estos equipos con altos niveles de tiro y de exceso de aire, estas variables de operación están alejadas de las condiciones recomendadas al diseñar el horno, lo que conlleva mayores emisiones de CO2 al ambiente y además afecta la integridad mecánica del equipo. Los altos niveles de exceso de aire, provocados por altos niveles de tiro, traen consigo un mayor consumo del combustible para compensar el calor perdido al calentar el aire en exceso, lo que a su vez provoca alteraciones en las características de las llamas haciéndolas incidir directamente sobre los tubos de la sección radiante causando la coquización localizada de la carga dentro de los tubos. 

\par En virtud de la dependencia de los hornos de proceso de la intervención continua y acertada de sus operadores se hace perentorio fomentar su capacitación como una manera de incrementar el desempeño y la eficiencia de estos equipos. 

\par Por esta razón, se ha considerado necesario construir un simulador de hornos de procesos, con base en los estándares API 560 y API RP 535, para su utilización como herramienta práctica en la capacitación del personal operativo de las refinerías de petróleo. Esta herramienta permitiría, de una manera más didáctica, observar e interactuar con las variables fundamentales de control del horno, a saber, el tiro y el exceso de aire, para controlar los efectos de las operaciones alejadas de las condiciones de diseño.

\par Este proyecto utilizará ecuaciones de balance de masa y transferencia de calor para representar las condiciones de operación en un modelo gráfico capaz de responder a diferentes estímulos en las variables. 
\vfill
\par Esta es la versión \version~del documento de tesis para la \ac{USB}, creada y mantenida por:\\
\begin{tabular}{lc}
Br. Esteban Camargo & 11-10145@usb.ve 
\end{tabular}
\input{usodelaclase}
\input{capitulo-4}
\input{capitulo-5}
\chapter*{Conclusiones}

%\begin{itemize}
    %\item 
    \par Se desarrollo exitosamente un modelo capaz de simular la operación de hornos de proceso, enfocado al adiestramiento de ingenieros y operadores, y con el objetivo de aumentar la eficiencia de estos equipos para disminuir el consumo de combustibles fósiles y generar menos gases de efecto invernadero.

    %\item 
    \par El modelo integra las ecuaciones de transferencia de calor y conservación de masa y energía en un algoritmo que simula el comportamiento estacionario de un horno de fuego directo, se comprobó su capacidad para generar resultados estables y confiables de las variables de proceso establecidas.
    
    %\item 
    \par El simulador para adiestramiento de hornos de proceso es una herramienta computacional de disposición pública y de código abierto, escrita en el lenguaje de programación JavaScript, que refiere parámetros confiables y prácticos para el uso académico o industrial dentro de las características mecánicas establecidas. El acceso directo se encuentra siguiendo el enlace (\url{https://e-usb.github.io/heater}).
    
    %\item 
    \par Los resultados generados por el simulador fueron validados aceptablemente mediante su comparación con un software comercial empleado para el diseño y evaluación termomecánica de hornos de proceso. Las tendencias de las variables operacionales simuladas mostraron resultados totalmente coherentes con la operación real de estos hornos en refinerías y plantas petroquímicas.

    %\item 
    \par Se incrementó el alcance del algoritmo desarrollado con la implementación de un modo comparativo y un modo de visualización de tendencias, aumentando las opciones de los usuarios al interactuar con el simulador.
%\end{itemize}
\nocite{*}
\bibliography{referencias}
\appendix
\chapter{Código desarrollado}

La versión organizada por archivos y carpetas se encuentra en el siguiente enlace \url{http://github.com}
\chapter{FIGURAS AMPLIADAS}\label{apx:img}

\begin{figure}[ht]
\includegraphics[scale=0.6,angle=0]{images/diagrama-algo}
\caption[Diagrama Algoritmo Ampliado]{Diagrama detallado del algoritmo desarrollado para la simulación del horno de proceso}\label{img:dia-algo-full}
\end{figure}

\begin{figure}[ht]
\includegraphics[scale=0.6,angle=0]{images/diagrama-meca}
\caption[Diagrama Mecánico Ampliado]{Diagrama descriptivo de la estructura mecánica del horno simulado}\label{img:dia-meca-full}
\end{figure}

\end{document}
