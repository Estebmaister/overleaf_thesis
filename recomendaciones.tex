\chapter*{Recomendaciones}

%\begin{itemize}
    %\item 
    \par Como todo simulador, siempre existe la posibilidad de aumentar su alcance y posibles usos, por ejemplo, se pueden incluir ecuaciones empíricas para calcular la formación de CO y NOx, que permitan a los usuarios tomar estas emisiones en cuenta al variar las condiciones de operación.
    
    %\item 
    \par Expandir las capacidades del simulador al incorporar opciones adicionales para el algoritmo, como el flujo de combustible como una variable de entrada, e incorporar la temperatura de salida del residuo como un resultado de la simulación; añadir el exceso de oxígeno en base seca para los cálculos de combustión por tratarse de una variable medida usualmente con equipos portátiles durante las labores de optimización energética de los hornos.
    
    %\item 
    \par Ampliar las capacidades del simulador incorporando el cálculo del perfil de tiro dentro del horno y la posibilidad de controlarlo mediante, a) el empleo de un accionador hipotético del dámper de la chimenea y b) de los registros de aire de los quemadores.

    %\item 
    \par El acceso a tecnologías libres que permitan desarrollar y continuar perfeccionando la simulación de procesos complejos, es aún muy limitada. De allí, nace la necesidad de fomentar nuevos retos en los currículos universitarios para seguir facilitando los procesos de aprendizajes en la academia y en su posterior uso en la industria.
%\end{itemize}