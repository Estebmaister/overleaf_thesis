\chapter*{Recomendaciones}

\begin{itemize}
    \item El acceso a tecnologías libres que permitan desarrollar y continuar perfeccionando la simulación de procesos complejos, es aún muy limitada. De allí, nace la necesidad de establecer nuevos retos en los currículos de estudio para seguir facilitando los procesos de aprendizajes en la academia y en su posterior uso en la industria.
    
    \item Expandir las capacidades del simulador al incorporar caminos adicionales para el algoritmo, como el uso del flujo de combustible como una variable de entrada y hacer la temperatura de salida del fluido de residuo de vacío un resultado de la simulación.
    
    \item Añadir un camino en el algoritmo para incluir la opción de especificar el exceso de oxígeno en base seca para la combustión, ya que es común obtener este valor directamente de los medidores de oxígeno en hornos.
    
    \item Investigar a profundidad las consideraciones del tiro del horno y su relación con el movimiento del dámper en la chimenea, desarrollando ecuaciones que puedan acoplar estos conceptos con los fenómenos de transferencia de calor en el horno.
\end{itemize}