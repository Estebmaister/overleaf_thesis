\chapter{Análisis de Resultados}

\par En este capitulo se describirá el simulador desarrollado con sus condiciones de diseño, limitaciones, suposiciones, funcionalidades y sus posibles usos y aplicaciones. 

\section{Simulador}

\subsection{Validación del modelo y tendencias}

3-      El modelo debe ser presentado como una herramienta amplia que está limitado por las características físicas del horno: área de transferencia de calor en las tres zonas de intercambio, capacidad de los quemadores y las dimensiones de la chimenea. Es decir, el diseño mecánico del horno es invariable.

4-      El horno   fue diseñado (dimensionado)  con capacidad de calentamiento para 90.000 BPD de residuo  con un incremento de temperatura desde  632 K a 684 K. Condiciones de diseño.

5-      El modelo permite variar el volumen del fluido del proceso, temperatura de entrada y salida del fluido del proceso, condiciones ambientales, composición del gas de combustión y relación A/C. Las propiedades del fluido, Cp, viscosidad y gravedad específica que utiliza el modelo corresponden a la temperatura de entrada y salida y deben ser introducidas como datos. Las propiedades del gas de combustión son calculadas a partir de sus componentes.

6-      No existe impedimento para variar el fluido del proceso en un amplio margen de crudos con diferentes API u otras corrientes de refinería. Esto debe ser recalcado entre los puntos importantes de las virtudes del modelo.

7-      En forma ilustrativa, se debe mantener el fluido del diseño para demostrar la validez del modelo y  la tendencia de los resultados ante la variación de los parámetros independientes.

8-      Deben quedar claro las limitantes del modelo en aquellos elementos no considerados cuantitativamente (ej. formación de CO, despegue de la llama por exceso de aire, etc) y que han sido tratados mediante rangos empíricos.

\subsection{Aplicaciones}