\chapter{Uso de la clase}

\section{Sobre el uso correcto de ciertos comandos}
\subsection{Tablas}
\par Por su parte, las tablas se incluyen como usual asegurandose que la leyenda est\'e ubicada encima de la tabla. El siguiente ejemplo prodice la Tabla \ref{tbl:tabla1}.
\begin{table}
\begin{center}
\caption[T\'itulo corto]{T\'itulo largo de la tabla explicando la misma. La leyenda est\'a ubicado encima para las tablas.}
\label{tbl:tabla1}
\begin{tabular}{rcl}
\hline
Nombre & centrado & apellido\\
\hline
A & B & C \\
Gino & 4 & Lampariello \\
Judith & 6 & Del Terranova\\
\hline
\end{tabular}
\end{center}
\end{table}
\par Al hacer menci\'on a alguna tabla usar la palabra ``Tabla'' (con la primera letra en may\'uscula) seguido de la refencia a la tabla \verb+\ref{fig:tabla1}+. El \'indice de tablass deber ser incluido solamente cuando el n\'umero de tablas es superior a diez.
